%!TEX root = paper.tex

\section{Verification} % (fold)
\label{sec:verification}
From the last step, $\textsc{Zilu}$ has got a predicate which does not only classify all the samples perfectly but get converged.
But this still can not guarantee that predicate is an actual invariant due to the sampling bias compared to the real states distribution of the given program.
For example, if there is a branch in loop body such as $$if (x == 159) \cdots$$, 
our samples would hardly cover this branches except we do sampling amazing right to $(x\mapsto 159)$.
%So we should check all the branches whether our test
To handle this problem, we apply program verification as the last security checkpoint on these ``bad guys''.
%he last security checkpoint on these ``bad guys'' is verification. 
Therefore, in this step, having a learned predicate $\mathcal{C}$, we verify whether it is a valid loop invariant by
checking the satisfiability of constraints (1), (2) and (3) using state-of-art tools from symbolic execution and constraint solving communities.
%In addition, we separated this part as a standalone tool for loop invariant verification.

In implementation, to verify constraint (1) and (3), 
$\textsc{Zilu}$ submits constraints 
$$Pre \Rightarrow \mathcal{C}$$ 
and 
$$\neg {Cond} \wedge \mathcal{C} \Rightarrow Post$$
to Z3 solver for checking satisfiability.
For example, if we get a predicate $\mathcal{C} = \{33-2*x+2*y>=0\}$
for the loop example in the introduction part, 
we submit 
$$(x<y) \Rightarrow (33-2*x+2*y>=0)$$
to check constraint (1), 
and 
$$\big(\neg(x<y) \wedge (33-2\cdot x+2\cdot y>=0)\big) \Rightarrow ((x >= y) \wedge (x <= y + 16))$$
to check constraint (3).

For checking the inductive property, which is formulated as constraint (2),
its complexity depends on the $Body$'s computational complexity.
When the $Body$ contains lots of lines of code, especially with branches, 
it is very hard to encode them into a constraint manually,
not to say verifying each of them.
So we need some tools to help us finding these constraints, 
Otherwise, our technique would limit its power to learn only simple and trivial programs. 
That is why $\textsc{Zilu}$ uses KLEE\cite{cadar2008klee} which is a symbolic execution tool to get path condition.
%and Z3 solver as a representative to verify the learned predicate.

%\subsection{Program Recoding}
%$\textsc{Zilu}$ first separates the given loop program with the learned predicate into three loop-free programs,
%which corresponds to constraints (1), (2) and (3).




\subsection{KLEE}
General speaking, KLEE\cite{cadar2008klee} is a symbolic virtual machine built on top of the LLVM compiler infrastructure
that can enumerate all the possible paths based on target configuration on each of them to get all the path conditions.
In our setting, we are only interested in specific paths which can lead to the assertion failure in the target program.
So we have to modify KLEE source code by adding a new method call to KLEE,
which can trigger KLEE to emit all the paths constraints in which execution can be reached at the given program location.

For example, when $\textsc{Zilu}$ check constraint (2), 
we first construct a loop-free program based on the original program:
\begin{figure}[!h]
\begin{subfigure}{0.20\textwidth}
    \centering
    %\vspace{-0.1cm}
    {\scriptsize\begin{verbatim}
void P(int x, int y) {
    assume(x < y);
    while (x < y) {
        if (x < 0) x = x+7;
        else x = x+10;
    
        if (y < 0) y = y-10;
        else y = y+3; 
    }
    assert(x >= y
        && x <= y+16);
}
    \end{verbatim}}
    \vspace{-5mm}
    \caption{Loop Program}
    \label{fig:running:example:program}
\end{subfigure}%
\begin{subfigure}{.20\textwidth}
      \centering
      \vspace{-0.1cm}
        {\scriptsize\begin{verbatim}
    void P_loopfree(int x, int y) {
        assume(x < y);
        assume(33-2*x+2*y >= 0);
        {
            if (x < 0) x = x+7; //....1
            else x = x+10; //.........2
    
            if (y < 0) y = y-10; //...3
            else y = y+3; //..........4
        }
        assert(33-2*x+2*y >= 0);
    }
    \end{verbatim}}
    \vspace{-5mm}
    \caption{Loop-free program}
      \label{fig:running:example:verification}
\end{subfigure}
\caption{Converting program to check inductive property}
%\vspace{-0.2cm}
\label{fig:running:example}
\end{figure}

After handled by our modified KLEE,
several path constraints are emitted as following, each of them corresponding to paths 
$\{branch2 \to branch4\}$, $\{branch1 \to branch4\}$ and $\{branch1 \to branch3\}$:


\begin{itemize}
\item $\big(33+2\cdot(y-x-7)\ge0\big) \bigwedge \big(x<y\big) \bigwedge \big(33+2\cdot(y-x)\ge0\big) \bigwedge \big(x\ge0\big)$
\item $\big(33+2\cdot(y-x-4)\ge0\big) \bigwedge \big(x<y\big) \bigwedge \big(33+2\cdot(y-x)\ge0\big) \bigwedge \big(x<0\big) \bigwedge \big(y\ge0\big)$
\item $\big(33+2\cdot(y-x-7)\ge0\big) \bigwedge \big(x<y\big) \bigwedge \big(33+2\cdot(y-x)\ge0\big) \bigwedge \big(x<0\big) \bigwedge \big(y<0\big)$
\end{itemize}
As $\alpha \Rightarrow \beta \Longleftrightarrow \neg \big(\alpha \wedge \beta\big)$, 
instead of checking the satisfiability of inference rule
we prefer to check the unsatisfiability of constraints with conjunctive forms. 
After getting these constraints, we can apply Z3 solver to check the satisfiability (or unsatisfiability).

\subsection{Z3 Solver}
The state-of-the-art Satisfiability Modulo Theories (SMT) solvers can be used to check the satisfiability of logical formulas over one or more theories. 
Among these solvers, Z3, from Microsoft Research, is a high-performance theorem prover which is both sound and complete.
In our project, $\textsc{Zilu}$ applies Z3 solver to do the constraints solving for invariant validation. 

So in the implementation, we submit the constraints got by inference rules (1), (3) or by KLEE to Z3 solver.
If all of them passes,we successfully get a valid invariant and also verify the program. 
Otherwise, if any of them is violated, the counterexample obtained is added to the set of sample $X$, 
(named as counter-example sampling in section Sampling)
which is then tested, categorized, used for active learning accordingly in next iteration to help refine our invariant candidate.

%For example, if an invariant candidate $\mathcal{C} = \{33-2*x+2*y >= 0\}$
For the example, $\textsc{Zilu}$ uses Z3 solver to check all the constraints mentioned in the last subsection,
and finds all of them passes, and thus we find a valid loop invariant and have proved the correctness of the loop program.

If the learned predicate is not an actual invariant, i.e. $\mathcal{C} = \{11-2*x+2*y>=0\}$,
after getting the constraints, similar as the constraints in the last part except substituting all the number 33 with number 11, 
Z3 solver could provide a counter-example $s_{ce} = (x \mapsto 1, y \mapsto 0)$ after solving.
We can see that $s_{ce}$ contradict the inductive property of loop invariants (constraints (2)).
Then $\textsc{Zilu}$ use $s_{ce}$ to restart learning in the next iteration, until finding a valid loop invariant or disprove the program.


 
The overall algorithm is presented in Figure~\ref{alg:overall}.
\begin{algorithm}[!h]
\SetAlgoVlined
\Indm
\KwIn{$Pre$, $Cond$, $Body$, $Post$}
\KwOut{an invariant which completes the proof or a counter-example}
\Indp
let $S$ be a set of random samples\;
\While{true} {
    test the program for each sample in $S$\;
    \If {a state $s$ in $\mathcal{S}^x$ is identified} {
        \Return $s$ as a counterexample;
    }
    let $\mathcal{S}^+$, $\mathcal{S}^-$ and $\mathcal{S}^\rightarrow$ be respective sets accordingly\;
    let $\mathcal{C}$ = activeLearning($\mathcal{S}^+$, $\mathcal{S}^-$, $\mathcal{S}^\rightarrow$)\;
    Extract path constraints $\textsc{Pc}$ based on (1)(2)(3)\;
    \For {each $pc$ in $\textsc{Pc}$} {
        \If { $pc$ is not satisfied} {
            add the counter-example into $S$\;
            continue\;
        }
    }
    \Return $\mathcal{C}$ as the proof;
}
\caption{Algorithm $overall$}
\label{alg:overall}
\end{algorithm}

%update $\mathcal{S}^+$, $\mathcal{S}^-$, and $\mathcal{S}^\rightarrow$ accordingly\;

%We remark that we learn three classifiers as candidates for the loop invariant: $U$, $OU$, $O$ such that
%\begin{itemize}
%\item $U$ classifies states in $P$ and those in $N \cup NP$.
%\item $O$ classifies states in $N$ and those in $P \cup NP$.
%\item $OU$ classifies states in $P$ and $N$;
%\end{itemize}
%Intuitively, $U$ would be an under-approximation of $Inv$ (by assuming states in $NP$ does not satisfy $Inv$); 
%$O$ would be an over-approximation of $Inv$ (by assuming states in $NP$ does satisfy $Inv$); 
%and $OU$ would be an safe-approximation of $Inv$ (by using states which we are certain whether they are in $Inv$ or not).



\begin{theorem}
Algorithm $overall$ always eventually terminates and it is correct. \hfill \qed
\end{theorem}


% section verification (end)