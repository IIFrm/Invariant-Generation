%!TEX root = paper.tex

\section{Verification} % (fold)
\label{sec:verification}

In this step, having a learned predicate $\mathcal{C}$, we verify whether it is a valid loop invariant by
checking the satisfiability of constraints (1), (2) and (3) using tools from symbolic execution and constraint solving community.
%In addition, we separated this part as a standalone tools for loop invariant verification.

In implementation, to verify constraint (1) and (3), 
$\textsc{Zilu}$ submits constraints $$\{Pre \Rightarrow \mathcal{C}\}$$ and $$\{\neg {Cond} \wedge \mathcal{C} \Rightarrow Post\}$$ to Z3 solver for checking satisfiability.
For example, if we get a predicate $\mathcal{C} = \{33-2*x+2*y>=0\}$
for the loop example in the introduction part, 
we submit 
$$\{(x<y) \Rightarrow (33-2*x+2*y>=0)\}$$
to check constraint (1), 
and 
$$\{\neg(x<y) \wedge (33-2*x+2*y>=0) \Rightarrow ((x >= y \wedge x <= y + 16))\}$$
to check constraint (3).

For checking constraint (2), it might be a bit of complicated due to $Body$'s complexity.
That is why $\textsc{Zilu}$ uses KLEE\cite{cadar2008klee} as a symbolic execution tool to get path condition.
%and Z3 solver as a representative to verify the learned predicate.

%\subsection{Program Recoding}
%$\textsc{Zilu}$ first separates the given loop program with the learned predicate into three loop-free programs,
%which corresponds to constraints (1), (2) and (3).




\subsection{KLEE}
General speaking, KLEE\cite{cadar2008klee} is a symbolic virtual machine built on top of the LLVM compiler infrastructure
that can enumerate all the possible paths based on target configuration on each of them to get all the path conditions.
In our setting, we are only interested in specific paths which can lead to assertion failure in the target program.
So we have to modify KLEE source code by adding a new method call to KLEE,
which can trigger KLEE to emit all the paths constraints in which execution can be reached at given program location.

For example, when $\textsc{Zilu}$ check constraint (2), 
we first construct a loop-free program based on the original program:
\begin{figure}[!h]
\begin{subfigure}{0.20\textwidth}
    \centering
    %\vspace{-0.1cm}
    {\scriptsize\begin{verbatim}
void P(int x, int y) {
    assume(x < y);
    while (x < y) {
        if (x < 0) x = x+7;
        else x = x+10;
    
        if (y < 0) y = y-10;
        else y = y+3; 
    }
    assert(x >= y
        && x <= y+16);
}
    \end{verbatim}}
    \vspace{-5mm}
    \caption{Loop Program}
    \label{fig:running:example:program}
\end{subfigure}%
\begin{subfigure}{.20\textwidth}
      \centering
      \vspace{-0.1cm}
        {\scriptsize\begin{verbatim}
    void P_loopfree(int x, int y) {
        assume(x < y);
        assume(33-2*x+2*y >= 0;
        {
            if (x < 0) x = x+7; //....1
            else x = x+10; //.........2
    
            if (y < 0) y = y-10; //...3
            else y = y+3; //..........4
        }
        assert(33-2*x+2*y >= 0);
    }
    \end{verbatim}}
    \vspace{-5mm}
    \caption{Loop-free program}
      \label{fig:running:example:verification}
\end{subfigure}
\caption{Converting program to check inductive property}
%\vspace{-0.2cm}
\label{fig:running:example}
\end{figure}

After handled by our modified KLEE,
several path constraints are emitted as following, each of which corresponds to path 
$\{branch2 \to branch4\}$, $\{branch1 \to branch4\}$ and $\{branch1 \to branch3\}$:


\begin{itemize}
\item $33+2*(y-x-7)\ge0 \mathbf{\wedge} x<y \wedge 33+2*(y-x)\ge0 \wedge x\ge0$
\item $33+2*(y-x-4)\ge0 \wedge x<y \wedge 33+2*(y-x)\ge0 \wedge x<0 \wedge y\ge0$
\item $33+2*(y-x-7)\ge0 \wedge x<y \wedge 33+2*(y-x)\ge0 \wedge x<0 \wedge y<0$
\end{itemize}
As $\alpha \Rightarrow \beta \Longleftrightarrow \neg (\alpha \wedge \beta)\}$, 
instead of checking the satisfiability of inference rule
we prefer to check the unsatisfiability of constraints with conjunctive forms. 
After getting these constraints, we can apply Z3 solver to check the satisfiability (or unsatisfiability).

\subsection{Z3 Solver}
The state-of-the art Satisfiability Modulo Theories (SMT) solvers can be used to check the satisfiability of logical formulas over one or more theories. 
Among these solvers, Z3 is a high-performance theorem prover, that is sound and complete, from Microsoft Research.
In our project, $\textsc{Zilu}$ applies Z3 to do the constraint solving for invariant validation. 

So in the implementation, we submit the constraints got by inference rules (1), (3) or by KLEE to Z3 solver.
If all of them passes,we successfully get a valid invariant and also verify the program. 
Otherwise, if any of them is violated, the counterexample obtained is added to the set of sample $X$, 
(named as counter-example sampling in section Sampling)
which is then tested, categorized, used for active learning accordingly in next iteration to help refine our invariant candidate.




 
The overall algorithm is presented in Figure~\ref{alg:overall}.
\begin{algorithm}[!h]
\SetAlgoVlined
\Indm
\KwIn{$Pre$, $Cond$, $Body$, $Post$}
\KwOut{an invariant which completes the proof or a counterexample}
\Indp
let $S$ be a set of random samples\;
\While{true} {
    test the program for each sample in $S$\;
    \If {a state $s$ in $C$ is identified} {
        \Return $s$ as a counterexample;
    }
    let $P$, $N$ and $U$ be the respective sets accordingly\;
    %let $Inv_u = activeLearning(P, N \cup NP)$\;
    %let $Inv_o = activeLearning(P \cup NP, N)$\;
    let $Inv$ = ClassificationAlgorithm($P$, $N$)\;
    \If {$Inv$ not converged} {
        add selectiveSampling($Inv$) into $S$\;
        continue\;
    }
    %\For {each $Inv$ in $\{Inv_u, Inv_o, Inv_s\}$} {
    Extract path constraints$PCs$ based on (1)(2)(3)\;
    \For {each $pc$ in $PCs$} {
        \If { $pc$ is not satisfied} {
            add the counterexample into $S$\;
            continue\;
        }
    }
        %\Else {
    \Return $Inv$ as the proof;
        %}
    
}
\caption{Algorithm $overall$}
\label{alg:overall}
\end{algorithm}


%We remark that we learn three classifiers as candidates for the loop invariant: $U$, $OU$, $O$ such that
%\begin{itemize}
%\item $U$ classifies states in $P$ and those in $N \cup NP$.
%\item $O$ classifies states in $N$ and those in $P \cup NP$.
%\item $OU$ classifies states in $P$ and $N$;
%\end{itemize}
%Intuitively, $U$ would be an under-approximation of $Inv$ (by assuming states in $NP$ does not satisfy $Inv$); 
%$O$ would be an over-approximation of $Inv$ (by assuming states in $NP$ does satisfy $Inv$); 
%and $OU$ would be an safe-approximation of $Inv$ (by using states which we are certain whether they are in $Inv$ or not).
\begin{example}
\end{example}


\begin{theorem}
Algorithm $overall$ always eventually terminates and it is correct. \hfill \qed
\end{theorem}


% section verification (end)