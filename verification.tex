%!TEX root = paper.tex

\section{Verification} % (fold)


In this section, we present the technique to verify the given program with the learnt candidate
and thus prove the candidate is a real loop invariant.
Otherwise, a counter-example will be provided.
% from which can drive the new learning attempt.
%From the last step, $\textsc{Zilu}$ has got a predicate which does not only classify all the samples perfectly but get converged.
%But this still can not guarantee that predicate is an actual invariant due to the sampling bias compared to the real states distribution of the given program.
%For example, if there is a branch in loop body such as $$if (x == 159) \cdots.$$
%our samples would hardly cover this branches except we do sampling amazing right to $(x\mapsto 159)$.
%So we should check all the branches whether our test
%To handle this problem, we apply program verification as the last security checkpoint on these ``bad guys''.
%he last security checkpoint on these ``bad guys'' is verification.
Mathematically, having a learnt predicate $\mathcal{C}$, we verify whether it is a valid loop invariant through
checking the satisfiability of the three constraints (\ref{inv:pre}), (\ref{inv:loop}) and (\ref{inv:post})
using state-of-the-art tools from symbolic execution and constraint solving communities.
%In addition, we separated this part as a standalone tool for loop invariant verification.

%\LL{Make clear references to the example, rather than ask the readers to think and guess.}
In our framework, to verify constraint (\ref{inv:pre}) and (\ref{inv:post}),
we submit constraints $Pre \Rightarrow \mathcal{C}$
and $\neg {Cond} \wedge \mathcal{C} \Rightarrow Post$
to Z3~\cite{de2008z3} solver for checking satisfiability.
For example, if we get a predicate $\mathcal{C} = \{33-2*x+2*y>=0\}$
for the loop example in the introduction part,
we submit
$$(x<y) \Rightarrow (33-2*x+2*y>=0)$$
to check constraint (1), and
$$\big(\neg(x<y) \wedge (33-2\cdot x+2\cdot y>=0)\big) \Rightarrow ((x >= y) \wedge (x <= y + 16))$$
to check constraint (3).

For checking the inductive property, which is formulated as constraint~\ref{inv:loop},
its complexity depends on the $Body$'s computational complexity.
When the $Body$ contains numbers of lines of code, especially with branches,
it is very hard to encode them into a constraint manually.
%not to say verifying each of them.
%So tools from symbolic execution community are needed to help us finding these constraints.
%\LL{Do not used `community'.
%Just say,
So we adopt symbolic execution tools to execute the program loop body
and check the satisfaction of the property.
%(I think symbolic testing is still different from concolic testing,
%so you may consider say something about it.)}
Otherwise, our technique would limit its power to learn only simple and trivial programs.
In $\textsc{Zilu}$, it takes KLEE~\cite{cadar2008klee} as a symbolic execution tool to get path constraints.
%and Z3 solver as a representative to verify the learned predicate.

%\subsection{Program Recoding}
%$\textsc{Zilu}$ first separates the given loop program with the learned predicate into three loop-free programs,
%which corresponds to constraints (1), (2) and (3).




\subsection{KLEE}
\label{subsec:klee}
KLEE is a symbolic virtual machine built on top of the LLVM compiler infrastructure
that can enumerate all the possible paths based on the target configuration.
In our setting, only specific paths which can lead to the assertion failure are interested.
So we get a modified version of KLEE by adding a new native method call
which can trigger KLEE to emit all the execution paths that can be reached at the given program location.

For example, when \textsc{Zilu} checks constraint~\ref{inv:loop},
it first takes $Body$ and substitute other elements in the original program with the elements in constraint~\ref{inv:loop}.
Figure~\ref{fig:klee:program} shows this procedure,
with left part original program is shown in Figure.~\ref{fig:klee:program:in}.
right part a converted loop-free program for being further proceeded by KLEE.

\begin{figure}[!h]
\begin{subfigure}{0.19\textwidth}
    \centering
    %\vspace{-0.1cm}
    {\scriptsize\begin{verbatim}
void P(int x, int y) {
    assume(x < y);
    while (x < y) {
        if (x < 0) x = x+7;
        else x = x+10;

        if (y < 0) y = y-10;
        else y = y+3;
    }
    assert(x >= y
        && x <= y+16);
}
    \end{verbatim}}
    \vspace{-5mm}
    \caption{Loop Program}
    \label{fig:klee:program:in}
\end{subfigure}%
\begin{subfigure}{.19\textwidth}
      \centering
      \vspace{-0.1cm}
        {\scriptsize\begin{verbatim}
    void P_loopfree(int x, int y) {
        assume(x < y);
        assume(33-2*x+2*y >= 0);
        {
            if (x < 0) x = x+7; //....1
            else x = x+10; //.........2

            if (y < 0) y = y-10; //...3
            else y = y+3; //..........4
        }
        assert(33-2*x+2*y >= 0);
    }
    \end{verbatim}}
    \vspace{-5mm}
    \caption{Loop-free program}
      \label{fig:klee:program:out}
\end{subfigure}
\caption{Converting program to check inductive property}
%\vspace{-0.2cm}
\label{fig:klee:program}
\end{figure}

\LL{Same here.}
Several path constraints are provided as follows after the program is executed by KLEE.
(As $\alpha \Rightarrow \beta \Longleftrightarrow \neg \big(\alpha \wedge \beta\big)$,
KLEE returns constraints with conjunctive forms, instead of inference rules.)
The listed constraints corresponds to the execution path in the program as expected:
the first item encodes the execution path that takes $branch2 \to branch4$;
the second item encodes the execution path that takes $branch1 \to branch4$;
the last item encodes the execution path that takes $branch1 \to branch3$.
% $\{branch1 \to branch4\}$ and $\{branch1 \to branch3\}$.

\begin{itemize}
\item $\big(33+2\cdot(y-x-7)\ge0\big) \bigwedge \big(x<y\big) \bigwedge \big(33+2\cdot(y-x)\ge0\big) \bigwedge \big(x\ge0\big)$
\item $\big(33+2\cdot(y-x-4)\ge0\big) \bigwedge \big(x<y\big) \bigwedge \big(33+2\cdot(y-x)\ge0\big) \bigwedge \big(x<0\big) \bigwedge \big(y\ge0\big)$
\item $\big(33+2\cdot(y-x-7)\ge0\big) \bigwedge \big(x<y\big) \bigwedge \big(33+2\cdot(y-x)\ge0\big) \bigwedge \big(x<0\big) \bigwedge \big(y<0\big)$
\end{itemize}



%After getting these constraints, we can apply Z3~\cite{de2008z3} solver to check the satisfiability (or unsatisfiability).

\subsection{Z3}
\label{subsec:z3}

%\LL{Say we adopt SMT solvers to solve our problem.}
We adopt Satisfiability Modulo Theories(SMT) \cite{barrett2009satisfiability} solvers to solve the obtained constraints
%The state-of-the-art SMT ()\cite{barrett2009satisfiability}  solvers can be used to
Generally, SMT solver check the satisfiability of logical formulas over one or more theories.
Among all these solvers, Z3\cite{de2008z3}, from Microsoft Research, is a high-performance theorem prover which is both sound and complete.

So in the implementation, $\textsc{Zilu}$ applies Z3 solver to do the constraints solving for the invariant validation task.
It submits the constraints got from inference rules (\ref{inv:pre}) and (\ref{inv:post}) or path contraints returned by KLEE to Z3 solver.
If all of them passes, we successfully get a valid invariant and also verify the program.
Otherwise, if any of them is violated, the counter-example obtained is added to the set of samples $\mathcal{S}$,
%then \textsc{Zilu} starts a new learning iteration from counter-example sampling (Section ~\ref{sec:sampling}).
which is then tested, categorized, used (as counter-example sampling in Section ~\ref{sec:sampling}) for active learning accordingly
in next iteration to help refine our invariant candidate.

%For example, if an invariant candidate $\mathcal{C} = \{33-2*x+2*y >= 0\}$
\LL{This does not only applies to the loop example, this is your general approach.}
For the loop example, $\textsc{Zilu}$ uses Z3 solver to check all the constraints mentioned in the last subsection,
and finds all of them passes, and thus we find a valid loop invariant and have proved the correctness of the loop program.

If the learned predicate is not an actual invariant, i.e. $\mathcal{C} = \{11-2*x+2*y>=0\}$,
after getting the constraints, similar as the constraints in the last part except substituting all the number 33 with number 11,
Z3 solver provides a counter-example, i.e. $s^{\chi} = (x \mapsto 1, y \mapsto 0)$, after solving.
Apparently, $s^{\chi}$ contradicts the inductive property of loop invariants (constraints~\ref{inv:loop}),
and then can be used to restart learning in the next iteration, until finding a valid loop invariant or disprove the program.




%update $\mathcal{S}^+$, $\mathcal{S}^-$, and $\mathcal{S}^\rightarrow$ accordingly\;

%We remark that we learn three classifiers as candidates for the loop invariant: $U$, $OU$, $O$ such that
%\begin{itemize}
%\item $U$ classifies states in $P$ and those in $N \cup NP$.
%\item $O$ classifies states in $N$ and those in $P \cup NP$.
%\item $OU$ classifies states in $P$ and $N$;
%\end{itemize}
%Intuitively, $U$ would be an under-approximation of $Inv$ (by assuming states in $NP$ does not satisfy $Inv$);
%$O$ would be an over-approximation of $Inv$ (by assuming states in $NP$ does satisfy $Inv$);
%and $OU$ would be an safe-approximation of $Inv$ (by using states which we are certain whether they are in $Inv$ or not).

% section verification (end)
