%!TEX root = paper.tex

\section{Classification} % (fold)
\label{sec:classification}

After sampling and labeling, we obtain some program states must be in $Inv$ and some must not. 
Thus, any candidate invariant must be able to perfectly classify these states. 
We apply classification techniques from the machine learning community to obtain classifiers as candidate invariants.
Due to different technique can results in different forms of classifiers,
we apply SVM and some of its derivatives on our training data. 

\subsection{Linear SVM}
\subsection{Polynomial SVM}
\subsection{Conjunctive SVM}
In the following, we present how we obtain a classifier automatically using SVM. 
SVM is a supervised machine learning algorithm for classification and regression analysis. 
We use its binary classification functionality. 
Mathematically, the binary classification functionality of (linear) SVM works as follows. 
Given two sets of feature vectors $F^+$ and $F^-$, it generates, if there is any, 
a linear constraint in the form of $ax + by + \cdots \geq d$ where $x$ and $y$ are feature values and $a, b, d$ are constants, 
such that every state $s \in F^+$ satisfies the constraint and every state $s' \in F^-$ fails the constraint. 
In this work, we always choose the \textit{optimal margin classifier} (see the definition in~\cite{Sharma2012}) if possible. 
This half space could be seen as the strongest witness why the two data states are different. 
In the following, we write $svm(F^+, F^-)$ to denote the function which returns a linear classifier

If, however, $F^+$ and $F^-$ cannot be perfectly classified by one half space only, 
a more complicated function $f$ must be adopted. 
For instance, if there is a classifier in the form of conjunctive of multiple half spaces, 
the algorithm presented in~\cite{Sharma2012} can be used to identify such a classifier.

%\section{Active Learning}
%Due to the limited set of samples we have (which is often referred to as labeled samples in the machine learning community), 
%the guessed classifier obtained from previous iteration might be far from being correct. 
%In fact, without labeled samples which are right on the boundary of the `actual' classifier, 
%it is very unlikely that we would find it. 
%Intuitively, in order to get the `actual' classifier, we would require samples which would distinguish the actual one from any nearby one. 
%This problem has been discussed and addressed in the machine learning community using active learning and selective sampling~\cite{DBLP:conf/icml/SchohnC00}.

%The concept of active learning or selective sampling refers to the approaches 
%that aim at reducing the labeling effort by selecting only the most informative samples to be labeled. 
%SVM selective sampling techniques have been proven effective in achieving a high accuracy 
%with fewer examples in many applications~\cite{DBLP:conf/mm/TongC01,DBLP:journals/jmlr/TongK01}. 
%The basic idea of selective sampling is that at each round, 
%we select the samples that are the closest to the classification boundary so that they are the most difficult to classify and the most informative to label. 
%Since an SVM classification function is represented by support vectors which are the samples closest to the boundary, 
%this selective sampling effectively learns an accurate function with fewer labeled data~\cite{DBLP:conf/icml/SchohnC00}. 
%In our setting, this means that we should sample a program state right by the classifier and test the program 
%with that state to label that feature vector so that the classifier would be improved.

\begin{algorithm}[b]
\SetAlgoVlined
\Indm
\KwIn{$F^+$ and $F^-$}
\KwOut{a classifier for $F^+$ and $F^-$}
\Indp
let $old$ be $null$\;
\While{true} {
    let $f = classify(F^+, F^-)$\;
    \If {$f$ is identical to $old$} {
        \Return $f$;
    }
    let $old = f$\;
    let $sam$ be a set of samples computed by selective sampling\;
    test the program and update $F^+$ and $F^-$ accordingly\;
}
\caption{Algorithm $activeLearning$}
\label{alg:active}
\end{algorithm}

Algorithm~\ref{alg:active} presents details on how active learning is implemented in \textsc{Zilu}. 
At line 2, we obtain a classifier based on Algorithm~\ref{classify}. 
We compare the newly obtained classifier with the previous one at line 4, if they are identical, we return the classifier; 
otherwise we apply selective sampling so that we can generate additional labeled samples for improving the classifier. 
In particular, at line 5, we apply standard techniques~\cite{DBLP:conf/icml/SchohnC00} to select the most informative sample. 
Notice that in our setting, the most informative samples are those which are exactly on the lines and therefore can be obtained by solving an equation system. 
At line 8, we test the program with the newly generated samples so as to label them accordingly.

In our implementation, after getting a classifier, which is usually a single polynomials or conjunction or disjunction of polynomials,
we can get some solutions of these polynomials using \textbf{GSL} (GNU Scientific Library).
Then we use these solutions or the points near these solutions as sampling points.
% section classification (end)
