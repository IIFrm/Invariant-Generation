%\section{Location-based Disjunctive Invariant Learning}
\section{Classifier Learning}
In the machine learning community, many classification algorithms have been proposed, 
e.g., perceptron~\cite{perceptron}, decision tree~\cite{quinlan1986induction}, $\mathit{SVM}$~\cite{svm:original} and neutral network~\cite{nn}.
%Among plenty of classification approaches, \textsc{Zilu} takes Supported Vector Machines as a primary approach.
%Before demonstrating the specific technique applied in \textsc{Zilu},
%it should be noted that there are at least two main differences between machine learning problems and the problem in our setting:
In our framework, the classification algorithms must generate perfect classifiers. 
Formally, a perfect classifier $\phi$ for two sets of samples $P$ and $N$ is a predicate such that $s \models \phi$ for all $\mathit{s \in P}$ and $\mathit{s \not \models \phi}$ for all $\mathit{s \in N}$. 
Furthermore, the classification algorithms must generate classifiers which are human-interpretable or can be handled by existing program verification techniques. 
In the following, we present two such classification algorithms based on Support Vector Machine ($\mathit{SVM}$). 
We remark that our framework is not restricted to $\mathit{SVM}$-based classification algorithms.
%But there are still differences between the classification problem in machine learning field and the one in our framework:
%\begin{itemize}
%\item Our framwork need to learn an classifier with explicit form which is readable by human and can be proceeded by off-the-shelf constraints solvers.
%However, the current machine learning techniques always produce an implicit classifier that is used to do prediction.
%So we need to convert the implicit classifier to an explicit one, which will be used in later stage.
%%Most machine learning techniques values more on prediction correctness rather than understanding the underlying problem. \LL{What is the difference?}
%%although there is a research trend in machine learning to understand ongoing during these years.
%%However, in the software engineering area, program verification cares about formalization and reasoning.
%%Therefore, a classification technique from machine learning area would be highly suggested as long as it can predict correctly on newly received data.
%%But in our setting, we need an explicit classifier which is readable by human and can be proceeded by off-the-shelf constraints solvers.
%\item
%%Machine learning algorithms consider more of generalization ability than classification correctness on the training data.
%%Thus they would \LL{sacrifice the prediction correctness on the training data} for exchanging the higher generalization ability.  \LL{I feel our method is incorrect after reading this. Be precise. }
%In our context, the classification accuracy on the training data is of vital importance,
%and even a slight classification error can not be tolerated.
%But some machine learning algorithms would partially sacrifice the classification correctness on the training data
%for exchanging a higher prediction accuracy on the unseen data,
%as they values the generalization ability most.
%%On this basis, we hope the learned classifier can reflect the underlying logic of the given program.
%% classification algorithm can have better generalization ability.
%%\end{itemize}
%Considering these differences,
%in the following text,
%we show our classification approaches which extend the native SVM to learning the invariant candidate.
%\LL{I understand your meaning, but maybe say, we thus extend svm to capture ...
%Do not describe methods in a vague manner. }


%\subsection{Support Vector Machines}
%\label{subsec:svm}
%, is one of the most powerful approaches.
%SVM is a supervised learning model with associated learning algorithms that analyze data used for classification.
%In most setting, given a set of training examples, each marked as belonging to one of two categories,
%an SVM training algorithm builds a model, which is a representation of the examples as points in space,
%that can assign new examples into one category or the other,
%making it a non-probabilistic binary linear classifier.
\subsection{Linear Classifier}
$\mathit{SVM}$ is a supervised machine learning algorithm for classification and regression analysis~\cite{svm:original}. 
In this work, we use its binary classification functionality.
In general, the binary classification functionality of (linear) $\textsc{Svm}$ works as follows.
Given two sets of samples $P$ and $N$, $\mathit{SVM}$ generates a perfect linear classifier to separate them if there is any.
In the following, we write $\mathit{svm}(P, N)$ to denote the function which returns a perfect linear classifier for $P$ and $N$ if there is any; 
or returns $\textsc{null}$ otherwise.
We refer the readers to~\cite{svm:smo} for details on how the classifier is computed.
In this work, we always choose the \textit{optimal margin classifier} (see the definition in~\cite{sharma2012interpolants}) if possible.
Intuitively, the optimal margin classifier could be seen as the strongest witness why $P$ and $N$ are different.
$\mathit{SVM}$ by default learns only classifiers in the form of a linear inequality (a.k.a.~a half space),
e.g., in the form of $\mathit{a x + b y \geq c}$ where $x$ and $y$ are variables whereas $a$, $b$, and $c$ are constants.
In practice, such classifiers may not be sufficient and thus, we develop algorithms to learn more expressive invariants.

\subsection{Polynomial Classifiers} Algorithm~\ref{alg:conjunctiveSVM} shows our algorithm for learning classifiers in the form of polynomial inequalities.
The inputs of the algorithm include the two sets of samples $P$ and $N$ as well as a threshold on the maximum degree of the polynomial classifier.
During the loop from line 2 to 7, we first map all the samples in $P$ (similarly $N$) to a set of samples $P'$ (similarly $N'$) in a high dimensional space using function $\mathit{mapToDegree}$.
Given a set of samples $X$ and a target degree, function $\mathit{mapToDegree}$ returns a set of new samples $X'$ such that each element in $X$ is mapped to an element in $X'$.
For instance, let $\{x \mapsto 2, y \mapsto 1\}$ be an element in $X$ and the target $\mathit{degree}$ be 2, 
the following element is added to $X'$: $\{x \mapsto 2, y \mapsto 1, x^2 \mapsto 4, xy \mapsto 2, y^2 \mapsto 1\}$.
That is, given a valuation in $P$, we systematically add all terms constituted by the same variables with a degree no more than $\mathit{degree}$, 
as if they are new variables.
%The primary idea is simple, after mapping raw data (program states) from original space in $\mathcal{S}^+$, $\mathcal{S}^-$ and $\mathcal{S}^\rightarrow$ to a high dimensions,
$\mathit{SVM}$ is then applied to learn a perfect linear classifier for $P'$ and $N'$. % as is shown in~\ref{subsec:svm}.
Mathematically, a linear classifier in the high dimensional space is the same as a polynomial classifier in the original space~\cite{svm:kernel}.
In order to favor classifiers with lower degrees (which often are easier to understand or verify), 
we start with looking for a polynomial classifier with degree 1 and only look for higher-degree polynomial classifiers if we find no lower-degree ones.
The algorithm terminates when we reach the threshold on the maximum degree.
We remark that the size of each sample in $P'$ or $N'$ grows rapidly with $\mathit{degree}$. 
%The number of monomials of $\mathit{degree}$ in $\mathit{n}$ is $(^{n+degree-1}_{degree})$.
However, in our experiments, we can often identify useful classifiers with a small value for $\mathit{degree}$.
We remark that the polynomial classifiers learnt by Algorithm~\ref{alg:polynomialSVM} can represent classifiers in the form of disjunctive or conjunctive linear inequalities.
%Because some invariants of conjunctive or disjunctive form can be expressed in polynomials equivalently,
For instance, the classifier $\big((x \ge d_0) \wedge (x \le d_1)\big) \vee (x \ge d_2)$
where $\mathit{d_0 < d_1 < d_2}$ are constants can be represented as: $\mathit{x^3 + (d_0d_1 + d_0d_2 + d_1d_2)x^2 - (d_0 + d_1 + d_2)x - d_0d_1d_2 \geq 0}$.

\begin{algorithm}[t]
\SetAlgoVlined
\Indm
\Indp
    Initialize $degree$ as 1\;
    \While {$degree \le \mathit{max\_degree}$} {
        $P' := \mathit{mapToDegree}(P, \mathit{degree})$\;
        $N' := \mathit{mapToDegree}(N, \mathit{degree})$\;
        \If {$\mathit{svm}(P', N')$ is not $\textsc{null}$} {
            \Return $\mathit{svm}(P', N')$\;
        }
    	$\mathit{degree} := \mathit{degree} + 1$\;
    }
    \Return $\textsc{null}$;
\caption{Algorithm $\mathit{polynomial}(P,N)$}
\label{alg:polynomialSVM}
\end{algorithm}
\vspace{-2mm}


%The whole classification algorithm is described in Algorithm~\ref{alg:classify}.
%Note that,
%$\textsc{Svm}$ in this algorithm can be substitute with other classification techniques for different learning purposes.
%We will take two classification algorithms as examples to demonstrate this promotion in \LL{?}~\ref{subsec:svm:derivatives}.
%\begin{algorithm}[!h]
%\SetAlgoVlined
%\Indm
%\KwIn{$\mathcal{S}^+$, $\mathcal{S}^-$, and $\mathcal{S}^\rightarrow$}
%\KwOut{$\textsc{Null}$ or a perfect classifier for $\mathcal{S}^+$ and $\mathcal{S}^-$ without violating $\mathcal{S}^\rightarrow$}
%\Indp
%    let $f$ = $\textsc{Svm}$($\mathcal{S}^+$, $\mathcal{S}^-)$\;
%    \If {$f$ violates any data in $\mathcal{S}^+$ or $\mathcal{S}^-$} {
%        \Return $\textsc{Null}$\;
%    }
%    \If {$f$ violates any inference rules in $\mathcal{S}^\rightarrow$} {
%        \Return $\textsc{Null}$\;
%    }
%    \Return $f$;
%\caption{Algorithm $classify$}
%\label{alg:classify}
%\end{algorithm}



%In the following, we present how we obtain a classifier automatically using $\textsc{Svm}$.
%In this work, we always choose the \textit{optimal margin classifier} (see the definition in~\cite{Sharma2012}) if possible.
%This half space could be seen as the strongest witness why the two data states are different.
%In the following, we write $svm(S^+, S^-)$ to denote the function which returns a linear classifier

%\subsection{Checking}
%With the learned $\textsc{Svm}$ model,we check whether it can perfectly classify these states first.
%If yes, then we can automatically turn it back to a hyperplane form, which is regarded as our invariant candidate.
%Otherwise, we may apply other classification techniques for learning, which will be mentioned at ``$\textsc{Svm}$ derivatives'' part in the end of this section.


%\subsection{SVM Derivatives}
%\label{subsec:svm:derivatives}
%%$\mathcal{S}^+$, $\mathcal{S}^-$, and $\mathcal{S}^\rightarrow$
%If $\mathcal{S}^+$, $\mathcal{S}^-$ cannot be classified with 100\% accuary by one half-space only,
%a more complicated function $f$ must be adopted.
%For instance, there has been research on invariants in form of conjunctives~\cite{sharma2012interpolants},
%octagonal~\cite{mine2006octagon}, tree form~\cite{krishna2015learning}\cite{garg2015learning} and so on.
%
%%For instance, if there is a classifier in the form of conjunctive of multiple half spaces,
%%the algorithm presented in~\cite{sharma2012interpolants} can be used to identify such a classifier.
%
%Moreover, as is noted Algorithm~\ref{alg:classify} can be easily extended by substituting $\textsc{Svm}$ with other classification approaches,
%\textsc{Zilu} has implemented two classification algorithms by extending the native $\textsc{Svm}$:
%$Polynomial \textsc{Svm}$ for learning invariants in the form of polynomials or any equivalent expressions,
%and $Conjunctive$\textsc{Svm} for learning invariants in the form of conjunctives.
%
%\medskip\noindent
%\textbf{Polynomial SVM.}
%In previous research, several papers ~\cite{**} have studied invariants with conjunctive form or disjunctive form.
%However, there is still no efficient approach to learning these invariants.
%In our research, we found sometimes convert the conjunctive or disjunctives to a polynomial expression might be a nice try to this problem.
%In the real work, there are not only linear invariants but invariants of many other forms,
%such as, conjunctives~\cite{sharma2012interpolants}, octagonal~\cite{mine2006octagon}, polynomial, and tree form~\cite{krishna2015learning}\cite{garg2015learning}.
%In this part, we present our classification approach based on primitive \textsc{Svm} for learning polynomial invariants,
%which have not been discussed by previous research.
%In this part, we would like to learn this kind of invariants.
%Apparently methods that use linear template or linear classification algorithm do not work.
%Actually, polynomials are more powerful on this than they look at the first glance, especially univariate polynomials.
%Some cubic univariate polynomials can represent disjunctive of a conjunctive expression and a linear expression.
%For example, the following two expressions are equivalent:
%$$\big(x \ge x_0 \bigwedge x \le x_1) \bigvee x \ge x_2\big) \ where\ x_0 < x_1 < x_2$$
%$$x^3 + (x_0x_1 + x_0x_2 + x_1x_2)x^2 - (x_0 + x_1 + x_2)x - x_0x_1x_2 >= 0$$
%This leads us to develop a classification algorithm for learning polynomial divider.
%
%The whole procedure is shown in algorithm~\ref{alg:polynomialSVM}.
%In practice, \textsc{Zilu} provide polynomials up to degree 4 as we think degree 4 can cover most hackneyed invariants for now.


%\begin{align}
%    x^3 + a\dot x^2 - b\dot x - c &>= &0 \\
%    where a &= & x_0 \dot x_1 + x_0 \dot x_2 + x_1 \dot x_2 \\
%\end{align}
%For instance, if the target invariant is
%$$(x \ge x_0) \vee (x \le x_1) \ where\ x_0 < x_1$$
%we can find an equivalent polynomial expression:
%$$x^2 - (x_0 + x_1)x + x_0x_1 \ge 0$$
%So $polynomialSVM$ can be used to learn a polynomial classifier or any complex expression as long as it can be expressed in form of polynomials.

\subsection{Conjunctive Polynomial Classifiers}
As discussed above, polynomial classifiers can equivalently represent certain conjunction or disjunction of linear classifiers. It is however not always possible, i.e., some conjunctive or disjunctive linear inequalities cannot be expressed in the form of a polynomial classifier. A simple example is $\mathit{x \ge 0 \land y \ge 0}$. Thus, we develop an algorithm to learn more expressive classifiers.
\begin{algorithm}[!h]
\SetAlgoVlined
\Indm
\Indp
    Initialize set $C$\ as an empty set $\{\}$, set $\mathit{Misclassified}$\ as\ $N$\;
    \While {$\mathit{Misclassified}$ is not empty} {
        Random choose $s$ from $\mathit{Misclassified}$\;
        $f := \mathit{polynomial}(P, \{s\})$\;
        Add $f$ into $C$\;
        \For {$s' \in \mathit{Misclassified}$} {\
            \If {$f(s') < 0$} {
                remove $s'$ from $\mathit{Misclassified}$\;
            }
        }
    }
    Minimize $C$\;
    \Return the conjunction of all predicates in $C$;
\caption{Algorithm $\mathit{conjunctivePoly}(P, N)$}
\label{alg:conjunctiveSVM}
\end{algorithm}
Algorithm~\ref{alg:conjunctiveSVM} shows a classification algorithm which is designed to learn conjunctive polynomial classifiers.
%So in this part, we introduce the algorithm to learn conjunctive invariants directly.
%, avoiding the tries to convert them into a form of polynomials.
It is inspired by the algorithm presented in~\cite{sharma2012interpolants}. The difference is that~\cite{sharma2012interpolants} learns conjunctive linear classifiers, whereas we learn conjunctive polynomial classifiers. The idea is to pick one sample $s$ from $N$ (at line 3) each time and identify a polynomial classifier based on Algorithm~\ref{alg:polynomialSVM} separating $P$ and $\{s\}$ (a.k.a.~a clause). Next, we remove (at line 6 to 8) all samples from $N$ which can be correctly classified by the polynomial classifier found at line 4, and then repeat the process until $N$ becomes empty.
%Compared to the classification algorithm presented in~\cite{sharma2012interpolants}, Algorithm~\ref{alg:conjunctiveSVM} learns conjunctive polynomial classifiers whereas the one in~\cite{sharma2012interpolants} is limited to conjunctive linear classifier. The more important difference is however our classification algorithm is coupled with a selective sampling procedure.
At line 10, we minimize $C$ by identifying and removing any $f$ in $C$ such that the conjunction of the remaining predicates in $C$ perfectly classifies $P$ and $N$ still. We remark that Algorithm~\ref{alg:conjunctiveSVM}, like the one in~\cite{sharma2012interpolants}, may learn a classifier with many clauses, as one clause is introduced each time line 5 is executed. In the worse case, if each polynomial classifier found at line 4 only classifies the one sample $s$, the returned $C$ would conjunct as many clauses as the number of samples in $N$. Selective sampling, as we discuss below, helps to tame this problem, if there exists a conjunctive polynomial classifier with fewer clauses.



\subsection{Disjunctive Polynomial Classifier}
In the last section, we learn branch invariant candidates according to new divisions on our dataset and
disjunct the branch invariant candidates together to get the loop invariant candidate.
In this section, we show you why we can learn branch invariant candidate and synthesize disjunctive invariant in that way. 

In the Chapter~\ref{chap:introduction}, we assume that we are given a loop program as the Hoare triple on the left side in the following form,
$\emph{Inv}$ representing the invariant for the given loop program.

\begin{align}
&\{\mathit{Pre}\} && \emph{Pre} \Rightarrow \emph{Inv} \label{sl1:org:inv:pre}\\
&\mathit{while} (\mathit{Cond}) \{ \mathit{Body} \} && \{\emph{Inv} \wedge \emph{Cond}\} \emph{Body} \{\emph{Inv}\} \label{sl1:org:inv:loop}\\\
&\{\mathit{Post}\} && \emph{Inv} \wedge \neg \emph{Cond} \Rightarrow \emph{Post} \label{sl1:org:inv:post}
\end{align}

In general, loop invariants are defined as the above form. 
An equivalent definition of loop invariants will be given in this chapter.
Based on the new form, we can extend our framework to support disjunctive loop invariants.

In fact, there might be complex control structure in the loop body, involving multiple branches and even inner loops. 
For simplicity, we start with a simple loop body in which there are just two branches and no inner loops.
Then we extend our approach to loops with more complicated inner structure.

\subsubsection{Disjunctive Invariants for Loop Body with 2 Branches}
Formally, the Hoare triple for any loop body with two branches can be expressed in the following form on the left side, 
where $\mathit{Pre}$ is named the precondition while $\mathit{Post}$ is named the postcondition, and $\mathit{Cond}$ is named the loop condition.
In the loop body, $\mathit{C_1}$ guards the first branch $\mathit{Body_1}$, while $\mathit{\neg C_1}$ guards the other branch $\mathit{Body_2}$.
\begin{align}
&\{\mathit{Pre}\} && \emph{Pre} \Rightarrow \emph{$Inv_1$} \vee \emph{$Inv_2$} \label{sl1:ext:inv:pre}\\
&\mathit{while} (\mathit{Cond}) \{ && \\
&~~~~~~~~\mathit{if} (\mathit{C_1}) ~~\{ \mathit{Body_1} \} && \{(\emph{$Inv_1$} \vee \emph{$Inv_2$}) \wedge Cond \wedge C_1\} Body_1 \{\emph{$Inv_1$}\} \label{sl1:ext:inv:loop:b1}\\
&~~~~~~~~\mathit{else} ~~\{ \mathit{Body_2} \} && \{(\emph{$Inv_1$} \vee \emph{$Inv_2$}) \wedge Cond \wedge \neg C_1\} Body_2 \{\emph{$Inv_2$}\} \label{sl1:ext:inv:loop:b2}\\
&\} && \\
&\{\mathit{Post}\} && (\emph{$Inv_1$} \vee \emph{$Inv_2$}) \wedge \neg Cond \Rightarrow \emph{Post} \label{sl1:ext:inv:post}
\end{align}
% \begin{align*}
% &\mathit{if} (\mathit{$C_1$}) \{ \mathit{$Body_1; continue;$} \} \\
% &\mathit{if} (\mathit{$C_2$}) \{ \mathit{$Body_2; continue;$} \} 
% \end{align*}
If the Hoare triple is valid, $Inv_1 \vee Inv_2$ that satisfies the conditions on the right side is defined as the loop invariant for the program, 
in which $Inv_1$ and $Inv_2$ are branch invariants for corresponding branches.
% For example, $Inv_1$ is the branch invariant for the first branch as it is evaluated true after execution of $Body_1$ as shown in~\ref{sl1:ext:inv:loop:b1}.
In our context, branch invariant is a property that always holds for the given branches.

% in which $Inv_1$ and $Inv_2$ are named as branch invariants for the two branches.
%Assume the invariant for the loop is in the form of $Inv_1 \vee Inv_2$.
%Note that, any invariant $i$ can be converted to this form by linking itself with disjunction operator, such as $i \vee i$.
%If there are $Inv_1$ and $Inv_2$ satisfy the following conditions,
%then $Inv_1 \vee Inv_2$ is a loop invariant for the original program.

In the following, we prove that for loop programs with 2 branches, 
the above definition of loop invariant is equivalent with the previous invariant definition.
Therefore, we need to prove such $Inv_1 \vee Inv_2$ is a valid loop invariant for the loop program,
and any loop invariant for the program can be written as $Inv_1 \vee Inv_2$, where $Inv_1$ and $Inv_2$ are branch invariants.

% \begin{theorem}
% Algorithm~\ref{ta_feasiblefuncwithsim} is sound and complete.
% \vspace{-1mm}
% \end{theorem}
% \noindent \textbf{Proof:} As we discussed the difference between Algorithm~\ref{ta_feasiblefuncwithsim} and Algorithm~\ref{ta_feasiblefunc}, given a transition system $\mathcal{L}$ with a set of initial states $Init$, the transition relation $Tr$ and a set of \buchi conditions $J$, while $IsEmpty(Init, Tr, J)$ is checking the emptiness of $\mathcal{L}$, $IsEmpty_{sim}(Init, Tr, J)$ is actually checking the emptiness of the transition system $\mathcal{L'}$. Thus, the correctness of Algorithm~\ref{ta_feasiblefuncwithsim} is obtained based on Theorem~\ref{theoremofabstractedsystem}.\hfill \qed \\

\begin{theorem}
\label{thm:disjunctive:is:invariant}
	$Inv_1 \vee Inv_2$ is a loop invariant for the given loop program.
\end{theorem}

\noindent \textbf{Proof:} In order to prove $Inv_1 \vee Inv_2$ is a loop invariant for the program,
we need to show it satisfies all the three conditions~\ref{sl1:org:inv:pre}, ~\ref{sl1:org:inv:loop} and ~\ref{sl1:org:inv:post}.

By simply substituting $Inv$ with  $Inv_1 \vee Inv_2$,
we can see $Inv_1 \vee Inv_2$ satisfies condition~\ref{sl1:org:inv:pre} and ~\ref{sl1:org:inv:post}.

For condition~\ref{sl1:org:inv:loop},
as the valuations obtained through the two branches satisfy $Inv_1$ and $Inv_2$ respectively, 
the valuations for the loop body must satisfy $Inv_1 \vee Inv_2$ naturally.
Thus, by combining the condition~\ref{sl1:ext:inv:loop:b1} and~\ref{sl1:ext:inv:loop:b2},
\begin{align*}
&\{(Inv_1 \vee Inv_2) \wedge Cond \wedge C_1\} Body_1 \{Inv_1\} \\
&\{(Inv_1 \vee Inv_2) \wedge Cond \wedge \neg C_1\} Body_2 \{Inv_2\}
\end{align*}
we can get $\{(Inv_1 \vee Inv_2) \wedge Cond\}~if (C1)~{Body_1}~else~{Body_2}~\{Inv_1 \vee Inv_2\}$,
which satisfies the second condition in loop invariant definition.

Therefore, $Inv_1 \vee Inv_2$ is a loop invariant for the given loop program. %\hfill \qed \\

\begin{theorem}
\label{thm:invariant:is:disjunctive}
	Any invariant $Inv$ for the given loop program with branches can be expressed in the form of $Inv_1 \vee Inv_2$.
\end{theorem}

\noindent \textbf{Proof:} If $Inv$ is a loop invariant for the given loop program,
then $Inv$ satisfies the three conditions ~\ref{sl1:org:inv:pre}, ~\ref{sl1:org:inv:loop} and ~\ref{sl1:org:inv:post}.
As $Inv = Inv \vee Inv$ always holds, we assign $Inv_1 = Inv$ and $Inv_2 = Inv$.
Then the three conditions can be easily verified. %\hfill \qed \\

%in the loop body are exclusively executed,
\subsection{Disjunctive Invariants for Loop Body with $\emph{n}$ Branches}
For the loop body with $n$ branches and the valuations after each branch satisfy $Inv_1$, $Inv_2$, $\cdots$, $Inv_n$ respectively,
it is similar to prove the loop invariant defined as $Inv_1 \vee Inv_2 \vee \cdots Inv_n$ is equivalent to the invariant defined before,
which means we can apply same technique to combine all the branch invariants together as the whole loop invariant.

% \subsection{Disjunctive Invariants for Loop Body with Inner Loops}
% For the loop body with inner loops, 
% if the iterations in the inner loops are fixed,
% we can unfold the inner loops into branches as presented before.
% For the inner loop that has uncertain number of iterations, 
% it is not practical to unfold the inner loop into branches according to the iteration number it runs,
% especially the number of the iteration for inner loop varies in a large range.
% For this condition, we treat the inner loop as a basic code block which can be expressed in one node in the control flow graph.
% In other words, the structure of the inner loops are omitted in this case.

% As there is no restriction on the structure of $\mathit{Body_1}$ and $\mathit{Body_2}$ in Theorem~\ref{thm:disjunctive:is:invariant} and ~\ref{thm:invariant:is:disjunctive},
% the disjunctive invariant for the loop body with inner loops can also be defined as before. 
% In this view, we can convert control flow graph with loops into a tree-like control flow graph as well.
% That's all basis needed for our branch-based disjunctive invariant learning approach to work.

% We remark that there can be branches in $\mathit{Body_1}$ or $\mathit{Body_2}$ for the same reason. 
% For example, we can learn invariant for the loop body contains three branches $B_1, B_2, B_3$ in at least two ways.
% In one way, we can find branch invariants $\mathit{Inv_1}, \mathit{Inv_2}, \mathit{Inv_3}$ for each of them and disjunct them together as the loop invariant.
% In the other way, we can treat $B_1, B_3$ (or any other two branches) as one branch $B_{1, 3}$ and learn invariant $\mathit{Inv_{1, 3}}$ for this new big branch 
% and then disjunct it with the branch invariant $\mathit{Inv_2}$ together.
% That's the foundation of our graph-based disjunctive invariant learning approach.
% For the loop body with $n$ branches and the valuations after each branch satisfy $Inv_1$, $Inv_2$, $\cdots$, $Inv_n$ respectively,
% it is similar to prove the loop invariant defined as $Inv_1 \vee Inv_2 \vee \cdots Inv_n$ is equivalent to the invariant defined before,
% which means we can apply same technique to combine all the branch invariant together as the whole loop invariant.
