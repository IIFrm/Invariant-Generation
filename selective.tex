%!TEX root = paper.tex

\section{Selective Sampling}
Due to the limited set of samples we have (which is often referred to as labeled samples in the machine learning community), the classifier obtained above might be far from being correct. In fact, without labeled samples which are right on the boundary of the `actual' classifier, it is very unlikely that we would find it. Intuitively, in order to get the `actual' classifier, we would require samples which would distinguish the actual one from any nearby one. This problem has been discussed and addressed in the machine learning community using active learning and selective sampling~\cite{DBLP:conf/icml/SchohnC00}.

\begin{algorithm}[t]
\SetAlgoVlined
\Indm
\KwIn{$F^+$ and $F^-$}
\KwOut{a classifier for $F^+$ and $F^-$}
\Indp
let $old$ be $null$\;
\While{true} {
    let $f = classify(F^+, F^-)$\;
    \If {$f$ is identical to $old$} {
        \Return $f$;
    }
    let $old = f$\;
    let $sam$ be a set of samples computed by selective sampling\;
    test the program and update $F^+$ and $F^-$ accordingly\;
}
\caption{Algorithm $activeLearning$}
\label{alg:active}
\end{algorithm}

The concept of active learning or selective sampling refers to the approaches that aim at reducing the labeling effort by selecting only the most informative samples to be labeled. SVM selective sampling techniques have been proven effective in achieving a high accuracy with fewer examples in many applications~\cite{DBLP:conf/mm/TongC01,DBLP:journals/jmlr/TongK01}. The basic idea of  selective sampling is that at each round, we select the samples that are the closest to the classification boundary so that they are the most difficult to classify and the most informative to label. Since an SVM classification function is represented by support vectors which are the samples closest to the boundary, this selective sampling effectively learns an accurate function with fewer labeled data~\cite{DBLP:conf/icml/SchohnC00}. In our setting, this means that we should sample a program state right by the classifier and test the program with that state to label that feature vector so that the classifier would be improved.

Algorithm~\ref{alg:active} presents details on how active learning is implemented in \textsc{Zilu}. At line 2, we obtain a classifier based on Algorithm~\ref{classify}. We compare the newly obtained classifier with the previous one at line 4, if they are identical, we return the classifier; otherwise we apply selective sampling so that we can generate additional labeled samples for improving the classifier. In particular, at line 5, we apply standard techniques~\cite{DBLP:conf/icml/SchohnC00} to select the most informative sample. Notice that in our setting, the most informative samples are those which are exactly on the lines and therefore can be obtained by solving an equation system. At line 8, we test the program with the newly generated samples so as to label them accordingly.

\begin{algorithm}[t]
\SetAlgoVlined
\Indm
\KwIn{$Pre$, $Cond$, $Body$, $Post$}
\KwOut{an invariant which completes the proof or a counterexample}
\Indp
let $T$ be a set of random samples\;
\While{true} {
    test the program for each sample in $T$\;
    \If {a state $s$ in $CT$ is identified} {
        \Return $s$ as a counterexample;
    }
    let $P$, $N$ and $NP$ be the respective sets accordingly\;
    let $Inv_u = activeLearning(P, N \cup NP)$\;
    let $Inv_o = activeLearning(P \cup NP, N)$\;
    let $Inv_s = activeLearning(P, N)$\;
    \For {each $Inv$ in $\{Inv_u, Inv_o, Inv_s\}$} {
        \If {(1) or (2) or (3) is not satisfied} {
            add the counterexample into $T$\;
        }
        \Else {
            \Return $Inv$ as the proof;
        }
    }
}
\caption{Algorithm $overall$}
\label{alg:overall}
\end{algorithm}

\begin{example}
\end{example}

\begin{proposition}
Algorithm $activeLearning$ always eventually terminates. \hfill \qed
\end{proposition}

\section{Verification}
Given a learned predicate $Inv$, we verify whether constraint (1), (2) and (3) are satisfied using symbolic execution. If all of them are satisfied, we successfully verify the program. Otherwise, if any of them is violated, the counterexample obtained is added to the set of sample $X$, which is then tested, categorized, used for active learning accordingly. The overall algorithm is presented in Figure~\ref{alg:overall}.

We remark that we learn three classifiers as candidates for the loop invariant: $U$, $OU$, $O$ such that
\begin{itemize}
\item $U$ classifies states in $P$ and those in $N \cup NP$.
\item $O$ classifies states in $N$ and those in $P \cup NP$.
\item $OU$ classifies states in $P$ and $N$;
\end{itemize}
Intuitively, $U$ would be an under-approximation of $Inv$ (by assuming states in $NP$ does not satisfy $Inv$); $O$ would be an over-approximation of $Inv$ (by assuming states in $NP$ does satisfy $Inv$); and $OU$ would be an safe-approximation of $Inv$ (by using states which we are certain whether they are in $Inv$ or not).
\begin{example}
\end{example}


\begin{theorem}
Algorithm $overall$ always eventually terminates and it is correct. \hfill \qed
\end{theorem}
