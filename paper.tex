\documentclass{llncs}
\usepackage{times}
\usepackage{csp}
\usepackage[boxruled,linesnumbered]{algorithm2e}
\usepackage{algpseudocode}
\usepackage{float, graphicx, epstopdf, lipsum, fancyhdr}
\usepackage{wrapfig}
\usepackage{subfigure}
\begin{document}

\title{Loop Invariant Generation through Active Learning}
\author{
Jiaying Li, Li Li, Le Guang Loc, Jun Sun\\
\institute{
Singapore University of Technology and Design \\
\email{jiaying\_li@mymail.sutd.edu.sg\\ \{li\_li,guangloc\_le,sunjun\}@sutd.edu.sg}}
}

\maketitle

\begin{abstract}
	Loop invariant generation is one of the fundamental problems in program analysis and verification. 
    In this work, we propose an automatic generation method for inductive loop invariants 
    through iterations of runtime sampling, machine learning and constraint solving. 
    In each iteration, our method first collects real data at runtime based on selective sampling. 
    Then, based on their satisfaction of the assumptions and assertions in the program, 
    our method uses support vector machine to learn a loop invariant candidate, 
    i.e., a conjunction of linear and polynomial constraints. 
    Finally, if the candidate can be verified as the inductive loop invariant of the program, 
    our method returns it directly and ends the generation process. 
    Otherwise, the counter-example is used to refine the data sampling in the next iteration. 
    The experiment evaluation shows that our method can be used to learn loop invariants 
    effectively and automatically that cannot be learned by other loop invariant generation tools. 
\end{abstract}

%!TEX root = paper.tex

\section{Introduction} % (fold)
\label{sec:introduction}

Automatic loop invariant generation is fundamental for program analysis. A loop invariant can be useful for software verification, compiler optimization, program understanding, etc. In the following, we first define the loop invariant generation problem and then briefly describe existing approaches and then our proposal. For simplicity, we assume that we are given a Hoare triple in the following form.
%\[
%    P = \{ \mathit{Pre} \} \mathit{while}(\mathit{Cond}) \{ \mathit{Body} \} \{ \mathit{Post} \}
%\]
\begin{align*}
&\{Pre\} & & /\star\text{\emph{Assumption}}\star/ \\
& while (Cond) \{ Body \} && /\star\text{\emph{Loop Body}}\star/\\
&\{ Post \} & & /\star\text{\emph{Assertion}}\star/
\end{align*}
Assume that $V = \{x_1{,} x_2{,} \cdots{,} x_n\}$ is a finite set of program variables which are relevant to the loop body. $Pre$, $Cond$ and $Post$ are predicates constituted by variables in $V$.

%\begin{align}
%&\{\mathit{Pre}\} & & \emph{Pre} \Rightarrow \emph{Inv} \label{org:inv:pre}\\
%&\mathit{while} (\mathit{Cond}) \{ \mathit{Body} \} && \{\emph{Inv} \wedge \emph{Cond}\} \emph{Body} \{\emph{Inv}\} \label{org:inv:loop}\\\
%&\{\mathit{Post}\} & & \emph{Inv} \wedge \neg \emph{Cond} \Rightarrow \emph{Post} \label{org:inv:post}
%\end{align}
% \begin{align}
% &\{\mathit{Pre}\} && \emph{Pre} \Rightarrow \emph{Inv} \label{sl1:org:inv:pre}\\
% &\mathit{while} (\mathit{Cond}) \{ \mathit{Body} \} && \{\emph{Inv} \wedge \emph{Cond}\} \emph{Body} \{\emph{Inv}\} \label{sl1:org:inv:loop}\\\
% &\{\mathit{Post}\} && \emph{Inv} \wedge \neg \emph{Cond} \Rightarrow \emph{Post} \label{sl1:org:inv:post}
% \end{align}
%In practice, the pre-condition $\mathit{Pre}$ is often described by
%the specification documents and checking conditions of the program inputs,
%and the post-condition $\mathit{Post}$ is usually specified
%by assertions and exceptions leading to an error state in the program.
Let $s = \{ x_1 \mapsto v_1, \cdots, x_n \mapsto v_n \}$ be a valuation of $V$. Let $\phi$ be a predicate constituted by variables in $V$. $\phi$ is viewed as the set of valuations of $V$ such that $\phi$ evaluates to true given the valuation. We thus write $s \in \phi$ to denote that $\phi$ is evaluated to $true$ given $s$. Otherwise, we write $s \not \in \phi$.
$Body$ is an imperative program which updates the valuation of $V$. For simplicity, we assume that it is a deterministic function\footnote{Our approach works as long as the non-determinism in $Body$ or $Cond$ is irrelevant to whether the postcondition is satisfied or not.} on valuations of variables $V$, and write $Body(s)$ to denote the valuation of $V$ after executing $Body$ given the initial variable valuation $s$. For convenience, $Body^i(s)$ where $i \geq 0$ is defined as follows: $Body^0(s) = s$ and $Body^{i+1}(s) = Body(Body^i(s))$.
%the evaluation function of the program variables $x_1, \ldots, x_n$
%and $\mathit{Body}(s)$ stand for their new evaluation after the execution of $\mathit{Body}$,
%the above program means that (1) $\mathit{Pre}$ is the assumption to the initial value of $s$;
%(2) if the $\mathit{Cond}$ is satisfied by $s$ at an iteration,
%$\mathit{Body}$ will be executed and $s$ will be updated to $\mathit{body}(s)$;
%(3) if the $\mathit{Cond}$ is unsatisfied by $s$ at an iteration,
%the while-loop ends and $s$ should satisfy $\mathit{Post}$.

The problem is thus either to prove the Hoare triple or to disprove it. In order to prove the Hoare triple, we would like to find a loop invariant $\mathit{Inv}$ which satisfies the following three conditions.
\begin{align}
    & Pre \subseteq Inv && \label{inv:pre} \\
    &\forall s.~s \in Inv \land Cond \implies Body(s) \in Inv && \label{inv:loop} \\
    & Inv \land \neg Cond \subseteq Post && \label{inv:post}
\end{align}
In order to disprove it, we would like to find a valuation $s$ such that $s \models \mathit{Pre}$ and executing the loop until it terminates results in a valuation $s'$ such that $s' \not \models \mathit{Post}$. For simplicity, we further assume that the loop body always terminates and refer the readers to~\cite{Domagoj:FAC:2013,LeQC:PLDI:15,Hong:ASE:2015} %%\cite{acmcomm}
for extensive research on proving loop termination.

Many approaches have been proposed to solve this problem.
For example, there are proposals based on abstraction interpretation~\cite{cousot1978automatic,mine2006octagon,karr1976affine,vincent2009subpolyhedra},
%% cousot1979systematic,
 counterexample guided abstraction refinement~\cite{henzinger2003software,thomas2001slam,edmund2003counterexample}, interpolation~\cite{kenneth2010lazy,thomas2004abstractions,kenneth2003interpolation,Kenneth2006lazy} and constraint solving and inference~\cite{ashutosh2009invgen,michael2003linear,sumit2009constraint}.
Recently, the authors of~\cite{sharma2012interpolants,sharma2013verification,DBLP:conf/esop/0001GHALN13,sharma2014invariant} proposed to automatically generate loop invariants based on random searching~\cite{sharma2014invariant} as well as machine learning~\cite{sharma2012interpolants}.
Their approaches start with randomly generating valuations of $V$ (a.k.a.~the samples) and categorize them into different groups, e.g., one containing those satisfying the loop invariant $\mathit{Inv}$ (if there is any) and another containing those not. Machine learning techniques are then used to generalize them in a certain form to obtain candidate loop invariants.
%For instance, classification algorithms like Support Vector Machines (SVM) ~\cite{sharma2012interpolants} can be used to generate classifiers as candidate invariants.
The candidates are then checked using program verification techniques (like symbolic execution~\cite{symbolic}) to see whether they satisfy the three conditions. If any of the conditions is violated, we obtain counterexamples in the form of variable valuations.
For instance, given a candidate $\phi$, if condition (1) is violated, a valuation $s \in (Pre \land \neg \phi)$ is generated, which proves that $\phi$ is not an invariant.
With the new sample $s$, we can re-classify the samples to obtain a new candidate invariant. This guess-and-check process is repeated until either the Hoare triple is proved or disproved.

One problem with the guess-and-check approach is that its effectiveness is often limited by the samples which are generated randomly.
In order to learn the right invariant through classification, often a large number of samples are necessary.
Furthermore, often those samples right by the boundary between variable valuations which satisfy the actual invariant and those which do not must be sampled so that classification techniques would identify the right invariant. Obtaining those samples through random sampling is often hard.
As a result, many iterations of guess-and-check are required. Another problem is that the kinds of loop invariants obtained through existing guess-and-check approaches~\cite{sharma2012interpolants,sharma2013verification,DBLP:conf/esop/0001GHALN13,sharma2014invariant} are often limited, e.g., to conjunctive linear inequalities~\cite{sharma2012interpolants} or equalities~\cite{DBLP:conf/esop/0001GHALN13}. Despite the approach presented in~\cite{DBLP:conf/pldi/GulwaniSV08,DBLP:conf/cav/SharmaDDA11}, learning disjunctive loop invariants remains a challenge.

In this work, we propose a technique to improve the existing guess-and-check approaches~\cite{sharma2012interpolants,sharma2013verification,DBLP:conf/esop/0001GHALN13,sharma2014invariant}.
 %We improve existing approaches in two aspects. First, by adopting active learning techniques, we improve the quality of the candidate invariants prior to verifying them, in every iteration of learn-and-check. As a result, we can reduce the number of learn-and-check iterations significantly. Second, by supporting an extensible framework, we can easily integrate different classification techniques (e.g., SVM with kernel methods~\cite{}) as well as the corresponding active learning techniques so that we can learn a large class of invariants. %We have developed a prototype implementation of our method and applied to benchmark programs including those from the software verification competition. The results show that our method often reduces the number of guess-and-check iterations as well as is able to learning more loop invariants than existing approaches.
%In the following, we define our problem and briefly illustrate how our approach works.
Compared to the existing approaches, we make the following contributions.
Firstly, we propose an active learning technique to overcome the limitation of random sampling.
That is, the active learning technique allows us to automatically generate samples which are important in improving the quality of the candidate invariants
so that we can improve the candidates prior to verifying them during every guess-and-check iteration.
As a result, we can reduce the number of guess-and-check iterations significantly, or even completely in many cases.
%    for automatic invariant inference based on machine learning.
%    Since the samples are chosen for clear purpose
%    to refine the invariant candidate in the \emph{data collection} stage,
%    the invariant converges efficiently.
%    Furthermore, because the counter-examples generated in the \emph{invariant verification} stage
%    give very accurate information to amend the invariant candidate,
%    they become a useful supplementary to overcome the weakness of machine learning
%    and fine-tune the invariant candidate.
Secondly, our approach is designed to be extensible so that we can learn different kinds of invariants.
For instance, we show that we can learn candidate invariants in the form of polynomial inequalities or their conjunctions using a different classification algorithm.
Furthermore, we show that by partitioning the samples according to the control locations they visit and classifying each partition separately, we are able to generate disjunctive invariants. Lastly, we implement our framework as a tool called \textsc{Zilu} (available at~\cite{zilu:repo}) and compare it with state-of-the-art tools like Interproc~\cite{jeannet2010interproc} as well as CPAChecker~\cite{DBLP:conf/cav/BeyerK11}.
%%    i.e.,
    %CPAChecker~\cite{beyer2011cpachecker} and
 %%   Interproc~\cite{jeannet2010interproc}.
%    Our experiment results show that
%    we are the only tool that can work with polynomial invariant inference.
%    Notice that the polynomial invariant inference works in our framework
%    naturally with very light additional programming.
    % Based on the design of different approaches,
    % we also claim that our framework have better extensibility comparing with their method.
%\textsc{Zilu} is built upon existing tools (e.g., GNU Scientific Library ($\mathit{GSL}$)~\cite{gough2009gnu} for active learning,   $\mathit{LibSVM}$~\cite{chang2011libsvm} for $\mathit{SVM}$ classification, revised $\mathit{KLEE}$~\cite{cadar2008klee} for symbolic execution~\cite{king1976symbolic,symbolic}, and Z3~\cite{de2008z3} for verification) and can be used as a language/platform independent tool to verify programs.

The remainders of the paper are organized as follows.
Section~\ref{sec:overview} presents an overview of our approach using an illustrative example.
Section~\ref{sec:classifierlearning} shows how candidate loop invariants are generated and refined through active learning.
%Section~\ref{sec:activelearning} then demonstrates the active learning technique which is applied to reduce the number of required samples.
Section~\ref{sec:evaluations} discusses our implementation and evaluates its effectiveness using a set of benchmark programs.
Section~\ref{sec:related} reviews related work and concludes.


%!TEX root = paper.tex

\section{Problem Definition and Solution Overview}
In the following, we assume that a program contains a finite set of integer variable $\{x,y,z,\cdots\}$ 
and thus a program state is a valuation of the variables. 
A predicate on the variables is viewed as the maximum set of program states which satisfies the predicate. 
We use predicates and sets of program states interchangeably. 
Without loss of generality, we assume the input to \textsc{Zilu} is a Hoare triple
\[
\{Pre\} \\
while~(Cond) \{ \\
~~~~~~Body \\
\} \\
\{Post\}
\]
where $Pre$ is the pre-condition, which should be satisfied before entering the loop; 
$Cond$ is the loop guard condition, which is the only way to enter or exit the loop $Body$; 
$Body$ is the loop body, in which we assume there is no \emph{break} or \emph{goto} statement which can jump out of the loop without checking $cond$; 
and $Post$ is the post-condition, which should be satisfied after the loop. 

For simplicity, we assume that $Body$ is a function such that $Body(s) = s'$ means 
that starting at a program state $s$, executing $Body$ would result in a program state $s'$. 
Furthermore, we write $Body(Pr)$ where $Pr$ to denote the set $\{s' | \exists s \in Pr: Body(s) = s'\}$. 
The goal is thus to automatically obtain a loop invariant such that that the following conditions are satisfied.
\[
Pre \implies Inv ~~~~~~~~~~~~~~~~~~~~~~~~~~~~~~~~~~~~~~~~~~(1) \\
Inv \implies Body(Inv \land Cond) ~~~~~~~~~~~~~~~~~~ (2) \\
Inv \land \neg Cond \implies Post ~~~~~~~~~~~~~~~~~~~~~~~~~ (3)
\]
%where $Inv'$ is the predicate obtained by replacing every variable in $Inv$ with its primed version, denoting the set of program states after executing $Body$.

\begin{figure}[t]
\centering
\begin{minipage}{.5\textwidth}
  \centering
{\scriptsize
\begin{verbatim}
    void ex1 (int x) {
        int y = 355;
        if (x > 46) x = 46;
        while (x <= 100) {
            if (x >= 46) {
                y = y+1;
            }
            x = x + 1;
        }
        assert(y==409);
    }
\end{verbatim}}
  \caption{An example adopted from~\cite{DBLP:conf/popl/GulwaniJ07}}
  \label{fig:test1}
\end{minipage}%
\begin{minipage}{.5\textwidth}
  \centering
{\scriptsize\begin{verbatim}
    void ex2 () {
        lock=0;new=old+1;
        while (new!=old) {
            lock=1;old=new;
            if (foo(new)) {
                lock=0;new++;
            }
        }
        if (lock==0)
            error();
    }
\end{verbatim}}
  \caption{An example adopted from~\cite{DBLP:conf/popl/HenzingerJMS02}}
  \label{fig:test2}
\end{minipage}
\end{figure}

\begin{example}
We use the two examples shown in Figure~\ref{fig:test1} and~\ref{fig:test2} to illustrate how our approach works. 
In $ex1$, the precondition of the loop is $y = 355 \land x \leq 46$ and the post-condition is $y=409$. 
In $ex2$, the precondition is that $lock=0 \land new=old+1$ and the post-condition (necessary so that there is no error) is $lock=1$. 
We remark that $foo(new)$ is an external function which \emph{deterministically} returns either true or false, 
i.e., it returns true if $new$ is even; otherwise, it returns false. 
We will discuss in Section how our approach would work if $foo(new)$ is non-deterministic.
\end{example}

\paragraph{Problem Definition} In this work, we assume that given the Hoare triple, 
there is either a counterexample (i.e., a program state $s$ such that $s \in Pre$ and executing the program from $s$ results in failing $post$) 
or there exists an invariant satisfying (1) and (2) and (3). 
Furthermore, the invariant $inv$ is a boolean formula over a linear inequality constraint 
of the form $ax + bx + \cdots \geq d$ where $a,b,d$ are bounded integer constants; 
and $inv$ contains no more than $k$ such statements. 
We remark that such invariant is in general not convex and thus existing approaches on learning convex invariants do not work~\cite{}.

\paragraph{Overview of Our Approach} 
Our approach to solve the problem is illustrated in Figure~\ref{overview}. 
Firstly, we randomly generate a set of program states right before the loop and test the program. 
Based on the testing results, we obtain program states which must or must not satisfy any invariant satisfying (1), (2) and (3). 
Secondly, we develop an algorithm for generating candidate invariants based classification techniques from the machine learning community. 
Thirdly, to overcome the limitation of the sampled program states, 
we adopt active learning techniques, in particular, selective sampling, to refine the candidate invariants. 
Lastly, we rely on constraint solving techniques to check whether the generate invariant satisfies (1) and (2) and (3). 
If it does, we report that our approach is successful; 
otherwise, using the counterexamples generated by the constraint solvers, we repeat from the second step. 
In this following sections, we present details of each step.

%!TEX root = paper.tex

\section{Sampling}
In this step, we sample, either randomly or using tools based on the idea of concolic testing~\cite{}, a set $T$ of program states and test the program starting with each program state $s$ in $T$. We write $Body^*(s)$ to denote the set of program states which could be reached after executing zero or more iterations of the loop starting from $s$. We write $Body^*(T)$ to denote $\{s' | \exists s \in T \cdot s' \in Body^*(s)\}$. Furthermore, we write $s \Rightarrow s'$ to denote that starting with a program state $s$ would result in state $s'$ when the loop terminates. We categorize program states in $Body^*(T)$ into four sets:
\begin{itemize}
    \item Set $CT_T$ is $\{s \in Body^*(T) | s \in Pre \land s \Rightarrow s' \land s' \notin Post\}$;
    \item Set $P_T$ is $\{s \in Body^*(T) | s \in Pre \land s \Rightarrow s' \land s' \in Post\}$;
    \item Set $N_T$ is $\{s \in Body^*(T) | s \notin Pre \land s \Rightarrow s' \land s' \notin Post\}$;
    \item Set $NP_T$ is $\{s \in Body^*(T) | s \notin Pre \land s \Rightarrow s' \land s' \in Post\}$;
\end{itemize}
We remark that anytime a program state in $CT_T$ is identified, a counterexample is found 
and \textsc{Zilu} reports that verification is failed. 
Otherwise, because $Inv$ must satisfy (1),(2) and (3), we know that $P_T \subseteq Inv$ 
and $N_T \land Inv = \emptyset$. 
The program states in $NP_T$ may or not may be in $Inv$. 
If we know that a program state $s \in NP_T$ is in $Inv$, $Body^*(s) \subseteq Inv$.

\begin{example}
\end{example}
%!TEX root = paper.tex

\section{Classification} % (fold)
\label{sec:classification}

After sampling and labeling in the last step, we obtain some program states must satisfy $Inv$ and some must not. 
%If we view these program states as a data set from the perspective of data scientists, 
%the challenge to find an invariant candidate can be viewed as the problem to find a classifier to divide the dataset according to their labels.
In this section, we state our classification algorithm to find a classifier which can perfectly divide these collected data.

%Fortunately, the machine learning community has studied this problem for years, where we can borrow the idea from.
%Specially, senior machine learning experts have developed several efficient supervised classification algorithms to handle this issue.
In fact, the classification problem has been studied by machine learning community for years,
and several efficient approaches, i.e. perceptron, decision tree, \textsc{Svm} etc., have been developed to handle this problem.
%Among plenty of classification approaches, \textsc{Zilu} takes Supported Vector Machines as a primary approach.
%Before demonstrating the specific technique applied in \textsc{Zilu},
%it should be noted that there are at least two main differences between machine learning problems and the problem in our setting:
But there are still differences between the problem in machine learning field and the one in our setting:
\begin{itemize}
\item Most machine learning techniques values more on prediction correctness rather than understanding the underlying problem.
%although there is a research trend in machine learning to understand ongoing during these years.
However, in the software engineering area, program verification cares about formalization and reasoning.
Therefore, a classification technique from machine learning area would be highly suggested as long as it can predict correctly on newly received data.
But in our setting, we need an explicit classifier which is readable by human and can be proceeded by off-the-shelf constraints solvers.

\item Machine learning algorithms consider more of generalization ability than classification correctness on the training data.
Thus they would sacrifice the prediction correctness on the training data for exchanging the higher generalization ability.
But in our context, the classification accuracy on the training data is of vital importance,
and even a slight classification error can not be tolerated.
On this basis, we hope the learned classifier can reflect the underlying logic of the given program.
% classification algorithm can have better generalization ability.
\end{itemize} 
Considering these differences,
in the following text, 
we show our classification approaches based on the classical machine learning algorithms.


\subsection{SVM}
\label{subsec:svm}
%, is one of the most powerful approaches.
%SVM is a supervised learning model with associated learning algorithms that analyze data used for classification. 
%In most setting, given a set of training examples, each marked as belonging to one of two categories, 
%an SVM training algorithm builds a model, which is a representation of the examples as points in space,
%that can assign new examples into one category or the other, 
%making it a non-probabilistic binary linear classifier. 
\textsc{Svm} (Supported Vector Machines) is a supervised machine learning algorithm for classification and regression analysis. 
We use its binary classification functionality in our framework. 
Mathematically speaking, the binary classification functionality of (linear) $\textsc{Svm}$ works as follows. 

Given two sets of feature vectors $S^+$ and $S^-$, it generates, if there is any, 
a linear constraint in the form of $ax + by + \cdots \geq d$ where $x$ and $y$ are feature values and $a, b, d$ are constants, 
such that every state $s \in \mathcal{S}^+$ satisfies the constraint and every state $s' \in \mathcal{S}^-$ fails the constraint. 
In the following, we write $\textsc{Svm}(\mathcal{S}^+, \mathcal{S}^-)$ to denote the function which returns a linear classifier.


Considering the differences between machine learning problem and our setting, 
\textsc{Zilu} tunes the primitive $\textsc{LibSvm}$~\cite{chang2011libsvm} in the following three aspects in order to get a perfect classifier.
\begin{itemize}
\item Convert \textsc{Svm} model to an explicit classifier.
The original \textsc{Svm} technique does not explicitly calculate a hyperplane, but emits some supported vectors and other parameters,
which can be used to do prediction on the given data.
%This might not a big problem if we do not apply $\textsc{Svm}$ with some kernel method (which is used to classify no linear separable data).
However, we need a explicit classifier as the loop invariant candidate which can be understood and proceeded later for verification.
(This is also why we do not apply $\textsc{Svm}$ with kernel methods~\cite{yu2009evolving},
considering converting $\textsc{Svm}$ models with kernel methods, i.e. $\textsc{Rbf}$ kernel, would be a complicated, sometimes even impossible, task.)

\item As \textsc{Zilu} treasures classification accuracy on the training dataset% than anything else,
before applying primitive $\textsc{Svm}$ technique, the parameters 
(mainly $C$ which tells the $\textsc{Svm}$ optimization how much you want to avoid misclassifying each training example)
%different from $\mathcal{C}$ used as loop invariant candidate in our context) 
should be carefully tuned to learn a perfect classifier which perform well on the training points.

\item Validating the learned classifier. 
Checking the classification correctness of the learned classifier 
on $\mathcal{S}^+$ and $\mathcal{S}^-$ is still needed as our setting needs to ensure the learned classifier is a perfect one.
\textsc{Zilu} also takes $\mathcal{S}^\rightarrow$ to validate the learned classifier(as is shown in Section~\ref{sec:sampling}).
\end{itemize} 

The whole classification algorithm is described in Algorithm~\ref{alg:classify}. 
Note that,
$\textsc{Svm}$ in this algorithm can be substitute with other classification techniques for different learning purposes. 
We will take two classification algorithms as examples to demonstrate this promotion in ~\ref{subsec:svm:derivatives}.
\begin{algorithm}[!h]
\SetAlgoVlined
\Indm
\KwIn{$\mathcal{S}^+$, $\mathcal{S}^-$, and $\mathcal{S}^\rightarrow$}
\KwOut{$\textsc{Null}$ or a perfect classifier for $\mathcal{S}^+$ and $\mathcal{S}^-$ without violating $\mathcal{S}^\rightarrow$}
\Indp
    let $f$ = $\textsc{Svm}$($\mathcal{S}^+$, $\mathcal{S}^-)$\;
    \If {$f$ violates any data in $\mathcal{S}^+$ or $\mathcal{S}^-$} {
        \Return $\textsc{Null}$\;
    }
    \If {$f$ violates any inference rules in $\mathcal{S}^\rightarrow$} {
        \Return $\textsc{Null}$\;
    }
    \Return $f$;
\caption{Algorithm $classify$}
\label{alg:classify}
\end{algorithm}


 
%In the following, we present how we obtain a classifier automatically using $\textsc{Svm}$. 
%In this work, we always choose the \textit{optimal margin classifier} (see the definition in~\cite{Sharma2012}) if possible. 
%This half space could be seen as the strongest witness why the two data states are different. 
%In the following, we write $svm(S^+, S^-)$ to denote the function which returns a linear classifier

%\subsection{Checking}
%With the learned $\textsc{Svm}$ model,we check whether it can perfectly classify these states first.
%If yes, then we can automatically turn it back to a hyperplane form, which is regarded as our invariant candidate.
%Otherwise, we may apply other classification techniques for learning, which will be mentioned at ``$\textsc{Svm}$ derivatives'' part in the end of this section.


\subsection{Active Learning} 
\label{subsec:active:learning}
Having a perfect classifier for the current dataset ($\mathcal{S}^+$, $\mathcal{S}^-$, and $\mathcal{S}^\rightarrow$), 
active learning technique keeps refining the classifier until it gets converged.
That means the classifier remains identical even adding more data points into the training set. 
Algorithm~\ref{alg:active} presents details on how active learning is implemented in \textsc{Zilu}. 

\begin{algorithm}[!h]
\SetAlgoVlined
\Indm
\KwIn{$\mathcal{S}^+$, $\mathcal{S}^-$, and $\mathcal{S}^\rightarrow$}
\KwOut{an invariant candidate $\mathcal{C}$}
\Indp
let $old_f$ be $null$\;
\While{true} {
    let $f$ = classify($\mathcal{S}^+$, $\mathcal{S}^-$, $\mathcal{S}^\rightarrow$)\;
    \If {$f$ is not equal to $\textsc{Null}$} {
        \If {$f$ is identical to $old_f$} {
            $\mathcal{C}$ = $f$\;
            \Return $\mathcal{C}$;
        }
        let $old_f = f$\;
    }
  %\textsc{Re-sampling}:\\
    $sam$ = selectiveSampling($old_f$)\;
    test the target program with $sam$\;
    update $\mathcal{S}^+$, $\mathcal{S}^-$ and $\mathcal{S}^\rightarrow$ accordingly\;
}
\caption{Algorithm $activeLearning$}
\label{alg:active}
\end{algorithm}

At line 3, we obtain a classifier based on Algorithm~\ref{alg:classify}. 
We compare the newly obtained classifier with the previous one at line 4, if they are identical, we return the classifier; 
otherwise, we apply selective sampling so that we can generate additional labeled samples for improving the classifier. 
In particular, at line 9, we apply standard techniques~\cite{DBLP:conf/icml/SchohnC00} to select the most informative sample. 
Notice that in our setting, as indicated in Section ~\ref{sec:sampling}, the most informative samples are those which are exactly on the lines 
and therefore can be obtained by solving an equation system using libraries. 
At line 10 and line 11, we test the program with the newly generated samples so as to label them accordingly.

%\begin{example}
%\LL{to be added}
%\end{example}

\begin{proposition}
Algorithm $activeLearning$ always eventually terminates. \hfill \qed
\end{proposition}


\subsection{SVM Derivatives}
\label{subsec:svm:derivatives}
%$\mathcal{S}^+$, $\mathcal{S}^-$, and $\mathcal{S}^\rightarrow$
If $\mathcal{S}^+$, $\mathcal{S}^-$ cannot be perfectly classified by one half-space only, 
a more complicated function $f$ must be adopted. 
For instance, there has been research on invariants in form of conjunctives~\cite{sharma2012interpolants}, 
octagonal~\cite{mine2006octagon}, tree form~\cite{krishna2015learning}\cite{garg2015learning} and so on.

%For instance, if there is a classifier in the form of conjunctive of multiple half spaces, 
%the algorithm presented in~\cite{sharma2012interpolants} can be used to identify such a classifier.

Moreover, as is noted in algorithm~\ref{alg:classify}, customers can replace $\textsc{Svm}$ with other classification approaches for invariant learning,
\textsc{Zilu} has implemented two classification algorithms besides native $\textsc{Svm}$: 
$Polynomial \textsc{Svm}$ for learning invariants in the form of polynomials or any equivalent expressions,
and $Conjunctive \textsc{Svm}$ for learning invariants in the form of conjunctives.

\subsubsection{Polynomial SVM}
%In previous research, several papers ~\cite{**} have studied invariants with conjunctive form or disjunctive form.
%However, there is still no efficient approach to learning these invariants.
%In our research, we found sometimes convert the conjunctive or disjunctives to a polynomial expression might be a nice try to this problem.
%In the real work, there are not only linear invariants but invariants of many other forms, 
%such as, conjunctives~\cite{sharma2012interpolants}, octagonal~\cite{mine2006octagon}, polynomial, and tree form~\cite{krishna2015learning}\cite{garg2015learning}.
In this part, we present our classification approach based on primitive \textsc{Svm} for learning polynomial invariants,
which have not been discussed by previous research.
%In this part, we would like to learn this kind of invariants.
%Apparently methods that use linear template or linear classification algorithm do not work.
%Actually, polynomials are more powerful on this than they look at the first glance, especially univariate polynomials. 
%Some cubic univariate polynomials can represent disjunctive of a conjunctive expression and a linear expression.
%For example, the following two expressions are equivalent:
%$$\big(x \ge x_0 \bigwedge x \le x_1) \bigvee x \ge x_2\big) \ where\ x_0 < x_1 < x_2$$
%$$x^3 + (x_0x_1 + x_0x_2 + x_1x_2)x^2 - (x_0 + x_1 + x_2)x - x_0x_1x_2 >= 0$$ 
%This leads us to develop a classification algorithm for learning polynomial divider.
To develop a polynomial learner, 
we first map all the raw data (program states) from original space in $\mathcal{S}^+$, $\mathcal{S}^-$ and $\mathcal{S}^\rightarrow$ 
to a high dimensional space.
%The primary idea is simple, after mapping raw data (program states) from original space in $\mathcal{S}^+$, $\mathcal{S}^-$ and $\mathcal{S}^\rightarrow$ to a high dimensions, 
The primitive \textsc{Svm} algorithm is then applied on the mapped data to learn a linear classifier on the projected space.% as is shown in~\ref{subsec:svm}.
Mathematically, a linear classifier in the high dimensional space is the same with a polynomial classifier in the original space.
In this way, we get a polynomial classifier as a result.
The whole procedure is shown in algorithm~\ref{alg:polynomialSVM}.
%In practice, \textsc{Zilu} provide polynomials up to degree 4 as we think degree 4 can cover most hackneyed invariants for now.

\begin{algorithm}[!h]
\SetAlgoVlined
\Indm
\KwIn{$\mathcal{S}^+$, $\mathcal{S}^-$}
\KwOut{a perfect polynomial classifier for $\mathcal{S}^+$ and $\mathcal{S}^-$}
\Indp
    $max\_dimension$ is pre-defined\;
    $dimension = 1$\;
    \While {$dimension \le max\_dimention$} {
        $\mathcal{Q}^+$ = mapToDimension($\mathcal{S}^+$, dimension)\;
        $\mathcal{Q}^-$ = mapToDimension($\mathcal{S}^-$, dimension)\;
        let $f$ = \textsc{Svm}($\mathcal{Q}^+$, $\mathcal{Q}^-$)\;
        \If {$f$ does not violate any raw data in $\mathcal{S}^+$ or $\mathcal{S}^-$} {
        	\Return $f$\;
    	}
    	$dimension = dimension + 1$\;
    }
    \Return $\textsc{Null}$;
\caption{Algorithm $polynomial$\textsc{Svm}}
\label{alg:polynomialSVM}
\end{algorithm}

Along with learning polynomial invariants,
$polynomialSVM$ can also used to learn some invariants with conjunctive or disjunctive forms.
Because some invariants of conjunctive or disjunctive form can be expressed in polynomials equivalently,
For instance, if the target invariant is 
$$\big((x \ge x_0) \wedge (x \le x_1)\big) \vee (x \ge x_2) ~~~where\ x_0 < x_1 < x_2$$
which is really a challenge for current verification approaches to get,
we can find an equivalent polynomial expression:
$$x^3 + (x_0x_1 + x_0x_2 + x_1x_2)x^2 - (x_0 + x_1 + x_2)x - x_0x_1x_2 >= 0$$

%\begin{align}
%    x^3 + a\dot x^2 - b\dot x - c &>= &0 \\
%    where a &= & x_0 \dot x_1 + x_0 \dot x_2 + x_1 \dot x_2 \\
%\end{align}
%For instance, if the target invariant is 
%$$(x \ge x_0) \vee (x \le x_1) \ where\ x_0 < x_1$$
%we can find an equivalent polynomial expression:
%$$x^2 - (x_0 + x_1)x + x_0x_1 \ge 0$$ 
So $polynomialSVM$ can be used to learn a polynomial classifier or any complex expression as long as it can be expressed in form of polynomials.


\subsubsection{Conjunctive SVM}
As is stated in the last paragraph, polynomials can be equivalent with complex conjunctive or disjunctive of linear expressions.
%seems powerful in expression abilities, 
but there actually exists simple expressions that can not be expressed in form of polynomials.
For example, $\mathcal{C} = (x \ge 0 \wedge y \ge 0)$,
it can be proved that there is no such a polynomial which can be equivalent with it.

So in this part, we introduce the algorithm to learn conjunctive invariants directly.
%, avoiding the tries to convert them into a form of polynomials.
Our algorithm is derived from `$\textsc{Svm\_i}$' in~\cite{sharma2012interpolants}.
%Rahul Sharma, A. V. Nori et al. have developed a classification algorithm, named as `$\textsc{Svm\_i}$' in ~\cite{sharma2012interpolants},
%based on $\textsc{Svm}$ to learn conjunctive invariants.
In their algorithm, they take $\textsc{Svm}$ as meta linear classification engine which limits each part of conjunctives to be the linear form, 
and they fail to simplify the learned classifiers which can cause large number of conjunctive parts that can not be understood.
So in \textsc{Zilu}, we apply a technique overcoming these defects by applying $polynomal$\text{Svm} as meta classification algorithm 
and using Z3~\cite{de2008z3} to resolve the inference relationship between classifiers.
The technique is shown in algorithm~\ref{alg:conjunctiveSVM}.% based on their $\textsc{Svm\_i}$ but overcome these defects:

\begin{algorithm}[!h]
\SetAlgoVlined
\Indm
\KwIn{$\mathcal{S}^+$, $\mathcal{S}^-$}
\KwOut{a set of perfect classifiers for $\mathcal{S}^+$ and $\mathcal{S}^-$}
\Indp
    let $\mathcal{C}$ = $\textsc{Null}$\;
    let $\textsc{Misclassified}$ = $\mathcal{S}^-$\;
    \While {$\textsc{Misclassified}$ is not empty} {
        Random choose $s$ from $\textsc{Misclassified}$\;
        let $f$ = polynomial\textsc{Svm}($\mathcal{S}^+$, s)\;
        add $f$ to $\mathcal{C}$\;
        \For {$s' \in \textsc{Misclassified}$} {\
            \If {$f(s') \le 0$} {
                remove $s'$ from $\textsc{Misclassified}$\;
            }
        }
    %}
    \For {$c \in \mathcal{C}$} {
        \If {$\mathcal{C}\diagdown c \Rightarrow c$} {
            remove $c$ from $\mathcal{C}$\;
        }
    }
    }
    \Return $\mathcal{C}$;
\caption{Algorithm $conjunctive$\textsc{Svm}}
\label{alg:conjunctiveSVM}
\end{algorithm}



%\section{Active Learning}
%Due to the limited set of samples we have (which is often referred to as labeled samples in the machine learning community), 
%the guessed classifier obtained from the previous iteration might be far from being correct. 
%In fact, without labeled samples which are right on the boundary of the `actual' classifier, 
%it is very unlikely that we would find it. 
%Intuitively, in order to get the `actual' classifier, we would require samples which would distinguish the actual one from any nearby one. 
%This problem has been discussed and addressed in the machine learning community using active learning and selective sampling~\cite{DBLP:conf/icml/SchohnC00}.

%The concept of active learning or selective sampling refers to the approaches 
%that aim at reducing the labeling effort by selecting only the most informative samples to be labeled. 
%SVM selective sampling techniques have been proven effective in achieving a high accuracy 
%with fewer examples in many applications~\cite{DBLP:conf/mm/TongC01,DBLP:journals/jmlr/TongK01}. 
%The basic idea of selective sampling is that at each round, 
%we select the samples that are the closest to the classification boundary so that they are the most difficult to classify and the most informative to be labeled. 
%Since an SVM classification function is represented by support vectors which are the samples closest to the boundary, 
%this selective sampling effectively learns an accurate function with fewer labeled data~\cite{DBLP:conf/icml/SchohnC00}. 
%In our setting, this means that we should sample a program state right by the classifier and test the program 
%with that state to label that feature vector so that the classifier would be improved.


% section classification (end)
%!TEX root = paper.tex

\section{Verification}
Given a learned predicate $Inv$, we verify whether constraint (1), (2) and (3) are satisfied using symbolic execution and constraint solving.
First we separate the given loop program with the leaned predicate into three loop-free programs,
which corresponds to constraints (1), (2) and (3). 
After compiling these programs, we apply KLEE on each of them to get all the path conditions.



In our implementation, we use KLEE\cite{cadar2008klee} as an invariant verifier to help us validate hypothesis invariants from the learner.
Generally speaking, KLEE is a symbolic virtual machine built on top of the LLVM compiler infrastructure
which can enumerate all the possible paths based on target configuration.
After we compile the source file, we can use KLEE to do symbolic execution on the generated object file.
KLEE will enumerate all the possible paths and then pass their path condition to a solver to get concrete values for all the symbolic variables.
Note that, for any program ran by KLEE, if all the possible paths pass the assertion, 
we can ensure the correctness of the program.
In other words, we have proved the correctness of the program.




If all of them are satisfied, we successfully verify the program. 
Otherwise, if any of them is violated, the counterexample obtained is added to the set of sample $X$, (named as counter-example sampling in section 3)
 which is then tested, categorized, used for active learning accordingly. 
 The overall algorithm is presented in Figure~\ref{alg:overall}.

We remark that we learn three classifiers as candidates for the loop invariant: $U$, $OU$, $O$ such that
\begin{itemize}
\item $U$ classifies states in $P$ and those in $N \cup NP$.
\item $O$ classifies states in $N$ and those in $P \cup NP$.
\item $OU$ classifies states in $P$ and $N$;
\end{itemize}
Intuitively, $U$ would be an under-approximation of $Inv$ (by assuming states in $NP$ does not satisfy $Inv$); 
$O$ would be an over-approximation of $Inv$ (by assuming states in $NP$ does satisfy $Inv$); 
and $OU$ would be an safe-approximation of $Inv$ (by using states which we are certain whether they are in $Inv$ or not).
\begin{example}
\end{example}


\begin{theorem}
Algorithm $overall$ always eventually terminates and it is correct. \hfill \qed
\end{theorem}

%!TEX root = paper.tex

\section{Evaluations} % (fold)
\label{sec:evaluations}


\begin{table*}[t]
    \begin{center}
    % \begin{minipage}{\textwidth}
    % \begin{adjustwidth}{-1in}{-1in}
    \begin{center}
    \begin{adjustbox}{max width=1\textwidth}
    \begin{tabular}{l | r | r | r | r | r | r | r | r | r}
        \hline\hline
        Benchmark 
            & $\sharp$Samples & $\sharp$Invariants & $\sharp$Iterations 
            & $\sharp$Traces & $\sharp$Variables
            & Time & Invariant Type 
            & Interproc & CPAChecker 
            \\
        \hline
        Linear 1
            & 196 & 4 & 1 
            & 1 & 1
            & 3.31s & Linear 
            & 0.01s & 3.42s
            \\
        \hline
        Linear 2
            & 3158 & 7 & 1 
            & 1 & 2
            & 9.86s & Linear 
            & 0.01s & 3.29s
            \\
        \hline
        Linear 3
            & 11102 & 6 & 1
            & 1 & 3
            & 40.24s & Linear 
            & 0.01s & 3.50s
            \\
        \hline
        Linear 4
            & 1143 & 10 & 1
            & 9 & 2
            & 12.54s & Linear 
            & 0.01s & 3.76s
            \\
        \hline
        Linear 5
            & 918 & 8 & 2 
            & 3 & 2
            & 14.47s & Linear 
            & Error & 3.66s
            \\
        \hline
        Poly 1
            & 64 & 7 & 2
            & 1 & 1 
            & 10.51s & Polynomial 
            & Unknown & Unknown 
            \\
        \hline
        Poly 2 
            & 32711 & 85 & 7 
            & 2 & 2
            & 23m43.1s & Polynomial
            & Unknown & Unknown 
            \\
        \hline
        Poly 3 
            & 272 & 17 & 4 
            & 3 & 1 
            & 15.82s & Polynomial 
            & 0.01s & 3.31s 
            \\
        \hline
        Poly 4 
            & 2287 & 112 & 9
            & 2 & 2
            & 13m43.7s & Polynomial
            & Unknown & Unknown 
            \\
        \hline
        Conjunction 1
            & 21247 & 81 & 1 
            & 3 & 2 
            & 20m41.35s & Conjunction
            & 0.01s & 3.16s
            \\
        \hline
    \end{tabular}
    \end{adjustbox}
    \end{center}
    % \end{adjustwidth}
    % \end{minipage}
    \end{center}
    \caption{Experiment Results}
    \label{tab:experiments}
\end{table*}

In this work, we implement our invariant inference framework into a tool called \textsc{Zilu}, 
written in C++ and shell code. 
\textsc{Zilu} uses GSL for selective sampling, LibSVM for machine learning, 
KLEE for concolic testing and Z3 for constraint solving. 
In our experimental evaluation, 
we test \textsc{Zilu} with \LL{Number} loop invariant benchmarks 
in the following form, where $\mathit{Body}$ can have nested loops and conditional choices. 
\[
    \{ \mathit{Pre} \} \mathit{while}(\mathit{Cond}) \{ \mathit{Body} \} \{ \mathit{Post} \}
\]
\LL{Introduce the sources of the benchmark.}
All benchmarks are available from~\cite{zilu}. 

The parameters chosen in our experimental evaluation can be elaborated as follows. 
In the \emph{Sampling} stage, 
the values of all of the program input variables in the random sampling 
and the chosen variables in the selective sampling 
follow the universal distribution over the range of $[-200, 200]$. 
In the \emph{Classification} stage, 
the accuracy of SVM linear learning are set to its maximum value 
in order to generate a absolutely correct classifier if it exists. 
In the \emph{Verification} stage, 
Z3 solver uses `integer' as the type of program variables. 

To evaluate the effectiveness of \textsc{Zilu}, 
we compared it with two available state-of-the-art invariant inference tools: 
Interproc~\cite{cite} and CAPChecker~\cite{cite}, 
which correspond to two different invariant generation methods. 
Interproc is based on abstract interpretation. 
In the experiment, Interproc uses its most expressive abstract domain, i.e., 
the reduced product of polyhedra and linear congruences abstraction. 
Similar to \textsc{Zilu}, Interproc explicitly labels the loop invariants in the loop program. 
We thus can manually check their correctness
and compare them with the invariants generated by our tool \textsc{Zilu}. 
On the other hand, CPACheck (formerly named after BLAST~\cite{cite}) is a software verification tool 
based on the framework of configurable program analysis~\cite{cite}. 
In the experiment, we ask it to generate loop invariant for program predicate abstraction. 
Since CPAChecker functions by proving the program correctness, 
we insert an error branch after each loop program and check for the reachability of the error. 

We present the experiment results in Table~\ref{tab:experiments}. 
Since our invariant reference method has random factors introduced by the sampling sources, 
the presented results are those with median running time in $10$ times of invariant reference process. 
All of the experiments are conducted using x86\_64 Ubuntu 14.04 (kernel 3.13.0-85-generic) 
with 2.3 GHz Intel Core i5 and 4G 1333MHz DDR3. 
In Table~\ref{tab:experiments}, the second column to the seventh column correspond to 
invariant generation details from \textsc{Zilu}. 
`$\sharp$Samples' represents the number of samples generated in the experiment, 
including those from random sampling, selective sampling and counter-example sampling. 
\LL{I need more information on the percentage of the samples. }
`$\sharp$Invariants' stands the numbers of invariants generated by SVM, 
and `$\sharp$Iterations' represents the number of invariant candidates. 
Notice that an invariant generated by SVM becomes an invariant candidate 
when it converges to two previously generated invariants from SVM in one iteration process. 
`$\sharp$Traces' represents the number of loop body traces produced by KLEE. 
and `$\sharp$Variables' represents the number of program variables. 
% In general, when numbers of traces and variables increase in a program,
% the invariant inference difficulty increases.



% section evaluations (end)


%!TEX root = paper.tex

\section{Conclusion and Related Work} % (fold)
\label{sec:related}
In this work, we propose a framework for improving loop invariant learning through active learning. We remark that in theory, we could learn arbitrary mathematical classifier using methods like SVM with kernel methods~\cite{svm:kernel}. Nonetheless, due to the limited proving capability of existing program verification techniques, we focus on invariants in the form of polynomial inequalities or conjunctions of polynomial inequalities.
%Furthermore, we assume there is a bound $k$ on the number of clauses in the variant.
%In practice, we would expect (refer to empirical evidence in Section~\ref{sec:evaluations}) often $k$ is of a small value.


% \begin{table}
%     \begin{center}
%     \begin{tabular}{| l | c | c | l |}
%         \hline
%         Name & Inference Strategy & Inference Source \\
%         \hline
%         ABS & Eager & \\
%         \hline
%         CONS & Eager & \\
%         \hline
%         CEGAR & Lazy & Counterexample \\
%         \hline
%         INTER & Lazy & Counterexample \\
%         \hline
%         G\&C & Eager & Empirical \\
%         \hline
%         ABD & Lazy & Semantic \\
%         \hline
%         DAL & Eager & All Above \\
%         \hline
%     \end{tabular}
%     \end{center}
%     \caption{Existing Invariant Inference Approaches}
%     \label{tab:related}
% \end{table}
This work is related to a large body of approaches on loop invariant generation. 
The existing approaches can be mainly categorized as:
Abstract Interpretation~\cite{cousot1978automatic,mine2006octagon,cousot1979systematic,karr1976affine,vincent2009subpolyhedra},
Constraint Synthesis~\cite{ashutosh2009invgen,michael2003linear,sumit2009constraint},
CounterExample Guided Abstraction Refinement (CEGAR)~\cite{henzinger2003software,thomas2001slam,edmund2003counterexample},
Program Interpolation~\cite{kenneth2010lazy,thomas2004abstractions,kenneth2003interpolation,Kenneth2006lazy},
Guess \& Check~\cite{cormac2001houdini,ernst2007daikon},
Abductive Inference~\cite{isil2013inductive}.
In this section, we compare them with our data-driven active learning approach.

On one hand, invariant inference methods based on Abstract Interpretation and Constraint Synthesis
tend to generate all possible invariants~\cite{mine2006octagon,vincent2009subpolyhedra,ashutosh2009invgen} regardless of
whether they are useful to prove the program correctness or not.
Hence, the invariants inferred by them can be too complex to generate
and thus they fail to prove the program correctness.
On the other hand, other methods based on CEGAR, Interpolation and Abduction
only generate those related to the program verification~\cite{isil2013inductive}.
Thus, they may miss some critical and necessary invariants for the program verification.
Our approach combines their strengths:
we treat the program as a black box
and label samples based on the program verification target,
i.e., the pre-conditions and the post-conditions;
we then infer all possible invariants based on the labels
to prove the program correctness.

Additionally, similar to CEGAR, Guess \& Check, Interpolation and Abduction approaches,
our active learning approach adopts an iterative refinement scheme.
After generating the invariant, we check its correctness and refine it
based on various new information (e.g., counter-examples, new samples)
if it cannot prove the program correctness.
Different from most of the existing refinement approaches,
our method is driven by data samples
rather than syntactic~\cite{cormac2001houdini} or semantic~\cite{ashutosh2009invgen,isil2013inductive} clues.
Hence, it is more flexible and extensible to capture new types of invariants,
and it is platform- and language-independent.
Based on the needs in practice, under-approximation or over-approximation
can also be applied to the data samples with ease.

% section related (end)

%!TEX root = paper.tex

\section{Conclusion} \label{conclusion}


\paragraph{Limitation and Potential Remedies}
We remark that in theory, we could learn arbitrary mathematical classifier using methods like SVM with kernel methods\cite{yu2009evolving}. 
Nonetheless, due to the limitation of proving capability and tools with regards to non-linear constraints, we leave those to our future work. 
Furthermore, we assume there is a bound $k$ on the number of clauses in the variant. 
In practice, we would expect (refer to empirical evidence in Section~\ref{sec:evaluations}) often $k$ is of a small value.


\bibliographystyle{abbrv}
\bibliography{learning}

\end{document}
