\documentclass{llncs}
\usepackage{times}
\usepackage{subcaption}
\usepackage{amssymb}
\usepackage{amsmath}
\usepackage{graphicx,wrapfig,lipsum}
%\usepackage{tikz}
\usepackage[ruled,vlined,linesnumbered]{algorithm2e}
\usepackage{adjustbox}
%\usepackage{ulem}
\usepackage[colorlinks,linkcolor=black,anchorcolor=black,
    citecolor=blue,urlcolor=black,bookmarks=true]{hyperref}
\usepackage{pifont}

\newcommand{\cmark}{\ding{51}}%
\newcommand{\xmark}{\ding{55}}%

\newcommand{\code}[1]{{\small {\ensuremath{\tt #1}}}}
\newcommand\LJ[1]{\textcolor{green}{#1}}
\newcommand\LL[1]{\textcolor{red}{#1}}
\newcommand\SJ[1]{\textcolor{blue}{#1}}
\newcommand\loc[1]{\textcolor{blue}{To do: #1}}

%\newtheorem{definition}{Definition}
%\newtheorem{theorem}{Theorem}
%\newtheorem{lemma}{Lemma}
%\newtheorem{example}{Example}
%\newtheorem{proposition}{Proposition}
\setcounter{tocdepth}{3}
\pagestyle{plain}

\begin{document}

%\setcopyright{acmcopyright}

\title{Loop-invariant Generation through Active Learning}
% \author{
% Jiaying Li, Li Li, Le Guang Loc, Jun Sun\\
% \institute{
% Singapore University of Technology and Design \\
% \email{\{jiaying\_li,li\_li,quangloc\_le,sunjun\}@sutd.edu.sg}}
% }

\author{Jiaying Li\inst{1}, Jun Sun\inst{1}, Li Li\inst{1}, Quang Loc Le\inst{1} and Shang-Wei Lin\inst{2}}
\institute{Singapore University of Technology and Design
\and
School of Computer Science and Engineering, Nanyang Technological University
}

%%\numberofauthors{4}
%\author{
%% You can go ahead and credit any number of authors here,
%% e.g. one 'row of three' or two rows (consisting of one row of three
%% and a second row of one, two or three).
%%
%% The command \alignauthor (no curly braces needed) should
%% precede each author name, affiliation/snail-mail address and
%% e-mail address. Additionally, tag each line of
%% affiliation/address with \affaddr, and tag the
%% e-mail address with \email.
%%
%% 1st. author
%\alignauthor
%Jiaying Li\\
%       \affaddr{Singapore University of Technology and Design}\\
%       \email{jiaying\_li@mymail.sutd.edu.sg}
%% 2nd. author
%\alignauthor
%Li Li\\
%       \affaddr{Singapore University of Technology and Design}\\
%       \email{li\_li@sutd.edu.sg}
%% 3rd. author
%\and  % use '\and' if you need 'another row' of author names
%\alignauthor
%Le Guang Loc\\
%       \affaddr{Singapore University of Technology and Design}\\
%       \email{guangloc\_le@sutd.edu.sg}
%% 4th. author
%\alignauthor
%Jun Sun\\
%       \affaddr{Singapore University of Technology and Design}\\
%       \email{sunjun@sutd.edu.sg}
%}

\maketitle

\begin{abstract}
Loop invariant generation is important in program analysis and verification. In this work, we propose a technique for automatic loop-invariant generation through a combination of active learning and verification. Given a Hoare triple of a program containing a loop, we start with randomly testing the program. We collect program states at run-time and categorize them based on whether they satisfy the invariant to be discovered. Next, classification techniques are employed to generate candidate loop invariants. In particular, we refine the candidates through active learning so as to overcome the lack of sampled program states.
Only after the candidate invariant cannot be improved further through active learning, we verify whether a candidate can be used to prove the Hoare triple.
If it cannot, the generated counterexample is used for re-classification and we repeat the above process. Furthermore, we show that by introducing
path-sensitive learning, i.e., partitioning the program states according to program locations they visit and classifying each partition separately, we are able to learn disjunctive loop invariants. We have developed a prototype tool and applied it to verify a set of benchmark programs. The evaluation shows that our approach complements existing approaches.

%-------------------------------------------------------------------------
%  \keywords Loop Invariant $\cdot$ Active Learning $\cdot$ Verification
% $\cdot$ Program Analysis
%-------------------------------------------------------------------------
\end{abstract}

%!TEX root = paper.tex

\section{Introduction} % (fold)
\label{sec:introduction}

Automatic loop invariant generation is fundamental for program analysis. A loop invariant can be useful for software verification, compiler optimization, program understanding, etc. In the following, we first define the loop invariant generation problem and then briefly describe existing approaches and then our proposal. For simplicity, we assume that we are given a Hoare triple in the following form.
%\[
%    P = \{ \mathit{Pre} \} \mathit{while}(\mathit{Cond}) \{ \mathit{Body} \} \{ \mathit{Post} \}
%\]
\begin{align*}
&\{Pre\} & & /\star\text{\emph{Assumption}}\star/ \\
& while (Cond) \{ Body \} && /\star\text{\emph{Loop Body}}\star/\\
&\{ Post \} & & /\star\text{\emph{Assertion}}\star/
\end{align*}
Assume that $V = \{x_1{,} x_2{,} \cdots{,} x_n\}$ is a finite set of program variables which are relevant to the loop body. $Pre$, $Cond$ and $Post$ are predicates constituted by variables in $V$.

%\begin{align}
%&\{\mathit{Pre}\} & & \emph{Pre} \Rightarrow \emph{Inv} \label{org:inv:pre}\\
%&\mathit{while} (\mathit{Cond}) \{ \mathit{Body} \} && \{\emph{Inv} \wedge \emph{Cond}\} \emph{Body} \{\emph{Inv}\} \label{org:inv:loop}\\\
%&\{\mathit{Post}\} & & \emph{Inv} \wedge \neg \emph{Cond} \Rightarrow \emph{Post} \label{org:inv:post}
%\end{align}
% \begin{align}
% &\{\mathit{Pre}\} && \emph{Pre} \Rightarrow \emph{Inv} \label{sl1:org:inv:pre}\\
% &\mathit{while} (\mathit{Cond}) \{ \mathit{Body} \} && \{\emph{Inv} \wedge \emph{Cond}\} \emph{Body} \{\emph{Inv}\} \label{sl1:org:inv:loop}\\\
% &\{\mathit{Post}\} && \emph{Inv} \wedge \neg \emph{Cond} \Rightarrow \emph{Post} \label{sl1:org:inv:post}
% \end{align}
%In practice, the pre-condition $\mathit{Pre}$ is often described by
%the specification documents and checking conditions of the program inputs,
%and the post-condition $\mathit{Post}$ is usually specified
%by assertions and exceptions leading to an error state in the program.
Let $s = \{ x_1 \mapsto v_1, \cdots, x_n \mapsto v_n \}$ be a valuation of $V$. Let $\phi$ be a predicate constituted by variables in $V$. $\phi$ is viewed as the set of valuations of $V$ such that $\phi$ evaluates to true given the valuation. We thus write $s \in \phi$ to denote that $\phi$ is evaluated to $true$ given $s$. Otherwise, we write $s \not \in \phi$.
$Body$ is an imperative program which updates the valuation of $V$. For simplicity, we assume that it is a deterministic function\footnote{Our approach works as long as the non-determinism in $Body$ or $Cond$ is irrelevant to whether the postcondition is satisfied or not.} on valuations of variables $V$, and write $Body(s)$ to denote the valuation of $V$ after executing $Body$ given the initial variable valuation $s$. For convenience, $Body^i(s)$ where $i \geq 0$ is defined as follows: $Body^0(s) = s$ and $Body^{i+1}(s) = Body(Body^i(s))$.
%the evaluation function of the program variables $x_1, \ldots, x_n$
%and $\mathit{Body}(s)$ stand for their new evaluation after the execution of $\mathit{Body}$,
%the above program means that (1) $\mathit{Pre}$ is the assumption to the initial value of $s$;
%(2) if the $\mathit{Cond}$ is satisfied by $s$ at an iteration,
%$\mathit{Body}$ will be executed and $s$ will be updated to $\mathit{body}(s)$;
%(3) if the $\mathit{Cond}$ is unsatisfied by $s$ at an iteration,
%the while-loop ends and $s$ should satisfy $\mathit{Post}$.

The problem is thus either to prove the Hoare triple or to disprove it. In order to prove the Hoare triple, we would like to find a loop invariant $\mathit{Inv}$ which satisfies the following three conditions.
\begin{align}
    & Pre \subseteq Inv && \label{inv:pre} \\
    &\forall s.~s \in Inv \land Cond \implies Body(s) \in Inv && \label{inv:loop} \\
    & Inv \land \neg Cond \subseteq Post && \label{inv:post}
\end{align}
In order to disprove it, we would like to find a valuation $s$ such that $s \models \mathit{Pre}$ and executing the loop until it terminates results in a valuation $s'$ such that $s' \not \models \mathit{Post}$. For simplicity, we further assume that the loop body always terminates and refer the readers to~\cite{Domagoj:FAC:2013,LeQC:PLDI:15,Hong:ASE:2015} %%\cite{acmcomm}
for extensive research on proving loop termination.

Many approaches have been proposed to solve this problem.
For example, there are proposals based on abstraction interpretation~\cite{cousot1978automatic,mine2006octagon,karr1976affine,vincent2009subpolyhedra},
%% cousot1979systematic,
 counterexample guided abstraction refinement~\cite{henzinger2003software,thomas2001slam,edmund2003counterexample}, interpolation~\cite{kenneth2010lazy,thomas2004abstractions,kenneth2003interpolation,Kenneth2006lazy} and constraint solving and inference~\cite{ashutosh2009invgen,michael2003linear,sumit2009constraint}.
Recently, the authors of~\cite{sharma2012interpolants,sharma2013verification,DBLP:conf/esop/0001GHALN13,sharma2014invariant} proposed to automatically generate loop invariants based on random searching~\cite{sharma2014invariant} as well as machine learning~\cite{sharma2012interpolants}.
Their approaches start with randomly generating valuations of $V$ (a.k.a.~the samples) and categorize them into different groups, e.g., one containing those satisfying the loop invariant $\mathit{Inv}$ (if there is any) and another containing those not. Machine learning techniques are then used to generalize them in a certain form to obtain candidate loop invariants.
%For instance, classification algorithms like Support Vector Machines (SVM) ~\cite{sharma2012interpolants} can be used to generate classifiers as candidate invariants.
The candidates are then checked using program verification techniques (like symbolic execution~\cite{symbolic}) to see whether they satisfy the three conditions. If any of the conditions is violated, we obtain counterexamples in the form of variable valuations.
For instance, given a candidate $\phi$, if condition (1) is violated, a valuation $s \in (Pre \land \neg \phi)$ is generated, which proves that $\phi$ is not an invariant.
With the new sample $s$, we can re-classify the samples to obtain a new candidate invariant. This guess-and-check process is repeated until either the Hoare triple is proved or disproved.

One problem with the guess-and-check approach is that its effectiveness is often limited by the samples which are generated randomly.
In order to learn the right invariant through classification, often a large number of samples are necessary.
Furthermore, often those samples right by the boundary between variable valuations which satisfy the actual invariant and those which do not must be sampled so that classification techniques would identify the right invariant. Obtaining those samples through random sampling is often hard.
As a result, many iterations of guess-and-check are required. Another problem is that the kinds of loop invariants obtained through existing guess-and-check approaches~\cite{sharma2012interpolants,sharma2013verification,DBLP:conf/esop/0001GHALN13,sharma2014invariant} are often limited, e.g., to conjunctive linear inequalities~\cite{sharma2012interpolants} or equalities~\cite{DBLP:conf/esop/0001GHALN13}. Despite the approach presented in~\cite{DBLP:conf/pldi/GulwaniSV08,DBLP:conf/cav/SharmaDDA11}, learning disjunctive loop invariants remains a challenge.

In this work, we propose a technique to improve the existing guess-and-check approaches~\cite{sharma2012interpolants,sharma2013verification,DBLP:conf/esop/0001GHALN13,sharma2014invariant}.
 %We improve existing approaches in two aspects. First, by adopting active learning techniques, we improve the quality of the candidate invariants prior to verifying them, in every iteration of learn-and-check. As a result, we can reduce the number of learn-and-check iterations significantly. Second, by supporting an extensible framework, we can easily integrate different classification techniques (e.g., SVM with kernel methods~\cite{}) as well as the corresponding active learning techniques so that we can learn a large class of invariants. %We have developed a prototype implementation of our method and applied to benchmark programs including those from the software verification competition. The results show that our method often reduces the number of guess-and-check iterations as well as is able to learning more loop invariants than existing approaches.
%In the following, we define our problem and briefly illustrate how our approach works.
Compared to the existing approaches, we make the following contributions.
Firstly, we propose an active learning technique to overcome the limitation of random sampling.
That is, the active learning technique allows us to automatically generate samples which are important in improving the quality of the candidate invariants
so that we can improve the candidates prior to verifying them during every guess-and-check iteration.
As a result, we can reduce the number of guess-and-check iterations significantly, or even completely in many cases.
%    for automatic invariant inference based on machine learning.
%    Since the samples are chosen for clear purpose
%    to refine the invariant candidate in the \emph{data collection} stage,
%    the invariant converges efficiently.
%    Furthermore, because the counter-examples generated in the \emph{invariant verification} stage
%    give very accurate information to amend the invariant candidate,
%    they become a useful supplementary to overcome the weakness of machine learning
%    and fine-tune the invariant candidate.
Secondly, our approach is designed to be extensible so that we can learn different kinds of invariants.
For instance, we show that we can learn candidate invariants in the form of polynomial inequalities or their conjunctions using a different classification algorithm.
Furthermore, we show that by partitioning the samples according to the control locations they visit and classifying each partition separately, we are able to generate disjunctive invariants. Lastly, we implement our framework as a tool called \textsc{Zilu} (available at~\cite{zilu:repo}) and compare it with state-of-the-art tools like Interproc~\cite{jeannet2010interproc} as well as CPAChecker~\cite{DBLP:conf/cav/BeyerK11}.
%%    i.e.,
    %CPAChecker~\cite{beyer2011cpachecker} and
 %%   Interproc~\cite{jeannet2010interproc}.
%    Our experiment results show that
%    we are the only tool that can work with polynomial invariant inference.
%    Notice that the polynomial invariant inference works in our framework
%    naturally with very light additional programming.
    % Based on the design of different approaches,
    % we also claim that our framework have better extensibility comparing with their method.
%\textsc{Zilu} is built upon existing tools (e.g., GNU Scientific Library ($\mathit{GSL}$)~\cite{gough2009gnu} for active learning,   $\mathit{LibSVM}$~\cite{chang2011libsvm} for $\mathit{SVM}$ classification, revised $\mathit{KLEE}$~\cite{cadar2008klee} for symbolic execution~\cite{king1976symbolic,symbolic}, and Z3~\cite{de2008z3} for verification) and can be used as a language/platform independent tool to verify programs.

The remainders of the paper are organized as follows.
Section~\ref{sec:overview} presents an overview of our approach using an illustrative example.
Section~\ref{sec:classifierlearning} shows how candidate loop invariants are generated and refined through active learning.
%Section~\ref{sec:activelearning} then demonstrates the active learning technique which is applied to reduce the number of required samples.
Section~\ref{sec:evaluations} discusses our implementation and evaluates its effectiveness using a set of benchmark programs.
Section~\ref{sec:related} reviews related work and concludes.


\section{The Overall Approach}
\label{sec:overview}
Loop-invariant generation using a guess-and-check approach is an iterative process of \emph{data collection}, \emph{guessing} (i.e., classification in this work) and \emph{checking} (i.e., verification of the invariant candidate).
%The overall workflow is shown in the Figure~\ref{fig:overview}.
In the following, we present how our approach works step-by-step and illustrate each step with simple examples.

\begin{example}
A few example Hoare triples are shown in Figure~\ref{fig:running:example}, where an \code{assume} statement captures the precondition and an \code{assert} statement captures the postcondition.
The set of variables $V$ for each program contains two integer-type ones: $x$ and $y$. For simplicity, we write $(a, b)$ where $a$ and $b$ are integer constants to denote the  evaluation $\{x \mapsto a, y \mapsto b\}$. Further, we interpret integers in the programs as mathematical integers (i.e., they do not overflow).
%During each loop iteration, $x$ is increased by $7$ if it is negative (line 4); otherwise, it is increased by $10$ (line 5).
%$y$ is decreased by $10$ if it is negative (line 6);
%otherwise, it is increased by $3$ (line 7).
%The postcondition $Post$ is $y \le x \le y + 16$.
One example invariant which can be used to prove the Hoare triple is shown for each program. For instance, the Hoare triple shown in Figure~\ref{fig:running:example}(a) can be proven using a loop invariant: $x \le y + 16$, whereas conjunctive or disjunctive invariants are necessary to prove the other Hoare triples. In the following, we show how we generate loop invariants for proving these Hoare triples.
\end{example}
\noindent The overall approach is shown as Algorithm~\ref{alg:active}. We start with randomly generating a set of valuations of $V$, denoted as $SP$, at line 1 (a.k.a.~random sampling). Random sampling provides us the initial set of samples to learn the very first candidate for the loop invariant.
In this work, we have two ways to generate random samples. One is that we generate random values for each variable in $V$ based on its domain,
assuming a uniform probabilistic distribution over all values in its domain.
The other is that we apply an SMT solver~\cite{barrett2009satisfiability,de2008z3} to generate valuations that satisfy $Pre$
as well as those that fail $Pre$. These two ways are complementary.
On one hand, without using a solver, we may not be able to generate valuations which satisfy $Pre$ if $Pre$ is very restrictive
(or fail $Pre$ if the negation of $Pre$ is very restrictive). On the other hand, using a solver often generates biased valuations. %We remark that the cost of generating a random sample is often negligible.

\begin{figure}[t]
\begin{subfigure}{0.5\textwidth}
    \raggedright
\[
 \begin{array}{ll}
1 & \code{~~~ assume(x~{<}~y);}  \\
2 & \code{~~~ while(x~{<}~y)\{}  \\
3 & \code{~~~ \quad if~(x~{<}~0)~x~{:=}~x~{+}~7;}  \\
4 & \code{~~~ \quad else~ x~{:=}~x~{+}~10;}\\
5 & \code{~~~ \quad if~(y~{<}~0)~y~{:=}~y~{-}~10;} \\
6 & \code{~~~ \quad else~ y~{:=}~y~{+}~3;}\\
7 & \code{~~~\}} \\
8 & \code{~~~assert(y~{\leq}~x~{\leq}~y~{+}~16);}
\end{array}
\]
    \caption{Invariant: $x \le y + 16$}
\end{subfigure}%
\begin{subfigure}{.5\textwidth}
        \[
      \begin{array}{ll}
      1 & \code{assume(x~{>}~0~\lor~y~{>}~0);}  \\
      2 & \code{while(x~{+}~y~{\le}-2)\{}  \\
      3 & \code{\quad if~(x~{>}~0) \{}  \\
      4 & \code{\quad \quad ~~~x~{:=}~x+1;}  \\
      5 & \code{\quad \}~else ~\{} \\
      6 & \code{\quad \quad ~y {:=} y + 1;}\\
      7 & \code{\quad \}} \\
      8 & \code{\}} \\
      9 & \code{assert(x~{>}~0~ \lor ~y~{>}~0);}\\
      \end{array}
    \]
    \caption{Invariant: $x > 0 \lor y > 0$}
\end{subfigure}
   \begin{subfigure}{0.5\textwidth}
    \raggedright
     \vspace{0.3cm}
\[
 \begin{array}{ll}
1 & \code{~~~ assume(x=1 {\land} y=0);}  \\
2 & \code{~~~ while(*)\{}  \\
3 & \code{~~~ \quad x {:=} x + y;}  \\
4 & \code{~~~ \quad y {:=} y + 1;}\\
5 & \code{~~~\}} \\
6 & \code{~~~assert(x~{\geq}~y);}
\end{array}
\]
    \caption{Invariant: $y \geq 0 \land x \geq y$}
  \end{subfigure}%
   \begin{subfigure}{0.5\textwidth}
     \vspace{0.3cm}
      \[
      \begin{array}{ll}
      1 & \code{assume(x < 0);} \\
      2 & \code{while(x < 0)\{}  \\
      3 & \code{~~~ \quad x = x + y;}  \\
      4 & \code{~~~ \quad y{++};}  \\
      5 & \code{\}} \\
      6 & \code{assert(y > 0);}
      \end{array}
      \]
    \caption{Invariant: $x < 0 \lor y > 0$}
   \end{subfigure}
\caption{Example programs}
\label{fig:running:example}
\end{figure}

Next, for any valuation $s$ in $SP$, we execute the program starting with initial variable valuation $s$ and record the valuation of $V$ after each iteration of the loop. We write $s \Rightarrow s'$ to denote that there exists $i \geq 0$ such that $s' = Body^i(s)$ and $Body^k(s) \in Cond$ for all $k \in [0, i)$. That is, if we start with valuation $s$, we obtain $s'$ after some number of iterations. At line 3 of Algorithm~\ref{alg:active}, we add all such valuations $s'$ into $SP$. Next, we categorize $SP$ into the four disjoint sets: $CE$, $Positive$, $Negative$ and $NP$. Intuitively, $CE$ contains counterexamples which disprove the Hoare triple; $Positive$ contains those valuations of $V$ which we know must satisfy any loop invariant which proves the Hoare triple; $Negative$ contains those valuations of $V$ which we know must not satisfy any loop invariant which proves the Hoare triple; and $NP$ contains the rest. %Formally,
\[
CE(SP) = \{s \in SP |s \in Pre \land~\exists s'.~s \Rightarrow s' \land s' \not \in Cond \land s' \not \in Post\} \]
A valuation in $CE(SP)$ satisfies $Pre$ and becomes a valuation $s'$ which fails $Post$ when the loop terminates. If $CE(SP)$ is non-empty, the Hoare triple is disproved.
%\begin{align*}
% \mathit{Positive}(\mathit{SP}) = & \{s | \exists s_0 \in SP.~\exists s'.~s_0 \Rightarrow s' \Rightarrow s \Rightarrow s_n \land \\
%     & ~~~~~~s' \models \mathit{Pre} \land s_n \not \models Cond \land s_n \models \mathit{Post}\}
%\end{align*}
%\begin{align*}
\begin{align*}
Positive(SP) = & \{s \in SP | \exists s_0,s_1: SP. \\
& ~~~~~~ s_0 \in Pre \land s_0 \Rightarrow s \Rightarrow s_1 \land s_1 \not \in Cond \land s_1 \in Post\}
\end{align*}
$Positive(SP)$ contains a valuation $s$ if there exists a valuation $s_0$ in $SP$ which satisfies $Pre$ and becomes $s$ after zero or more iterations. Furthermore, $s$ subsequently becomes $s'$, which satisfies $Post$ when the loop terminates. Let $Inv$ be any loop invariant that proves the Hoare triple. Because $s_0 \in Pre$, $s_0 \in Inv$ since $Inv$ satisfies condition (1). Since $Inv$ satisfies condition (2) and $Body(s_0) \in Inv$ if $Body(s_0) \in Cond$. By a simple induction, we prove $s \in Inv$.
\begin{align*}
    Negative(SP) = & \{s \in SP | s \not \in Pre \land \exists s'. \\
    & ~~~~~~s \Rightarrow s' \land s' \not \in Cond \land s' \not \in Post\}
\end{align*}
$Negative(SP)$ is the set of valuations which violates $Pre$ and becomes a valuation $s'$ which violates $Post$ when the loop terminates. We show that $s \not \in Inv$ for all $Inv$ satisfying condition (1), (2) and (3). Assume that $s \in Inv$, by condition (2), $s'$ must satisfy $Inv$ through a simple induction. By condition (3), $s'$ must satisfy $Post$, which contradicts the definition of $Negative(SP)$.
\[    NP(SP) = SP - CE(SP) - Positive(SP) - Negative(SP)
\]
$NP(SP)$ contains the rest of the samples. We remark that a valuation $s$ in $NP(SP)$ may or may not satisfy an invariant $Inv$ which satisfies condition (1), (2) and (3).

\begin{algorithm}[t]
\SetAlgoVlined
\Indm
\Indp
let $SP$ be a set of randomly generated valuations of $V$\;
\While{not time out} {
    add all valuations $s'$ such that $s \Rightarrow s'$ for some $s \in SP$ into $SP$\;
    call $actL(SP)$ to generate a candidate invariant\;
    return ``proved'' if the program is verified with $\phi$ otherwise add the counterexample into $SP$\;
}
\caption{Algorithm $zilu()$}
\label{alg:active}
\end{algorithm}

\begin{example} \label{example2}
Take the program shown in Figure~\ref{fig:running:example}(a) as an example. Assume that the following three valuations are randomly generated:
$(1, 2)$, $(10, 1)$ and $(100, 0)$ at line 1. Three sequences of valuations are generated after executing the program with these three valuations: $\langle (1, 2), (11, 5) \rangle$, $\langle (10, 1) \rangle$ and $\langle (100, 0) \rangle$ respectively.
Note that the loop is skipped entirely for the latter two cases. After categorization, set $CE(SP)$ is empty; $Positive(SP)$ is $\{(1, 2),(11, 5)\}$; $Negative(SP)$ is $\{(100, 0)\}$; and $NP(SP)$ is $\{(10, 1)\}$.
\end{example}
After obtaining the samples and labeling them as discussed above, method $actL(SP)$ at line 4 in Algorithm~\ref{alg:active} is invoked to generate a candidate invariant $\phi$. We leave the details on how candidate invariants are generated in Section~\ref{sec:classifierlearning}, which is our main contribution in this work. Once a candidate is identified, we move on to check whether $\phi$ satisfies condition (1), (2) and (3) at line 5. In particular, we check whether any of the following constraints is satisfiable or not using an $SMT$ solver~\cite{barrett2009satisfiability,de2008z3}.
\begin{align}
    & \mathit{Pre} \land \neg \phi \label{check:inv:pre} \\
     & sp(\phi \land Cond, Body) \land \neg \phi \label{check:inv:loop} \\
    & \phi \land \neg Cond \land \neg Post \label{check:inv:post}
\end{align}
where $sp(\phi \land Cond,Body)$ is the strongest postcondition obtained by symbolically executing program $Body$ starting from precondition $\phi \land Cond$~\cite{DBLP:journals/cacm/Dijkstra75}. If all the three constraints are unsatisfiable, we successfully prove the Hoare triple with the loop invariant $\phi$. If any of the constraints is satisfiable, a model in the form of a variable valuation is generated, which is then added to $SP$ as a new sample. Afterwards, we restart from line 2, i.e., we execute the program with the counterexample valuations, collect and add the variable valuations after each iteration of the loop to the four categories accordingly, move on to active learning and so on.
\begin{example}
For the example shown in Figure~\ref{fig:running:example}(a), a candidate invariant which is automatically learned is $x - y \leq 16$. It is easy to check that this candidate satisfies all the three conditions and thus the Hoare triple shown in Figure~\ref{fig:running:example}(a) is proved. For Figure~\ref{fig:running:example}(c), a candidate invariant returned by method $actL(SP)$ is as follows.
\[
490 + 16x - 9y \geq 0 \land  510 + 6x + 29y \geq 0 \land 56 - y \geq 0 \land 166 - 2x + 5y \geq 0
\]
A counterexample $(-28, -11)$ is generated when we check the satisfiability of (45, which is then used to generate a new candidate. After multiple iterations of guess-and-check, the following invariant is generated.
\[
1 + 2y \geq 0 \land 1 + 2x - 2y \geq 0 \land -1 + 2x \geq 0
\]
Though different from the one we expect, this invariant turns out to be one which is strong enough to prove the Hoare triple.
\end{example}



\section{Our Approach: Classification and Active Learning}
\label{sec:classifierlearning}
In this section, we present details on how candidate loop invariants are generated. The work flow is shown in Algorithm~\ref{classification}, which iteratively generates a candidate through classification (at line 3) and improves it through active learning at line 5 until a fixed point is reached. Note that any time a counterexample is identified (at line 2), our approach exits and reports that the Hoare triple is disproved.

\begin{algorithm}[t]
\SetAlgoVlined
\Indm
\Indp
\While{true} {
    if ($CE(SP)$ is not empty) { exit and report ``disproved''; } \\
    let $\phi$ be a set of candidates generated by $classify(SP)$\;
%    let $c$ be $classify(Positive(SP), Negative(SP))$\;
%    //alternatively: let $c$ be $classify(Positive(SP), Negative(SP) \cup NP(SP))$; \\
%    //alternatively: let $c$ be $classify(Positive(SP) \cup NP(SP), Negative(SP))$; \\
    if ($\phi$ is the same as last iteration){ return $\phi$; } \\
    add $selectiveSampling(\phi)$ into $SP$\;
    add all valuations $s'$ such that $s \Rightarrow s'$ for some $s \in SP$ into SP\;
}
\caption{Algorithm $actL(SP)$}
\label{classification}
\end{algorithm}

The method call $classify(SP)$ at line 3 in Algorithm~\ref{classification} generates a candidate invariant based in classification. Intuitively, since we know that valuations in $Positive(SP)$ must satisfy $Inv$ and valuations in $Negative(SP)$ must not satisfy $Inv$, a predicate separating the two sets (a.k.a.~a classifier) may be a candidate invariant. In order to automatically generate classifiers, we apply existing classification techniques to generate classifiers. There are many classification algorithms, e.g., perceptron~\cite{perceptron}, decision tree~\cite{quinlan1986induction}, Support Vector Machine (SVM)~\cite{svm:original} and neutral network~\cite{nn}.
In our approach, the classification algorithms must generate perfect classifiers. Formally, a perfect classifier $\phi$ for two sets of samples $P$ and $N$ is a predicate such that $s \in \phi$ for all $s \in P$ and $s \not \in \phi$ for all $s \in Negative$. Furthermore, the classification algorithms must generate classifiers which are human-interpretable or can be handled by existing program verification techniques.
In the following, we show how to adopt previously proposed algorithms~\cite{sharma2012interpolants} to generate classifiers based on SVM. Next, we extend the algorithms to generate disjunctive invariants. Lastly, we show how to improve all these candidates systematically through active learning.

\subsection{Linear and Polynomial Classifiers}
In~\cite{sharma2012interpolants}, the authors propose to use SVM to generate candidate invariants. SVM is a supervised machine learning algorithm for classification and regression analysis~\cite{svm:original}.
In general, the binary classification functionality of linear SVM works as follows. Given two sets of samples $P$ and $N$, SVM generates a perfect classifier to separate them if there is any.
In the following, we write $svm(P, N)$ to denote the function which returns a perfect linear classifier for $P$ and $N$ if there is any; or returns $\textsc{null}$ otherwise. We refer the readers to~\cite{svm:smo} for details on how the classifier is computed. In this work, we always choose the \textit{optimal margin classifier} if possible. Intuitively, the optimal margin classifier could be seen as the strongest witness why $P$ and $N$ are different.
SVM by default learns classifiers in the form of a linear inequality (a.k.a.~a half space), e.g., in the form of $a x + b y \geq c$ where $x$ and $y$ are variables whereas $a$, $b$, and $c$ are constants.

In practice, linear classifiers may not be sufficient and thus more expressive invariants are necessary. We can extend SVM to learn polynomial classifiers. Given two sets of samples $P$ and $N$ as well as a degree of the polynomial classifier, we can systematically map all the samples in $P$ (similarly $N$) to a set of samples $P'$ (similarly $N'$) in a high dimensional space. For instance, assume that the target degree be 2, the sample valuation $\{ x \mapsto 2, y \mapsto 1\}$ in $P$ is mapped to $\{x \mapsto 2, y \mapsto 1, x^2 \mapsto 4, xy \mapsto 2, y^2 \mapsto 1\}$.
SVM is then applied to learn a perfect linear classifier for $P'$ and $N'$. Mathematically, a linear classifier in the high dimensional space is the same as a polynomial classifier in the original space~\cite{svm:kernel}.
We remark that the size of each sample in $P'$ or $N'$ grows rapidly with the increase of the degree and thus the above method is often limited to polynomial classifiers with relatively low degree.

A polynomial classifier can represent classifiers in the form of disjunctive or conjunctive linear inequalities. For instance, the classifier $(x \ge d_0) \wedge (x \le d_1)\big) \vee (x \ge d_2)$
where $d_0 < d_1 < d_2$ are constants can be represented as follows.
\[
x^3 + (d_0d_1 + d_0d_2 + d_1d_2)x^2 - (d_0 + d_1 + d_2)x - d_0d_1d_2 \geq 0
\]
However, it is not always possible, i.e., some conjunctive or disjunctive linear inequalities cannot be expressed in the form of a polynomial classifier. One such simple example is: $x \ge 0 \land y \ge 0$.

In~\cite{sharma2012interpolants}, an algorithm for learning conjunctive classifiers is proposed. The idea is to pick one sample $s$ from $N$ each time and identify a classifier $\phi_i$ (i.e., a linear or polynomial one) to
separate $P$ and $\{s\}$, remove all samples from $N$ which can be correctly classified by $\phi_i$, and then repeat the process until $N$ becomes empty. The conjunction of all the classifier is then a perfect classifier. We refer the readers to~\cite{sharma2012interpolants} for details of the algorithm. %This approach however may learn a classifier with many clauses. In the worse case, if each classifier $\phi_i$ only classifies the one sample $s$, the returned classifier would conjunct as many clauses as the number of samples in $N$.

\subsection{Disjunctive Classifiers}
It is known~\cite{DBLP:conf/cav/SharmaDDA11,DBLP:conf/pldi/GulwaniSV08} that it is challenging to automatically generate disjunctive invariants, whereas certain loops can only be proved with disjunctive invariants. In the following, we show how to learn disjunctive invariants through classification. Our observation is that a disjunctive invariant is needed often when the loop contains multiple branches. For instance, proving the Hoare triple shown on the left of Figure~\ref{fig:disjunctive:example} requires the loop invariant $x > 0 \lor y > 0$, which is due to the conditional branch at line 3; and the Hoare triple shown on the right of Figure~\ref{fig:disjunctive:example} (adopted from~\cite{DBLP:conf/popl/HenzingerJMS02}) can be proved with the loop invariant: $lock = 0 \lor new = old$. The reason that such a loop invariant is required is due to the conditional branch at line 5. Based on this observation, we apply the following approach to learn conjunctive or disjunctive invariants. 

Assume that the loop body $Body$ contains a finite set of control locations. For instance, the loop in the first program in Figure~\ref{fig:disjunctive:example} has two locations: line 3 and 4. Given a valuation of $V$, say $s$, we write $visit(s)$ to be the set of control location which is visited if we execute the program with initial variable valuation $s$ during the first iteration of the loop. For instance, given the second program in Figure~\ref{fig:disjunctive:example}, $visit(\{lock \mapsto 0, new \mapsto 0, old \mapsto 1\})$ returns $\{\}$. We first partition $SP$ into a set of disjoint partitions such that all valuations in the same partition $SP_i$ visits the same set of control locations. Note that applying the same categorization discussed in Section~\ref{sec:overview}, samples in each partition can be similarly categorized into four sets. Next, we apply the above-mentioned classification algorithms to learn classifiers for each partition. Assume that $\phi_i$ is the classifier obtained for partition $i$ such that a sample in $Postitive(SP_i)$ satisfies $\phi_i$. The disjunction $\bigwedge_i \phi_i$ is then a perfect classifier separating $Positive(SP)$ from the rest.

\begin{example}

\end{example}

\begin{figure}[t]
   \begin{subfigure}{0.5\textwidth}
    \raggedright
    % \vspace{0.5cm}
    \[
      \begin{array}{ll}
      1 & \code{assume(x~{>}~0~ {||} ~y~{>}~0);}  \\
      2 & \code{while(x~{+}~y~{\le}-2)\{}  \\
      3 & \code{\quad if~(x~{>}~0) ~~~x{++};}  \\
      4 & \code{\quad else~ ~~~y{++};}\\
      5 & \code{\}} \\
      6 & \code{assert(x~{>}~0~ {||} ~y~{>}~0);}\\
      \end{array}
    \]
%     \caption{A sample program}
%     \label{fig:sl1:example:program}
   \end{subfigure}%
   \begin{subfigure}{0.5\textwidth}
      \[
      \begin{array}{ll}
      1 & \code{assume(lock=0~and~new=old+1)} \\
      2 & \code{while(new~!=~old)\{}  \\
      3 & \code{~~~ \quad lock=1;~old=new;}  \\
      4 & \code{~~~ \quad if~(*) \{ lock=0;~new++; \}}  \\
%      5 & \code{~~~ \quad ~~~}\\
%      6 & \code{~~~ \quad \}} \\
      5 & \code{\}} \\
      6 & \code{assert(lock~!=~0);}
      \end{array}
    \]
   \end{subfigure}
\caption{Sample programs with disjunctive loop invariants}
\label{fig:disjunctive:example}
\end{figure}

%\subsubsection{Disjunctive Invariants for Loop Body with 2 Branches}
%Formally, the Hoare triple for any loop body with two branches can be expressed in the following form on the left side.
%% where $\mathit{Pre}$ is named the precondition while $\mathit{Post}$ is named the postcondition, and $\mathit{Cond}$ is named the loop condition.
%In the loop body, $\mathit{C_1}$ guards the first branch $\mathit{Body_1}$, while $\mathit{\neg C_1}$ guards the other branch $\mathit{Body_2}$.
%\begin{align}
%&\{\mathit{Pre}\} && \emph{Pre} \Rightarrow \emph{$Inv_1$} \vee \emph{$Inv_2$} \label{ext:inv:pre}\\
%&\mathit{while} (\mathit{Cond}) \{ && \\
%&~~~~~~~~\mathit{if} (\mathit{C_1}) ~~\{ \mathit{Body_1} \} && \{(\emph{$Inv_1$} \vee \emph{$Inv_2$}) \wedge Cond \wedge C_1\} Body_1 \{\emph{$Inv_1$}\} \label{ext:inv:loop:b1}\\
%&~~~~~~~~\mathit{else} ~~\{ \mathit{Body_2} \} && \{(\emph{$Inv_1$} \vee \emph{$Inv_2$}) \wedge Cond \wedge \neg C_1\} Body_2 \{\emph{$Inv_2$}\} \label{ext:inv:loop:b2}\\
%&\} && \\
%&\{\mathit{Post}\} && (\emph{$Inv_1$} \vee \emph{$Inv_2$}) \wedge \neg Cond \Rightarrow \emph{Post} \label{ext:inv:post}
%\end{align}
%If the Hoare triple is valid, $Inv_1 \vee Inv_2$ that satisfies the conditions on the right side is defined as the loop invariant for the program,
%in which $Inv_1$ and $Inv_2$ are invariants for corresponding branches.
%% For example, $Inv_1$ is the branch invariant for the first branch as it is evaluated true after execution of $Body_1$ as shown in~\ref{sl1:ext:inv:loop:b1}.
%In our context, branch invariant is a property that always holds for the given branches.
%
%% in which $Inv_1$ and $Inv_2$ are named as branch invariants for the two branches.
%%Assume the invariant for the loop is in the form of $Inv_1 \vee Inv_2$.
%%Note that, any invariant $i$ can be converted to this form by linking itself with disjunction operator, such as $i \vee i$.
%%If there are $Inv_1$ and $Inv_2$ satisfy the following conditions,
%%then $Inv_1 \vee Inv_2$ is a loop invariant for the original program.
%
%In the following, we prove that for loop programs with 2 branches,
%the above definition of loop invariant is equivalent with the previous invariant definition.
%Therefore, we need to prove such $Inv_1 \vee Inv_2$ is a valid loop invariant for the loop program,
%and any loop invariant for the program can be written as $Inv_1 \vee Inv_2$, where $Inv_1$ and $Inv_2$ are branch invariants.
%
%% \begin{theorem}
%% Algorithm~\ref{ta_feasiblefuncwithsim} is sound and complete.
%% \vspace{-1mm}
%% \end{theorem}
%% \noindent \textbf{Proof:} As we discussed the difference between Algorithm~\ref{ta_feasiblefuncwithsim} and Algorithm~\ref{ta_feasiblefunc}, given a transition system $\mathcal{L}$ with a set of initial states $Init$, the transition relation $Tr$ and a set of \buchi conditions $J$, while $IsEmpty(Init, Tr, J)$ is checking the emptiness of $\mathcal{L}$, $IsEmpty_{sim}(Init, Tr, J)$ is actually checking the emptiness of the transition system $\mathcal{L'}$. Thus, the correctness of Algorithm~\ref{ta_feasiblefuncwithsim} is obtained based on Theorem~\ref{theoremofabstractedsystem}.\hfill \qed \\
%
%\begin{theorem}
%\label{thm:disjunctive:is:invariant}
%	$Inv_1 \vee Inv_2$ is a loop invariant for the given loop program.
%\end{theorem}
%
%\noindent \textbf{Proof:} In order to prove $Inv_1 \vee Inv_2$ is a loop invariant for the program,
%we need to show it satisfies all the three conditions~\ref{org:inv:pre}, ~\ref{org:inv:loop} and ~\ref{org:inv:post}.
%
%By simply substituting $Inv$ with  $Inv_1 \vee Inv_2$,
%we can see $Inv_1 \vee Inv_2$ satisfies condition~\ref{org:inv:pre} and ~\ref{org:inv:post}.
%For condition~\ref{org:inv:loop},
%as the valuations obtained through the two branches satisfy $Inv_1$ and $Inv_2$ respectively,
%the valuations for the loop body must satisfy $Inv_1 \vee Inv_2$ naturally.
%Thus, by combining the condition~\ref{ext:inv:loop:b1} and~\ref{ext:inv:loop:b2},
%\begin{align*}
%&\{(Inv_1 \vee Inv_2) \wedge Cond \wedge C_1\} Body_1 \{Inv_1\} \\
%&\{(Inv_1 \vee Inv_2) \wedge Cond \wedge \neg C_1\} Body_2 \{Inv_2\}
%\end{align*}
%we can get $\{(Inv_1 \vee Inv_2) \wedge Cond\}~if (C1)~{Body_1}~else~{Body_2}~\{Inv_1 \vee Inv_2\}$,
%which satisfies the second condition in loop invariant definition.
%
%Therefore, $Inv_1 \vee Inv_2$ is a loop invariant for the given loop program. %\hfill \qed \\
%
%\begin{theorem}
%\label{thm:invariant:is:disjunctive}
%	Any invariant $Inv$ for the given loop program with branches can be expressed in the form of $Inv_1 \vee Inv_2$.
%\end{theorem}
%
%\noindent \textbf{Proof:} If $Inv$ is a loop invariant for the given loop program,
%then $Inv$ satisfies the three conditions ~\ref{org:inv:pre}, ~\ref{org:inv:loop} and ~\ref{org:inv:post}.
%As $Inv = Inv \vee Inv$ always holds, we assign $Inv_1 = Inv$ and $Inv_2 = Inv$.
%Then the three conditions can be easily verified. %\hfill \qed \\
%
%\subsection{Disjunctive Invariants for Loop Body with $\emph{n}$ Branches}
%For the loop body with $n$ branches and valuations for each branch satisfy $Inv_1$, $Inv_2$, $\cdots$, $Inv_n$ respectively,
%it is similar to prove the loop invariant defined as $Inv_1 \vee Inv_2 \vee \cdots \vee Inv_n$ is equivalent to the invariant defined before,
%which means we can apply same technique to combine all the branch invariants together as the candidate loop invariant.
%
%\subsubsection{Disjunctive Invariant Learning Algorithm}
%Assume a Hoare triple that there are 2 branches in the loop body is given,
%and thus our invariants can be written as $(Inv_1 \vee Inv_2 \vee \cdots \vee Inv_n)$.
%Without loss of generality, we take branch $B_i~(1 \le i \le n)$ , whose branch invariant $Inv_i$ accordingly, as an example.
%We build a new set $\mathit{Positive\_B_i}$ by extracting the valuations in $\mathit{Positive}$ which are obtained after passing branch $B_i$.
%According to definition in~\ref{ext:inv:loop:b1} and \ref{ext:inv:loop:b2},
%valuations in $\mathit{Positive\_B_i}$ must satisfy $Inv_i$.
%For any valuation $s \in \mathit{Negative}$,
%we can infer $s \not \models Inv_i$ since $s \not \models Inv_1 \vee Inv_2 \vee \cdots \vee Inv_n$.
%Therefore, all the valuation in $\mathit{Negative}$ fails any branch invariant $Inv_i $.
%
%As a result, we can apply Algorithm~\ref{alg:polynomialSVM} on set $\mathit{Positive\_B_i}$ against set $\mathit{Negative}$ to learn a classifier as the candidate for $Inv_i$.
%Similar procedures are adapted for other branches and the approach can generate an invariant candidate by combining all the classifiers.

%\paragraph{Example}
%In the following, we use an illustrative example to show how our framework works on disjunctive invariant learning.

%% can apply the primary SVM
%\begin{figure}[t]
%  % \begin{subfigure}{0.5\textwidth}
%    \raggedright
%    % \vspace{0.5cm}
%     \vspace{-0.2cm} \[
%      \begin{array}{ll}
%      1 & \code{void~foo(int ~x{,} ~int~y)\{} \\
%      2 & \code{~~~ assume(x~{>}~0~ {||} ~y~{>}~0);}  \\
%      3 & \code{~~~ while(x~{+}~y~{\le}-2)\{}  \\
%      4 & \code{~~~ \quad if~(x~{>}~0) ~~~x{++};}  \\
%      5 & \code{~~~ \quad else~ ~~~y{++};}\\
%      6 & \code{~~~\}} \\
%      7 & \code{~~~assert(x~{>}~0~ {||} ~y~{>}~0);}\\
%      8 & \}
%      \end{array}
%    \]
%  %   \vspace{-0.2cm}
%  %   \caption{A sample program}
%  %   \label{fig:sl1:example:program}
%  % \end{subfigure}%
%  % \begin{subfigure}{0.5\textwidth}
%  %   \centering
%  %   \includegraphics[scale=0.25]{figures/sl1_cfg.pdf}
%  %   \caption{The Control flow graph of the loop}
%  %   \label{fig:sl1:example:cfg}
%  % \end{subfigure}
%\caption{Disjunctive loop example}
%\label{fig:disjunctive:example}
%\end{figure}
% \vspace{-0.2cm}
%We collect and label the variable valuations at line 4 and line 5.
%After classifier learning phase, a classifier $x>0$ can be learned at line 4 while $y>0$ can be obtained at line 5.
%Thus, we can get a candidate invariant $(x>0) \vee (y>0)$ by combining the classifiers together.
%Apparently, $(x>0) \vee (y>0)$ is the actual invariant for the given program.

\subsection{Making Use of Undetermined Samples}
The other issue is: how do we handle those valuations in $NP$, which may or may not satisfy $Inv$? If we simply ignore them, there may be a gap between $Positive$ and $Negative$ and as a result, the learnt classifier may not converge to the invariant we want, even with the help of active learning.
This is illustrated in Figure~\ref{fig:running:example:sampling}, where the set of valuations in $Positive$ (marked with $+$), $Negative$ (marked with $-$) and $NP$ (marked with $?$) in our running example are plotted in a 2-D plane. Many samples between the line $x=y$ and $x-y=16$ may be contained in $NP$. Without considering the samples in $NP$, classifiers located in the $NP$ region (e.g., $x - y \leq 10$, or $x - y \leq 15$) may be learned to perfectly classify $Positive$ and $Negative$. Identifying more samples in $Positive$ or $Negative$ may not help to improve the classifier either. To overcome the problem, in addition to learn a classifier separating $Positive$ and $Negative$, we learn two additional candidate invariants making use of $NP$:
one separating $Positive$ from $Negative$ and $NP$ (i.e., assuming valuations in $NP$ fail $Inv$);
and the other separating $Negative$ from $Positive$ and $NP$ (i.e., assuming valuations in $NP$ satisfy $Inv$).
We remark that active learning is applied to all three candidates until they converge.
In our example, if we restrict our learning classifier to linear inequalities, the classifier separating $Positive$ from $Negative$ and $NP$ converge to $\textsc{null}$ (no such classifier), whereas the classifier separating $Negative$ from $Positive$ and $NP$ converges to the desired $x - y \leq 16$.

Once we collect the four categories of samples, we generate candidate loop invariants are obtained through classification algorithms developed in the machine learning community.

\subsection{Active Learning}
There are however two issues to be solved.
The first issue is, with the limited samples in $Positive$ and $Negative$,
it is unlikely that we can obtain an ``accurate'' classifier.
For instance, given the above-mentioned set $Positive$ and $Negative$,
a classifier identified using classification techniques like $SVM$ could be: $3x-10y \leq 152$. %$x+y \leq 50$.
Although this classifier perfectly separates the current valuations in $Positive$ from $Negative$,
it is not useful in proving the Hoare triple and is clearly the result of having limited samples.
Researchers in the machine learning community have studied extensively on how to overcome the problem of limited samples and one of the remedies is active learning~\cite{DBLP:series/synthesis/2012Settles}.
Active learning is a semi-supervised machine learning in which a learning algorithm is able to interactively ask for samples which are important in improving a given classifier.
For instance, active learning for $SVM$ works by repeatedly generating samples on (or nearby) the current classification boundary,
categorizing them accordingly and re-applying $SVM$ to generate new classifiers.
This process is repeated until the classifier converges.
%For instance, Figure~\ref{fig:running:example:sampling} shows where the set of valuations
%in $\mathit{Positive}$, $\mathit{Negative}$ and $\mathit{NP}$ locate geographically in a 2-D plane for our running example.
%In particular, valuations in $\mathit{Positive}$ are labeled with $+$; % and color green;
%valuations in $\mathit{Negative}$ are labeled with $-$; % and color red;
%and valuations in $\mathit{NP}$ are labeled with ?. % and color yellow.
%And there are three areas: a pure positive area with color green, a pure negative area with color red, and a mixed area with color yellow.
%The mixed area exists because our labeling method depends much on the observed program valuation sequences.
%
In the above example, given the current classifier $3x-10y \leq 152$, we apply active learning
and generate new valuations $(7, 13)$ and $(14, -11)$ %$\{x \mapsto 44, y \mapsto -2\}$
by solving the equation $3x-10y = 152$. % (and using an existing $x$ values to figure out the corresponding $y$ value and vice versa).
Next, we execute the program with these valuations,
obtain the variable valuation after each iteration, and add them into $\mathit{CE}$, $\mathit{Positive}$, $\mathit{Negative}$ or $\mathit{NP}$ accordingly.
With these new samples, a new improved classifier is then learned.

Active learning aims at generating candidate invariants through iterations of classification and selective sampling. Our overall algorithm for active learning is shown in Algorithm~\ref{alg:active}. The inputs of the algorithm include the set of samples $\mathit{SP}$; a classification algorithm $\mathit{classify}(P,N)$ which takes two sets of samples $P$ and $N$ and generates a classifier separating $P$ and $N$; and an algorithm for selective sampling $\mathit{selectiveSampling}$ which is often coupled with the classification algorithm. During each iteration of the loop from line 2 to 10, we apply $\mathit{classify}$ three times at line 3, 4 and 5 to learn three classifiers. The reason has been discussed in Section~\ref{sec:overview}. At line 6, we check whether any of the classifiers is different from the ones obtained during the last iteration. If none of them is, we return the three classifiers as candidate invariants at line 7, which will be verified afterwards. Otherwise, at line 9, we apply selective sampling to add more samples into $\mathit{SP}$ and then move on to the next iteration. We remark this algorithm is customizable in term of the classification algorithm and the corresponding selective sampling algorithm. In the following, we present two classification algorithms and selective sampling methods as examples.

\paragraph{Selective Sampling} \label{subsec:active:learning}
Selective sampling is helpful in reducing the number of required samples.
Often, different selective sampling methods are adopted according to different classification algorithms.
In the following, we show how selective sampling works for the two above-mentioned classification algorithms. %We refer the readers to~\cite{???} on a survey on how selective sampling works in general.

Assume that we adopt Algorithm~\ref{alg:polynomialSVM} in our framework and learn a polynomial classifier: $\mathit{-4x^2+2y \geq -11}$.
Following the idea in~\cite{DBLP:conf/icml/OrabonaC11}, the following procedure is applied to identify samples right on the classification boundary for improving the classifier.
\begin{enumerate}
\item Choose a variable in the classifier, for example, $x$.
\item Generates random value for other variables. For example, we let $y$ be $12$.
\item Solve the equation $\mathit{-4x^2+2y = -11}$ after substituting variables with their values. If there is no solution, go back to (1) and retry.
In our example, $\mathit{x \approx 2.9580}$.
%\item Add a random variance $\epsilon \in [-1, 1]$ to the value of the picked variable. Here we add $\epsilon = 0.4$ to the value of $x$, and thus the new value of $x$ is $3.3580$.
\item Roundoff the values of all the variables according to their types in the given program. In our example, we obtain the valuation $\mathit{\{x \mapsto 3, y \mapsto 12\}}$.
\end{enumerate}
Alternatively, we can use existing equation system solvers directly to find solutions for equation $\mathit{-4x^2+2y = -11}$.
If Algorithm~\ref{alg:conjunctiveSVM} is adopted to learn conjunctive polynomial classifiers, we apply the above procedure to each and every polynomial clause in the classifier to obtain new samples.
While it is easy to see that the classifiers learnt in Algorithm~\ref{alg:active} may improve through the iterations (since more samples are available),
it is hard to predict how fast it converges.
We evaluate the effectiveness of these classification algorithms as well as selective sampling methods empirically in the next section.

We remark that with the help of active learning, we can often reduce the number of learn-and-check iterations. %, and also the corresponding learning time.
For our running example, with active learning, one iteration of learn-and-check is sufficient to prove the Hoare triple.
Without active learning, multiple iterations are often required as shown in Section~\ref{sec:evaluations}.

%\input{activelearning.tex}
\vspace{-0.3cm}
%!TEX root = paper.tex

\section{Evaluations} % (fold)
\label{sec:evaluations}


\begin{table*}[t]
    \begin{center}
    % \begin{minipage}{\textwidth}
    % \begin{adjustwidth}{-1in}{-1in}
    \begin{center}
    \begin{adjustbox}{max width=1\textwidth}
    \begin{tabular}{l | r | r | r | r | r | r | r | r | r}
        \hline\hline
        Benchmark 
            & $\sharp$Samples & $\sharp$Invariants & $\sharp$Iterations 
            & $\sharp$Traces & $\sharp$Variables
            & Time & Invariant Type 
            & Interproc & CPAChecker 
            \\
        \hline
        Linear 1
            & 196 & 4 & 1 
            & 1 & 1
            & 3.31s & Linear 
            & 0.01s & 3.42s
            \\
        \hline
        Linear 2
            & 3158 & 7 & 1 
            & 1 & 2
            & 9.86s & Linear 
            & 0.01s & 3.29s
            \\
        \hline
        Linear 3
            & 11102 & 6 & 1
            & 1 & 3
            & 40.24s & Linear 
            & 0.01s & 3.50s
            \\
        \hline
        Linear 4
            & 1143 & 10 & 1
            & 9 & 2
            & 12.54s & Linear 
            & 0.01s & 3.76s
            \\
        \hline
        Linear 5
            & 918 & 8 & 2 
            & 3 & 2
            & 14.47s & Linear 
            & Error & 3.66s
            \\
        \hline
        Poly 1
            & 64 & 7 & 2
            & 1 & 1 
            & 10.51s & Polynomial 
            & Unknown & Unknown 
            \\
        \hline
        Poly 2 
            & 32711 & 85 & 7 
            & 2 & 2
            & 23m43.1s & Polynomial
            & Unknown & Unknown 
            \\
        \hline
        Poly 3 
            & 272 & 17 & 4 
            & 3 & 1 
            & 15.82s & Polynomial 
            & 0.01s & 3.31s 
            \\
        \hline
        Poly 4 
            & 2287 & 112 & 9
            & 2 & 2
            & 13m43.7s & Polynomial
            & Unknown & Unknown 
            \\
        \hline
        Conjunction 1
            & 21247 & 81 & 1 
            & 3 & 2 
            & 20m41.35s & Conjunction
            & 0.01s & 3.16s
            \\
        \hline
    \end{tabular}
    \end{adjustbox}
    \end{center}
    % \end{adjustwidth}
    % \end{minipage}
    \end{center}
    \caption{Experiment Results}
    \label{tab:experiments}
\end{table*}

In this work, we implement our invariant inference framework into a tool called \textsc{Zilu}, 
written in C++ and shell code. 
\textsc{Zilu} uses GSL for selective sampling, LibSVM for machine learning, 
KLEE for concolic testing and Z3 for constraint solving. 
In our experimental evaluation, 
we test \textsc{Zilu} with \LL{Number} loop invariant benchmarks 
in the following form, where $\mathit{Body}$ can have nested loops and conditional choices. 
\[
    \{ \mathit{Pre} \} \mathit{while}(\mathit{Cond}) \{ \mathit{Body} \} \{ \mathit{Post} \}
\]
\LL{Introduce the sources of the benchmark.}
All benchmarks are available from~\cite{zilu}. 

The parameters chosen in our experimental evaluation can be elaborated as follows. 
In the \emph{Sampling} stage, 
the values of all of the program input variables in the random sampling 
and the chosen variables in the selective sampling 
follow the universal distribution over the range of $[-200, 200]$. 
In the \emph{Classification} stage, 
the accuracy of SVM linear learning are set to its maximum value 
in order to generate a absolutely correct classifier if it exists. 
In the \emph{Verification} stage, 
Z3 solver uses `integer' as the type of program variables. 

To evaluate the effectiveness of \textsc{Zilu}, 
we compared it with two available state-of-the-art invariant inference tools: 
Interproc~\cite{cite} and CAPChecker~\cite{cite}, 
which correspond to two different invariant generation methods. 
Interproc is based on abstract interpretation. 
In the experiment, Interproc uses its most expressive abstract domain, i.e., 
the reduced product of polyhedra and linear congruences abstraction. 
Similar to \textsc{Zilu}, Interproc explicitly labels the loop invariants in the loop program. 
We thus can manually check their correctness
and compare them with the invariants generated by our tool \textsc{Zilu}. 
On the other hand, CPACheck (formerly named after BLAST~\cite{cite}) is a software verification tool 
based on the framework of configurable program analysis~\cite{cite}. 
In the experiment, we ask it to generate loop invariant for program predicate abstraction. 
Since CPAChecker functions by proving the program correctness, 
we insert an error branch after each loop program and check for the reachability of the error. 

We present the experiment results in Table~\ref{tab:experiments}. 
Since our invariant reference method has random factors introduced by the sampling sources, 
the presented results are those with median running time in $10$ times of invariant reference process. 
All of the experiments are conducted using x86\_64 Ubuntu 14.04 (kernel 3.13.0-85-generic) 
with 2.3 GHz Intel Core i5 and 4G 1333MHz DDR3. 
In Table~\ref{tab:experiments}, the second column to the seventh column correspond to 
invariant generation details from \textsc{Zilu}. 
`$\sharp$Samples' represents the number of samples generated in the experiment, 
including those from random sampling, selective sampling and counter-example sampling. 
\LL{I need more information on the percentage of the samples. }
`$\sharp$Invariants' stands the numbers of invariants generated by SVM, 
and `$\sharp$Iterations' represents the number of invariant candidates. 
Notice that an invariant generated by SVM becomes an invariant candidate 
when it converges to two previously generated invariants from SVM in one iteration process. 
`$\sharp$Traces' represents the number of loop body traces produced by KLEE. 
and `$\sharp$Variables' represents the number of program variables. 
% In general, when numbers of traces and variables increase in a program,
% the invariant inference difficulty increases.



% section evaluations (end)


%!TEX root = paper.tex

\section{Conclusion and Related Work} % (fold)
\label{sec:related}
In this work, we propose a framework for improving loop invariant learning through active learning. We remark that in theory, we could learn arbitrary mathematical classifier using methods like SVM with kernel methods~\cite{svm:kernel}. Nonetheless, due to the limited proving capability of existing program verification techniques, we focus on invariants in the form of polynomial inequalities or conjunctions of polynomial inequalities.
%Furthermore, we assume there is a bound $k$ on the number of clauses in the variant.
%In practice, we would expect (refer to empirical evidence in Section~\ref{sec:evaluations}) often $k$ is of a small value.


% \begin{table}
%     \begin{center}
%     \begin{tabular}{| l | c | c | l |}
%         \hline
%         Name & Inference Strategy & Inference Source \\
%         \hline
%         ABS & Eager & \\
%         \hline
%         CONS & Eager & \\
%         \hline
%         CEGAR & Lazy & Counterexample \\
%         \hline
%         INTER & Lazy & Counterexample \\
%         \hline
%         G\&C & Eager & Empirical \\
%         \hline
%         ABD & Lazy & Semantic \\
%         \hline
%         DAL & Eager & All Above \\
%         \hline
%     \end{tabular}
%     \end{center}
%     \caption{Existing Invariant Inference Approaches}
%     \label{tab:related}
% \end{table}
This work is related to a large body of approaches on loop invariant generation. 
The existing approaches can be mainly categorized as:
Abstract Interpretation~\cite{cousot1978automatic,mine2006octagon,cousot1979systematic,karr1976affine,vincent2009subpolyhedra},
Constraint Synthesis~\cite{ashutosh2009invgen,michael2003linear,sumit2009constraint},
CounterExample Guided Abstraction Refinement (CEGAR)~\cite{henzinger2003software,thomas2001slam,edmund2003counterexample},
Program Interpolation~\cite{kenneth2010lazy,thomas2004abstractions,kenneth2003interpolation,Kenneth2006lazy},
Guess \& Check~\cite{cormac2001houdini,ernst2007daikon},
Abductive Inference~\cite{isil2013inductive}.
In this section, we compare them with our data-driven active learning approach.

On one hand, invariant inference methods based on Abstract Interpretation and Constraint Synthesis
tend to generate all possible invariants~\cite{mine2006octagon,vincent2009subpolyhedra,ashutosh2009invgen} regardless of
whether they are useful to prove the program correctness or not.
Hence, the invariants inferred by them can be too complex to generate
and thus they fail to prove the program correctness.
On the other hand, other methods based on CEGAR, Interpolation and Abduction
only generate those related to the program verification~\cite{isil2013inductive}.
Thus, they may miss some critical and necessary invariants for the program verification.
Our approach combines their strengths:
we treat the program as a black box
and label samples based on the program verification target,
i.e., the pre-conditions and the post-conditions;
we then infer all possible invariants based on the labels
to prove the program correctness.

Additionally, similar to CEGAR, Guess \& Check, Interpolation and Abduction approaches,
our active learning approach adopts an iterative refinement scheme.
After generating the invariant, we check its correctness and refine it
based on various new information (e.g., counter-examples, new samples)
if it cannot prove the program correctness.
Different from most of the existing refinement approaches,
our method is driven by data samples
rather than syntactic~\cite{cormac2001houdini} or semantic~\cite{ashutosh2009invgen,isil2013inductive} clues.
Hence, it is more flexible and extensible to capture new types of invariants,
and it is platform- and language-independent.
Based on the needs in practice, under-approximation or over-approximation
can also be applied to the data samples with ease.

% section related (end)


\newpage

\bibliographystyle{abbrv}
\bibliography{zeno}

\end{document}
