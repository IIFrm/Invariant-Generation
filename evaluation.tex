%!TEX root = paper.tex

\section{Evaluations} % (fold)
\label{sec:evaluations}


\begin{table*}[t]
    \begin{center}
    % \begin{minipage}{\textwidth}
    % \begin{adjustwidth}{-1in}{-1in}
    \begin{center}
    \begin{adjustbox}{max width=1\textwidth}
    \begin{tabular}{l | r | r | r | r | r | r | r | r | r}
        \hline\hline
        Benchmark 
            & $\sharp$Samples & $\sharp$Invariants & $\sharp$Iterations 
            & $\sharp$Traces & $\sharp$Variables
            & Time & Invariant Type 
            & Interproc & CPAChecker 
            \\
        \hline
        Interproc 1
            & 196 & 4 & 1 
            & 1 & 1
            & 3.31s & Linear 
            & 0.01s & 3.42s
            \\
        \hline
        Interproc 2
            & 3158 & 7 & 1 
            & 1 & 2
            & 9.86s & Linear 
            & 0.01s & 3.29s
            \\
        \hline
        Interproc 3
            & 11102 & 6 & 1
            & 1 & 3
            & 40.24s & Linear 
            & 0.01s & 3.50s
            \\
        \hline
        Interproc 4
            & 1143 & 10 & 1
            & 9 & 2
            & 12.54s & Linear 
            & 0.01s & 3.76s
            \\
        \hline
        Interproc 5
            & 918 & 8 & 2 
            & 3 & 2
            & 14.47s & Linear 
            & Error & 3.66s
            \\
        \hline
        Poly 1
            & 64 & 7 & 2
            & 1 & 1 
            & 10.51s & Polynomial 
            & Unknown & Unknown 
            \\
        \hline
        Poly 2 
            & N.T. & & 
            & & 
            & & 
            & Unknown & Unknown 
            \\
        \hline
        Poly 3 
            & 272 & 17 & 4 
            & 3 & 1 
            & 15.82s & Polynomial 
            & 0.01s & 3.31s 
            \\
        \hline
        Poly 4 
            & N.T. & & 
            & & 
            & & 
            & Unknown & Unknown 
            \\
        \hline
        Conjunction 1
            & 21247 & 81 & 1 
            & 3 & 2 
            & 20m41.35s & Conjunction
            & Error & 3.16s
            \\
        \hline
    \end{tabular}
    \end{adjustbox}
    \end{center}
    % \end{adjustwidth}
    % \end{minipage}
    \end{center}
    \caption{Experiment Results}
    \label{tab:experiments}
\end{table*}

In this work, we implement our invariant inference framework into a tool called \textsc{Zilu}, 
written in C++ and shell code. 
\textsc{Zilu} uses APIs provided by GSL for selective sampling, LibSVM for machine learning, 
KLEE for concolic testing and Z3 for constraint solving. 
In our experimental evaluation, 
we test \textsc{Zilu} with \LL{Number} loop invariant benchmarks 
in the following form, where $\mathit{Body}$ can have nested loops and conditional choices. 
\[
    \{ \mathit{Pre} \} \mathit{while}(\mathit{Cond}) \{ \mathit{Body} \} \{ \mathit{Post} \}
\]
\LL{Introduce the sources of the benchmark.}
All benchmarks are available from~\cite{zilu}. 

The parameters chosen in our experimental evaluation can be elaborated as follows. 
In the \emph{Sampling} stage, 
the values of all of the program input variables in the random sampling 
and the chosen variables in the selective sampling 
follow the universal distribution over the range of $[-200, 200]$. 
In the \emph{Classification} stage, 
the accuracy of SVM linear learning are set to its maximum value 
in order to generate a absolutely correct classifier if it exists. 
In the \emph{Verification} stage, 
Z3 solver uses `integer' as the type of program variables. 
Since our invariant reference process have random factor 
introduced by the sampling sources, 
we run each benchmark for 10 times 
and present the results using the median time in Table~\ref{tab:experiments}. 
All of the experiments are conducted using x86\_64 Ubuntu 14.04 (kernel 3.13.0-85-generic) 
with 2.3 GHz Intel Core i5 and 4G 1333MHz DDR3. 


% section evaluations (end)

