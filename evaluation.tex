%!TEX root = paper.tex

\section{Evaluation} % (fold)
\label{sec:evaluations}
We have implemented our invariant inference framework in a tool, called \textsc{Zilu}~\cite{zilu:repo}.
It is written using a combination of C++ as well as shell codes (for invoking external tools).
\textsc{Zilu} makes use of GSL~\cite{gough2009gnu} to solve equation systems which is necessary for selective sampling and
%% uses
 LibSVM~\cite{chang2011libsvm} as a primitive classification engine for
SVM-based classification. %% based on SVM.
For candidate verification, we %% reuse a large part of
modify the KLEE project~\cite{cadar2008klee} to
symbolically execute
%% generate verification conditions for
 C programs %% and use
prior to invoking Z3~\cite{de2008z3} for checking satisfiability of
condition (4), (5) and (6).
 We remark that KLEE is a symbolic executor; it may
concretely execute the programs and return
 under-approximated
abstraction. This may affect the soundness of our system.
To overcome this problem, we detect those path conditions produced from
concrete executions
and return sound abstraction (i.e. $true$) for them.
 %% KLEE was designed for test case generation and thus its default encoding may result in under-approximation of the program behavior. We have re-implemented the relevant part of KLEE to make sure the right verification conditions are generated.

%Our test subjects include a set of 32 benchmark programs %% which we
%either gathered from the previous publications~\cite{gulwani2008program}\cite{sharma2012interpolants} \cite{gulavani2008automatically} and from
%the software verification competition repository~\cite{beyer:SVCOMP:2013}\cite{Dirk:SVCOMP:2016}
%{or constructed by ourselves}.

To study the effectiveness of \textsc{Zilu}, we choose benchmark C programs from multiple sources.
While the benchmarks
 $zilu\_*$ are constructed by ourselves,
the benchmarks $f2, xy*, pldi08$ are taken from~\cite{gulwani2008program,sharma2012interpolants,gulavani2008automatically}, %% the benchmarks
$interproc*$ from~\cite{jeannet2010interproc},
%% The benchmark
 $terminator\_01\_safe$ %% is gathered
 from~\cite{beyer:SVCOMP:2013} and the remaining
%% are collected
from the software verification competition repository~\cite{Dirk:SVCOMP:2016}.  All benchmarks are available from~\cite{zilu:benchmark}.
%% Note
We remark that the loops in these benchmark programs often contain non-deterministic choices
%% in the loop
which are used to model I/O environment (e.g., an external function call).
As non-determinism is beyond the scope of this work, we syntactically replace these non-deterministic commands with free variables of type boolean. In the future, we would like to investigate how to learn loop invariants in the presence of non-determinism (e.g., random number generation in the loop).
%It would be interesting to investigate how our active learning verification system can be extended to infer non-determinism-based invariants.
 %% which by our assumption is not allowed. We thus transform the programs so that free boolean-type variables are introduced to replace the non-deterministic choice. We remark that the assumption of no non-determinism is less a problem for verification programs in practice as they are often deterministic. These benchmark programs are made non-deterministic often as a way of abstracting away certain complicated (e.g., an external function call) part of the program which is irrelevant to proving/disproving the Hoare triple.
%In our experimental evaluation,
%we test \textsc{Zilu} with \LL{Number} loop invariant benchmarks
%in the following form, where $\mathit{Body}$ can have nested loops and conditional choices.
%\[
%   \{ \mathit{Pre} \} \mathit{while}(\mathit{Cond}) \{ \mathit{Body} \} \{ \mathit{Post} \}
%\]
%% Do not apply this to `add' more content to the paper.
%% It lefts lots of empty space which did the contrary thing.
%% \begin{align*}
%% &Pre&\\
%% &while (Cond) \{&\\
%% &  \quad Body &\\
%% &\} &\\
%% &Post &
%% \end{align*}
%\LL{Introduce the sources of the benchmark.}
 %% All benchmarks are available from~\cite{zilu}.

The parameters in our experiments are set as follows. For random sampling, we generate random values of all input variables of the program from their default ranges. By default, we generate 8 random values for each free variable.
%which would be enlarged if we can not find samples in a few tries.
%Our experiments indicate this $\mathcal{R}$ is a relatively rational range for the initial sampling.
When we invoke LibSVM for classification, the parameter $C$ (i.e., the preference between avoiding misclassifying each training example and enlarging decision boundary) and the inner iteration for SVM learning are set to their maximum value so as to generate only perfect classifiers.
Each invocation of LibSVM's classification engine is set to time out in 600 seconds.
The maximum degree for learning polynomial classifiers is set to be 4. For candidate verification, we encode integer-type variables in the programs as integers in Z3.

%Considering the differences between machine learning problem and our setting,
%\textsc{Zilu} tunes $\textsc{LibSvm}$~\cite{chang2011libsvm} in the following three aspects in order to get a \underline{perfect classifier}.
%\begin{itemize}
%\item Convert \textsc{Svm} model to an explicit classifier.
%The original \textsc{Svm} technique does not explicitly calculate a hyperplane, but emits its own model
%which can be used to do prediction on the given data.
%%This might not a big problem if we do not apply $\textsc{Svm}$ with some kernel method (which is used to classify no linear separable data).
%However, we need a explicit classifier as the loop invariant candidate which can be understood and proceeded later for verification.
%(This is also why we do not apply $\textsc{Svm}$ with kernel methods~\cite{yu2009evolving},
%considering converting $\textsc{Svm}$ models with kernel methods, i.e. $\textsc{Rbf}$ kernel, would be a complicated, sometimes even impossible, task.)
%
%\item As \textsc{Zilu} treasures classification accuracy on the training dataset% than anything else,
%before applying primitive $\textsc{Svm}$ technique, the parameters
%(mainly $C$ which tells the $\textsc{Svm}$ optimization how much you want to avoid misclassifying each training example)
%%different from $\mathcal{C}$ used as loop invariant candidate in our context)
%should be carefully tuned to learn a perfect classifier which perform well on the training points.
%
%\item Validating the learned classifier.
%Checking the classification correctness of the learned classifier
%on $\mathcal{S}^+$ and $\mathcal{S}^-$ is still needed as our setting needs to ensure the learned classifier is a perfect one.
%\textsc{Zilu} also takes $\mathcal{S}^\rightarrow$ to validate the learned classifier(as is shown in Section~\ref{sec:sampling}).
%\end{itemize}

%\begin{table}[t]
%\scriptsize
%\centering
%\caption{Experiment results}
%%\begin{tabular}{l c | c c c c | c c c c | c c }
%%\cline{3-10}
%\begin{tabular}{l c | c c c c| c c c | c c }
%\cline{3-9}
%%& &\multicolumn{4}{|c|}{\textsc{Zilu} with Selective}&\multicolumn{4}{c|}{\textsc{Zilu} without Selective} & & \\
%& &\multicolumn{4}{|c|}{\textsc{Zilu} with Selective}&\multicolumn{3}{c|}{\textsc{Zilu} without Selective} & & \\
%\hline
%%\multicolumn{1}{|c|}{benchmark}&\multicolumn{1}{|c|}{inv type}& $\sharp$r. sample & $\sharp$s. sample & $\sharp$v. sample & time & $\sharp$r. sample & & $\sharp$v. sample & time & \multicolumn{1}{|c|}{Interproc} & \multicolumn{1}{|c|}{CPAChecker} \\
%\multicolumn{1}{|c|}{benchmark}&\multicolumn{1}{|c|}{inv type}& $\sharp$r. sample & $\sharp$s. sampls & $\sharp$v. sample &time(s) & $\sharp$r. sample & $\sharp$v. sample &time(s) & \multicolumn{1}{|c|}{Interproc} & \multicolumn{1}{|c|}{CPAChecker} \\
%\hline % inserts single horizontal line
%\multicolumn{1}{|c|}{afnp2014\_true\text{-}unreach\text{-}call}         	&conjunction	&248 &992 &27 &39.33	&5160 &129 &timeout  & &  \\
%%\multicolumn{1}{|c|}{afnp2014\_true\text{-}unreach\text{-}call}         	&conjunction	&248 &993 &27 &39.33	&5160 & &223.63  & &  \\
%\multicolumn{1}{|c|}{cav12foo1}         									&conjunction 	&228 &912 &20 &51.07	&1980 &48 &168.97  & &  \\
%\multicolumn{1}{|c|}{cav12foo2}         									&conjunction 	&36 &144 &2 &16.09		&260 &6 &15.98  & &  \\
%%\multicolumn{1}{|c|}{cggmp2005\_variant\_true\text{-}unreach\text{-}call}   &conjunction 	&210 &840 & &74045	&2220 & &timeout  & &  \\
%\multicolumn{1}{|c|}{cggmp2005\_variant\_true}   							&conjunction 	&204 &816 &13 &108.57	&1620 &24 &timeout  & &  \\
%\multicolumn{1}{|c|}{conj}         											&polynomial 	&10 &40 &2 &20.48		&70 &1 &28.48  & &  \\
%%\multicolumn{1}{|c|}{conj}         											&conjunction/polynomial & & &  &   &  & & &  & &  \\
%\multicolumn{1}{|c|}{css2003\_true\text{-}unreach\text{-}call}         		&conjunction 	&324 &1296 &33 &57.78	&5080 &125 &258.65  & &  \\
%\multicolumn{1}{|c|}{dis}         											&polynomial 	&22 &88 &1 &24.27  &70   &2 &41.62  & &  \\
%\multicolumn{1}{|c|}{down\_true\text{-}unreach\text{-}call\_1}         		&conjunction 	&264 &1056 &15 &116.21  &1770  &27  &timeout  & &  \\
%\multicolumn{1}{|c|}{down\_true\text{-}unreach\text{-}call\_2}         		&conjunction 	&222 &888 &14 &52.96  &720 &10   &44.99  & &  \\
%\multicolumn{1}{|c|}{f2}         											&linear 		&52 &208 &1 &10.15  &120 &1   &12.03  & &  \\
%\multicolumn{1}{|c|}{f3}         											&linear 		&36 &144 &1 &105.15  &330  &5  &93.04  & &  \\
%%\multicolumn{1}{|c|}{fig1a\_1}         										&linear & & &  &   &  & & &  & &  \\
%%\multicolumn{1}{|c|}{fig1a\_2}         										&linear & & &  &   &  & & &  & &  \\
%\multicolumn{1}{|c|}{interproc\_test1}         								&linear 		&8 &32 &1 &9.21  &40 &2   &10.38  & &  \\
%\multicolumn{1}{|c|}{interproc\_test2}         								&linear 		&84 &96 &1 &11.66  &240  &1  &171.14  & &  \\
%\multicolumn{1}{|c|}{interproc\_test3}         								&linear 		&42 &168 &1 &30.32  &420 &5   &43.34  & &  \\
%\multicolumn{1}{|c|}{interproc\_test4}         								&linear 		&28 &112 &4 &14.8  &240 &5   &38.25  & &  \\
%\multicolumn{1}{|c|}{interproc\_test5}         								&linear 		&32 &128 &1 &9.18  &180 &3   &28.05  & &  \\
%\multicolumn{1}{|c|}{interproc\_test6}         								&polynomial 	&6 &24 &1 &11.83  &70  &2  &18.27  & &  \\
%\multicolumn{1}{|c|}{interproc\_test8}         								&conjunction 	&24 &96 &6 &22.13  &220  &2  &97.47  & &  \\
%\multicolumn{1}{|c|}{interproc\_test11}         							&linear 		&24 &96 &2 &24.79  &180 &2   &206.07  & &  \\
%\multicolumn{1}{|c|}{lili2}         										&linear 		&84 &336 &4 &21.28  &720  &9  &82.24  & &  \\
%\multicolumn{1}{|c|}{linear6}         										&linear 		&18 &72 &1 &16.19  &270 &3  &179.39  & &  \\
%\multicolumn{1}{|c|}{multivar\_true\text{-}unreach\text{-}call1}         	&conjunction 	&68 &272 &3 &16.84  &220 &4   &15.22  & &  \\
%\multicolumn{1}{|c|}{terminator\_01\_safe}         							&linear 		&6 &24 &1 &9.2  &90  &4  &13.06  & &  \\
%\multicolumn{1}{|c|}{test}         											&linear 		&28 &112 &1 &10.19  &420 &2  &24.51  & &  \\
%\multicolumn{1}{|c|}{test2}         										&conjunction 	&56 &218 &3 &173.09  &280  &7  &timeout  & &  \\
%%\multicolumn{1}{|c|}{up\_true\text{-}unreach\text{-}call\_1}         		&conjunction & & &  &   &  & & &  & &  \\
%\multicolumn{1}{|c|}{up\_true\text{-}unreach\text{-}call\_2}         		&conjunction 	&240 &960 &14 &84.02  &540 &7   &89.77  & &  \\
%%\multicolumn{1}{|c|}{xeq10}         										&linear & & &  &   &  & & &  & &  \\
%\multicolumn{1}{|c|}{xle10}         										&linear 		&6 &24 &1 &8.57  &60 &3   &12.58  & &  \\
%\multicolumn{1}{|c|}{xy10}         											&linear 		&162 &648 &4 &39.69  &840  &11  &40.51  & &  \\
%\multicolumn{1}{|c|}{xyle0}         										&polynomial 	&60 &240 &4 &85.61  &620   &12 &472.68  & &  \\
%\hline
%\end{tabular}
%\label{tbl:stats}
%\end{table}

%\begin{table*}[t]
%    \begin{center}
%    % \begin{minipage}{\textwidth}
%    % \begin{adjustwidth}{-1in}{-1in}
%    \begin{center}
%    \begin{adjustbox}{max width=1\textwidth}
%    \begin{tabular}{l | r | r | r | r | r | r | r | r | r}
%        \hline\hline
%        Benchmark
%            & $\sharp$Random Samples & $\sharp$Selective Samples & $\sharp$Iterations
%            & $\sharp$Traces & $\sharp$Variables
%            & Time & Invariant Type
%            & Interproc & CPAChecker
%            \\
%        \hline
%        Linear 1
%            & 196 & 4 & 1
%            & 1 & 1
%            & 3.31s & Linear
%            & 0.01s & 3.42s
%            \\
%        \hline
%        Linear 2
%            & 3158 & 7 & 1
%            & 1 & 2
%            & 9.86s & Linear
%            & 0.01s & 3.29s
%            \\
%        \hline
%        Linear 3
%            & 11102 & 6 & 1
%            & 1 & 3
%            & 40.24s & Linear
%            & 0.01s & 3.50s
%            \\
%        \hline
%        Linear 4
%            & 1143 & 10 & 1
%            & 9 & 2
%            & 12.54s & Linear
%            & 0.01s & 3.76s
%            \\
%        \hline
%        Linear 5
%            & 918 & 8 & 2
%            & 3 & 2
%            & 14.47s & Linear
%            & Error & 3.66s
%            \\
%        \hline
%        Poly 1
%            & 64 & 7 & 2
%            & 1 & 1
%            & 10.51s & Polynomial
%            & Unknown & Unknown
%            \\
%        \hline
%        Poly 2
%            & 32711 & 85 & 7
%            & 2 & 2
%            & 23m43.1s & Polynomial
%            & Unknown & Unknown
%            \\
%        \hline
%        Poly 3
%            & 272 & 17 & 4
%            & 3 & 1
%            & 15.82s & Polynomial
%            & 0.01s & 3.31s
%            \\
%        \hline
%        Poly 4
%            & 2287 & 112 & 9
%            & 2 & 2
%            & 13m43.7s & Polynomial
%            & Unknown & Unknown
%            \\
%        \hline
%        Conjunction 1
%            & 21247 & 81 & 1
%            & 3 & 2
%            & 20m41.35s & Conjunction
%            & 0.01s & 3.16s
%            \\
%        \hline
%    \end{tabular}
%    \end{adjustbox}
%    \end{center}
%    % \end{adjustwidth}
%    % \end{minipage}
%    \end{center}
%    \caption{Experiment Results}
%    \label{tab:experiments}
%\end{table*}

In order to show the relevance of selective sampling, we measure the performance of \textsc{Zilu}~\cite{zilu:repo} with or without selective sampling. In addition, we compare \textsc{Zilu} with a state-of-the-art invariant inference tool Interproc~\cite{jeannet2010interproc}. We remark that though many other approaches have been reported~\cite{sharma2012interpolants,sharma2013verification,DBLP:conf/esop/0001GHALN13,sharma2014invariant}, their tools are no longer not available for evaluation.
Interproc generates invariants based on abstract interpretation. In the experiments, it is set to use its most expressive abstract domain, i.e., the reduced product of polyhedra and linear congruences abstraction. %Similar to \textsc{Zilu}, Interproc explicitly labels the loop invariants in the loop program.
We manually check the generated invariants for correctness. %CPAChecker (formerly BLAST~\cite{henzinger2003software}) is a software verification tool. In the experiments, CPAChecker is configured to generate loop invariants to prove the same Hoare triple.
We remark that the comparison should be taken with a grain of salt as the methods are different. The experiment results are presented in Table~\ref{tbl:stats}. All of the experiments are conducted using x64 Ubuntu 14.04.1 (kernel 3.19.0-59-generic) with 3.60 GHz Intel Core i7 and 32G DDR3.

\begin{table}[t]
\scriptsize
\centering
\caption{Experiment results}
\begin{tabular}{l c | c c c | c c c | c }
\cline{3-8}
& &\multicolumn{3}{|c|}{\textsc{Zilu} + Selective Sampling}&\multicolumn{3}{c|}{\textsc{Zilu} - Selective Sampling} & \\
\hline
\multicolumn{1}{|c|}{benchmark}&\multicolumn{1}{|c|}{inv type}& $\sharp$sample & $\sharp$iteration & time(s) & $\sharp$sample & $\sharp$iteration &time(s) & \multicolumn{1}{|c|}{Interproc} \\
\hline % inserts single horizontal line
%\multicolumn{1}{|c|}{TEMPLATE} 		        	&polynomial 	& & &  &  &  &  & &  \\
%\hline
\multicolumn{1}{|c|}{f2}         						&linear 		&260 &1 &\textbf{10.15}  		&120 &1   &12.03  			&\multicolumn{1}{|c|}{\cmark} \\
\multicolumn{1}{|c|}{xy10} 	        					&linear 		&810 &4 &\textbf{39.69} 		&840  &11  &40.51  			&\multicolumn{1}{|c|}{\cmark} \\
\multicolumn{1}{|c|}{xyz}   	      					&linear 		&180 &1 &105.15  				&330  &5  &\textbf{93.04}  	&\multicolumn{1}{|c|}{\cmark} \\
\multicolumn{1}{|c|}{xy0\_1}         					&conjunction 	&1140 &20 &\textbf{51.07}		&1980 &48 &168.97  			&\multicolumn{1}{|c|}{\cmark} \\
\multicolumn{1}{|c|}{xy0\_2}         					&conjunction 	&180  &2 &16.09					&260 &6 &\textbf{15.98}  	&\multicolumn{1}{|c|}{\cmark} \\
\multicolumn{1}{|c|}{pldi08} 		        			&disjunction 	&- & - &timeout  				&-  &-  &timeout  		&\multicolumn{1}{|c|}{\xmark} \\

%\hline
\multicolumn{1}{|c|}{interproc1}         				&linear 		&40 &1 &\textbf{9.21}  			&40 &2   &10.38  			&\multicolumn{1}{|c|}{\cmark} \\
\multicolumn{1}{|c|}{interproc2}         				&linear 		&600 &1 &\textbf{11.66} 		&240  &1  &171.14  			&\multicolumn{1}{|c|}{\cmark} \\
\multicolumn{1}{|c|}{interproc3}         				&linear 		&210 &1 &\textbf{30.32}  		&420 &5   &43.34  			&\multicolumn{1}{|c|}{\cmark} \\
\multicolumn{1}{|c|}{interproc4}         				&linear 		&140 &4 &\textbf{4.8}  			&240 &5   &38.25  			&\multicolumn{1}{|c|}{\xmark} \\
\multicolumn{1}{|c|}{interproc5}         				&linear 		&160 &1 &\textbf{9.18}  		&180 &3   &28.05  			&\multicolumn{1}{|c|}{\cmark} \\

%\hline
\multicolumn{1}{|c|}{zilu\_linear1}         			&linear 		&120 &2 &\textbf{24.79}  		&180 &2   &206.07  			&\multicolumn{1}{|c|}{\xmark} \\
\multicolumn{1}{|c|}{zilu\_linear2}         			&linear 		&420 &4 &\textbf{21.28}  		&720  &9  &82.24  			&\multicolumn{1}{|c|}{\cmark} \\
\multicolumn{1}{|c|}{zilu\_linear3}         			&linear 		&90 &1 &\textbf{16.19}  		&270 &3  &179.39  			&\multicolumn{1}{|c|}{\cmark} \\
\multicolumn{1}{|c|}{zilu\_linear4}         			&linear 		&140 &1 &\textbf{10.19}  		&420 &2  &24.51  			&\multicolumn{1}{|c|}{\cmark} \\
\multicolumn{1}{|c|}{zilu\_linear5}         			&linear 		&30 &1 &\textbf{8.57}  			&60 &3   &12.58  			&\multicolumn{1}{|c|}{\cmark} \\

\multicolumn{1}{|c|}{zilu\_poly1}         				&polynomial 	&50 &2 &\textbf{20.48}			&70 &1 &28.48  				&\multicolumn{1}{|c|}{\xmark} \\
\multicolumn{1}{|c|}{zilu\_poly2}         				&polynomial 	&110  &1 &\textbf{24.27}  		&70   &2 &41.62  			&\multicolumn{1}{|c|}{\xmark} \\
\multicolumn{1}{|c|}{zilu\_poly3}         				&polynomial 	&30 &1 &\textbf{11.83}  		&70  &2  &18.27  			&\multicolumn{1}{|c|}{\xmark} \\
\multicolumn{1}{|c|}{zilu\_poly4}         				&polynomial 	&- &-&timeout  				&-  &-  &timeout 			&\multicolumn{1}{|c|}{\xmark} \\
\multicolumn{1}{|c|}{zilu\_poly5}         				&polynomial 	&- &- &timeout  			&-  &-  &timeout  		&\multicolumn{1}{|c|}{\xmark} \\
\multicolumn{1}{|c|}{zilu\_poly6}         				&polynomial 	&300 &4 &\textbf{85.61}  		&620   &12 &472.68  		&\multicolumn{1}{|c|}{\xmark} \\

\multicolumn{1}{|c|}{zilu\_conj1}         				&conjunction 	&120 &6 &\textbf{22.13}  		&220  &2  &97.47  			&\multicolumn{1}{|c|}{\xmark} \\
\multicolumn{1}{|c|}{zilu\_conj2}         				&conjunction 	&280 &3 &\textbf{173.09}  		&-  &-  &timeout  		&\multicolumn{1}{|c|}{\cmark} \\

%\hline
\multicolumn{1}{|c|}{terminator\_01\_safe}         		&linear 		&30 &1 &\textbf{9.2}  			&90  &4  &13.06  			&\multicolumn{1}{|c|}{\cmark} \\
%\hline
\multicolumn{1}{|c|}{afnp2014\_true}         			&conjunction	&1240 &27 &\textbf{39.33}		&- &- &timeout  		&\multicolumn{1}{|c|}{\xmark} \\
\multicolumn{1}{|c|}{multivar\_true\_1}         		&conjunction 	&340 &3 &16.84  				&220 &4   &\textbf{15.22}  	&\multicolumn{1}{|c|}{\cmark} \\
\multicolumn{1}{|c|}{cggmp2005\_variant}   				&conjunction 	&1020 &13 &\textbf{108.57}		&- &- &timeout  		&\multicolumn{1}{|c|}{\cmark} \\
\multicolumn{1}{|c|}{css2003\_true}         			&conjunction 	&1420 &33 &\textbf{57.78}		&5080 &125 &258.65  		&\multicolumn{1}{|c|}{\cmark} \\
\multicolumn{1}{|c|}{up\_true\_2}         				&conjunction 	&1200 &14 &\textbf{84.02}  		&540 &7   &89.77  			&\multicolumn{1}{|c|}{\cmark} \\
\multicolumn{1}{|c|}{down\_true\_1}         			&conjunction 	&1320  &15 &\textbf{116.21}  	&-  &-  &timeout  		&\multicolumn{1}{|c|}{\cmark} \\
\multicolumn{1}{|c|}{down\_true\_2}         			&conjunction 	&1110 &14 &52.96  				&720 &10   &\textbf{44.99}  &\multicolumn{1}{|c|}{\cmark} \\

\hline
\end{tabular}
\label{tbl:stats}
\end{table}


The first column in the table shows the name of the benchmark program. The second shows column the type of invariant required for proving the Hoare triple. The next three columns present statistics of \textsc{Zilu} with selective sampling, i.e., the number of samples generated in total, the number of learn-and-check iterations and the total execution time. The next three columns present the corresponding statistics of \textsc{Zilu} without selective sampling.
The last column shows the result of Interproc, i.e., whether it generates a correct invariant. We do not show the time of Interproc because its result may not be correct and because it does not prove/disprove the Hoare triples.
%%\LL{I need more information on the percentage of the samples. }
%`$\sharp$Invariants' stands the numbers of invariants generated by SVM,
%and `$\sharp$Iterations' represents the number of invariant candidates.
%Notice that an invariant generated by SVM becomes an invariant candidate
%when it converges to two previously generated invariants from SVM in one iteration process.
%`$\sharp$Traces' represents the number of loop body traces produced by KLEE~\cite{cadar2008klee}.
%and `$\sharp$Variables' represents the number of program variables.
% In general, when numbers of traces and variables increase in a program,
% the invariant inference difficulty increases.
%The seventh column gives the time used for the invariant generation
%and the eighth column gives the invariant type of the generated invariant.
%We remark that since \textsc{Zilu} relies on random sampling, all numbers collected from \textsc{Zilu} in the table are the average of 10 experiments.

We have the following observations based on the experiment results. First, selective sampling is helpful in reducing the number of learn-and-check iterations.
On average, the number of iterations is reduced from 10.84 to 4.96. As a result, the time required for proving the program is often reduced.
\emph{We would like to highlight that for 12 out of 32 cases, \textsc{Zilu} is able to learn the correct invariant with one iteration}, whereas this is only the case for 3 programs without selective sampling. It implies that even if a program is too complicated to be verified through symbolic execution, \textsc{Zilu} may be able to learn the correct invariant with only random sampling and selective sampling.
This is particularly useful for handling real-world programs, i.e., in such a case, \textsc{Zilu} may be able to generate an invariant which can be used to manually verify the program.

Second, \textsc{Zilu} is reasonably efficient. All loop invariants are learned within three minutes. It implies that selective sampling converges reasonably fast.
We observe that \textsc{Zilu} often takes more time to learn conjunctive invariants. This is because Algorithm~\ref{alg:conjunctiveSVM} may invoke SVM classification many times in one learn-and-check iteration.
\textsc{Zilu} times out in three cases. One is $pldi08$, which requires a disjunctive invariant which is not supported by \textsc{Zilu} currently. The other two are \emph{zilu\_poly4} and \emph{zilu\_poly5}, which require polynomial invariants which are within the capability of \textsc{Zilu}. Our investigation reveals that the reason is that after we match the program states to a higher dimension (using function $mapToDegree$), the values needed for selective sampling in the higher-dimension space may not be feasible in the original space. For example, assume a program with an integer variable $x$ which can be any value in the set $\{\cdots, -1, 0,  1, 2, 3, \cdots\}$. After the mapping, $(x, x^2)$ is defined only with the value in \{$\cdots, (-1,1), (0,0), (1,1), (2, 4), (3, 9)\cdots$\}. As a result, a sample selected according to selective sampling may often be infeasible, i.e., the equation we need to solve for selective sampling has no solution.
%What's more, the problem becomes much worse as the degree increases, as these undefined holes contains much more points than the defined samples, which is also the reason that we restrict the degree of our $polynomial$ algorithm up to 4.

%To illustrate the reason, consider the case of benchmark \emph{poly 1}, the loop invariant is $x^2 \le y^2$.
%However, the general form of an order-2 polynomial with 2 variables
%is: $a \cdot x^2 + b \cdot y^2 + c \cdot x y + d \cdot x + e \cdot y + f \geq 0$. It takes XXX iterations before $c$, $d$, $e$ and $f$ to converge to 0.
Lastly, the results show that \textsc{Zilu} complements existing tools. For instance, \textsc{Zilu} can automatically generate polynomial loop invariants
which are often beyond the capability of Interproc. Interproc is usually fast, i.e., it generates invariants within seconds. However, the invariant generated by Interproc may not be correct.
%This can be demonstrated using our running example introduced in Section~\ref{sec:introduction}.
%Since we have the loop condition $x < y$, it is impossible to execute the loop body when $(x \ge 0) \land (y < 0)$.
%However, Interproc outputs the loop invariant corresponding to this execution trace,
%because the invariants in Interproc are global constraints
%without considering their generation paths and conditions.

% section evaluations (end)
