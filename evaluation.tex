%!TEX root = paper.tex

\section{Evaluations} % (fold)
\label{sec:evaluations}

In this work, we implement our invariant inference framework into a tool called \textsc{Zilu}, 
written in C++ and shell code. 
\textsc{Zilu} uses APIs provided by other projects, like GSL, LibSVM, KLEE and Z3, 
for machine learning, concolic testing and constraint solving. 

In our experimental evaluation, 
we test \textsc{Zilu} with \LL{Number} loop invariant benchmarks 
in the following form, where $\mathit{Body}$ can have nested loops and conditional choices. 
\[
    \{ \mathit{Pre} \} \mathit{while}(\mathit{Cond}) \{ \mathit{Body} \} \{ \mathit{Post} \}
\]
\LL{Introduce the sources of the benchmark.}
All benchmarks are available from~\cite{zilu}. 

Since our invariant reference process have random factor 
introduced by the sampling sources, 
we run each benchmark for 10 times and present their average in Table~\ref{tab:experiments}.  

\begin{table*}[t]
    \begin{center}
    % \begin{minipage}{\textwidth}
    % \begin{adjustwidth}{-1in}{-1in}
    \begin{center}
    \begin{adjustbox}{max width=1.0\textwidth}
    \begin{tabular}{l | r | r | r | r | r | r | r}
        \hline\hline
        Benchmark 
            & $\sharp$Samples & $\sharp$Invariants & $\sharp$Iterations 
            & $\sharp$Traces & $\sharp$Variables
            & Time & Invariant Type \\
        \hline
        Interproc 1
            & 196 & 4 & 1 
            & 1 & 1
            & 3.323s & Linear \\
        \hline
        Interproc 2
            & 3158 & 7 & 1 
            & 1 & 2
            & 9.897s & Linear \\
        \hline
        Interproc 3
            & 11102 & 6 & 1
            & 1 & 3
            & 40.213s & Linear \\
        \hline
        Interproc 4
            & 1143 & 10 & 1
            & 9 & 2
            & 12.675s & Linear \\
        \hline
        Interproc 5
            & 918 & 8 & 2 
            & 3 & 2
            & 14.477s & Linear \\
        \hline
        Poly 1
            & 64 & 7 & 2
            & 1 & 1 
            & 10.525s & Polynomial \\
        \hline
        Poly 2 
            & N.T. \\
        \hline
        Poly 3 
            & 272 & 17 & 4 
            & 3 & 1 
            & 15.851s & Polynomial \\
        \hline
        Poly 4 
            & N.T. \\
        \hline
        Conjunction 1
            & N.U. \\
        \hline
    \end{tabular}
    \end{adjustbox}
    \end{center}
    % \end{adjustwidth}
    % \end{minipage}
    \end{center}
    \caption{Experiment Results}
    \label{tab:experiments}
\end{table*}


% section evaluations (end)

