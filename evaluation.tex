\section{Evaluation} % (fold)
\label{sec:evaluations}
We have implemented our approach for loop-invariant generation in a tool called \textsc{Zilu} (available at~\cite{zilu:repo}).
\textsc{Zilu} is written using a combination of C++ and shell codes (for invoking external tools). It makes use of GSL~\cite{gough2009gnu} to solve equation systems; and uses LibSVM~\cite{chang2011libsvm} for SVM-based classification.
For candidate-invariant verification, we modify the KLEE project~\cite{cadar2008klee} to symbolically execute C programs prior to invoking Z3~\cite{de2008z3} for checking satisfiability of
condition (4), (5) and (6). We remark that KLEE is a concolic testing engine and thus it may concretely execute the programs and return under-approximated abstraction. This may affect the soundness of our system.
To overcome this problem, we detect those path conditions produced from concrete executions and return sound abstraction (i.e., $true$). % for them.

Our evaluation subjects include a set of XXX C programs we find from previous publications (e.g.,~\cite{DBLP:conf/pldi/GulwaniSV08,sharma2012interpolants,gulavani2008automatically,jeannet2010interproc,isil2013inductive}) as well as software verification competitions~\cite{Dirk:SVCOMP:2016} %%beyer:SVCOMP:2013,
(excluding those which cannot be made to satisfy our assumptions). All evaluated programs are available at~\cite{zilu:repo}. We remark that the loops in these benchmark programs often contain non-deterministic choices, which are often used to model I/O environment (e.g., an external function call). As non-determinism is beyond the scope of this work, we replace these non-deterministic commands with free boolean variables. The parameters in our experiments are set as follows. For random sampling, we generate 8 random values for every input variable of a program from their default ranges. Furthermore, during selective sampling, in addition to samples nearby the classification boundary, we add a few random samples in order to improve convergence. This is a strategy adopted from~\cite{DBLP:conf/icml/SchohnC00}. The ratio between random samples and selective samples is 1:4. When we invoke LibSVM for classification, the parameter $C$ (which controls the trade-off between avoiding misclassifying training examples and enlarging decision boundary) and the inner iteration for SVM learning are set to their maximum value so that it generates only perfect classifiers. During candidate verification, integer-type variables in programs are encoded as integers in Z3 (not as bit vectors). Since we have different ways of setting the samples for classification, e.g., by setting the two sets of samples $P$ and $N$ differently as discussed in~Section~\ref{alternative}, and different classification algorithms (linear vs. polynomial or conjunctive vs. disjunctive), we simultaneously try all combinations and terminate as soon as either the Hoare triple is proved or disproved. For polynomial inequalities, the maximum degree is bounded by 4. In order to give priority to simpler invariants, we look for a polynomial classifier with degree $d$ only if we cannot find any polynomial classifier with lower degree. All of the experiments are conducted using x64 Ubuntu 14.04.1 (kernel 3.19.0-59-generic) with 3.60 GHz Intel Core i7 and 32G DDR3.

\begin{table}[t]
\scriptsize
\centering
\caption{Summarized Comparison between ZILU with/without Selective Sampling}
\begin{tabular}{ c | c | c }
 & \textsc{Zilu} & \textsc{Zilu} - Selective Sampling \\
\hline
No. of programs proved 			& 35 & 31		\\
Average time for proving a program & 27 & 31  			\\
Average number of guess-and-check iterations & 2.9 & 3.3  			\\
No. of programs proved with one guess-and-check iteration & 10 & 2		\\
Average number of total samples & 784 & 872  			\\
\hline
\end{tabular}
\label{summary}
\end{table}

\begin{table}[t]
\scriptsize
\centering
\caption{Comparison between ZILU with/without Selective Sampling}
\begin{tabular}{l c | c c c | c c c |}
\cline{3-8}
& &\multicolumn{3}{|c|}{\textsc{Zilu}}&\multicolumn{3}{c|}{\textsc{Zilu} - Selective Sampling}\\
\hline
\multicolumn{1}{|c|}{benchmark}&\multicolumn{1}{|c|}{invariant type}& $\sharp$sample & $\sharp$iteration & time(s) & $\sharp$sample & $\sharp$iteration &time(s) \\
\hline
% \multicolumn{1}{|c|}{list the programs} & linear inequality 			& 107 & \textbf{1} &\textbf{4}	& \textbf{100} & 2 & \textbf{4} 	\\
% \multicolumn{1}{|c|}{list the programs} & polynomial inequality			& 107 & \textbf{1} &\textbf{4}	& \textbf{100} & 2 & \textbf{4} 	\\
% \multicolumn{1}{|c|}{list the programs} & conjunction of linear inequalities 			& 107 & \textbf{1} &\textbf{4}	& \textbf{100} & 2 & \textbf{4} 	\\
% \multicolumn{1}{|c|}{list the programs} & disjunction of linear inequalities 			& 107 & \textbf{1} &\textbf{4}	& \textbf{100} & 2 & \textbf{4} 	\\


\multicolumn{1}{|c|}{05~\cite{isil2013inductive}}		&linear			&\textbf{111}	&\textbf{1}	&8			       &117	&2	&\textbf{6}						\\
\multicolumn{1}{|c|}{21~\cite{isil2013inductive}}		&linear			&\textbf{154}	&\textbf{1}	&\textbf{9}	       &to	&to	&to									\\
\multicolumn{1}{|c|}{23~\cite{isil2013inductive}}		&conjunctive	&\textbf{178}	&\textbf{3}	&\textbf{10}       &229	&\textbf{3}	&12						\\
\multicolumn{1}{|c|}{28~\cite{isil2013inductive}}		&conjunctive	&\textbf{211}	&\textbf{3}	&19			       &236	&4	&\textbf{9}						\\
\multicolumn{1}{|c|}{30~\cite{isil2013inductive}}		&conjunctive	&\textbf{1984}	&\textbf{47}	&\textbf{34}	&to	&to	&to									\\
\multicolumn{1}{|c|}{35~\cite{isil2013inductive}}		&linear			&\textbf{145}	&\textbf{2}	&\textbf{8}	       &153	&\textbf{2}	&11						\\
\multicolumn{1}{|c|}{43~\cite{isil2013inductive}}		&linear			&269	&\textbf{1}	&32					       &\textbf{245}	&2	&\textbf{16}			\\
\multicolumn{1}{|c|}{bound~\cite{isil2013inductive}}	&linear			&52	&\textbf{1}	&\textbf{6}				       &\textbf{49}	&2	&15						\\
\multicolumn{1}{|c|}{down~\cite{gupta2009invgen}}		&linear			&\textbf{207}	&\textbf{3}	&\textbf{19}       &278	&4	&24								\\
\multicolumn{1}{|c|}{f2~\cite{zilu:repo}}				&linear			&\textbf{84}	&\textbf{1}	&\textbf{15}       &115	&2	&18								\\
% \multicolumn{1}{|c|}{fig1c}								&linear			&179	&4	&\textbf{12}				       &\textbf{112}	&\textbf{2}	&\textbf{12}	\\
\multicolumn{1}{|c|}{fm11~\cite{schwartznon}}			&conjunctive	&\textbf{171}	&\textbf{2}	&\textbf{32}       &313	&7	&70								\\
\multicolumn{1}{|c|}{fig1b}								&disjunctive	&\textbf{3533} 	& \textbf{7} &\textbf{34} 	   & 6400 & 10 		& 51 					\\

\multicolumn{1}{|c|}{interproc1~\cite{jeannet2010interproc}}	&linear			&\textbf{120}	&\textbf{5}	&\textbf{15}   &175	&8	&34								\\
\multicolumn{1}{|c|}{interproc2~\cite{jeannet2010interproc}}	&linear			&\textbf{84}	&\textbf{1}	&\textbf{51}	&to	&to	&to									\\
\multicolumn{1}{|c|}{interproc3~\cite{jeannet2010interproc}}	&linear			&\textbf{110}	&\textbf{1}	&\textbf{30}   &206	&2	&32								\\
\multicolumn{1}{|c|}{interproc4~\cite{jeannet2010interproc}}	&linear			&\textbf{81}	&\textbf{1}	&47			   &100	&2	&\textbf{22}					\\
\multicolumn{1}{|c|}{multivar\_1~\cite{jeannet2010interproc}}	&conjunctive	&126	&\textbf{2}	&33					   &\textbf{124}	&\textbf{2}	&\textbf{16}	\\
\multicolumn{1}{|c|}{pldi08\_fig1~\cite{gulavani2008automatically}}&disjunctive	& \textbf{583} & \textbf{2} & \textbf{7}   &798 & \textbf{2} & 13						\\
\multicolumn{1}{|c|}{pldi08\_fig7~\cite{gulavani2008automatically}}	&linear		&\textbf{40}	&\textbf{1}	&\textbf{7}	   &45	&\textbf{1}	&15							\\
\multicolumn{1}{|c|}{terminator\_01~\cite{isil2013inductive}}	&linear			&52	&\textbf{1}	&\textbf{7}				   &\textbf{50}	&\textbf{1}	&15				\\
\multicolumn{1}{|c|}{up\_true\_2~\cite{isil2013inductive}}		&conjunctive	&\textbf{328}	&6	&62					   &421	&\textbf{5}	&\textbf{60}			\\
\multicolumn{1}{|c|}{xle10~\cite{sharma2012interpolants}}		&linear 		&52	&\textbf{1}	&\textbf{7}				   &\textbf{49}	&2	&15						\\
\multicolumn{1}{|c|}{xy0\_1~\cite{sharma2012interpolants}}		&conjunctive	&\textbf{224}	&5	&\textbf{23}		   &230	&\textbf{4}	&25						\\
\multicolumn{1}{|c|}{xy0\_2~\cite{sharma2012interpolants}}		&conjunctive	&\textbf{144}	&\textbf{3}	&29			   &194	&4	&\textbf{27}					\\
\multicolumn{1}{|c|}{xy4\_1~\cite{sharma2012interpolants}}		&conjunctive	&\textbf{181}	&\textbf{4}	&\textbf{18}   &221	&\textbf{4}	&24						\\
\multicolumn{1}{|c|}{xyle0~\cite{sharma2012interpolants}}		&polynomial 	&251	&\textbf{2}	&\textbf{76}		   &\textbf{182}	&3	&81						\\
\multicolumn{1}{|c|}{xyz\_2~\cite{sharma2012interpolants}}		&conjunctive	&422	&8	&38							   &\textbf{394}	&\textbf{5}	&\textbf{28}	\\

\multicolumn{1}{|c|}{zilu\_conj1}	&conjunctive				&\textbf{214}	&\textbf{2}	&\textbf{45}	               &to	&to	&to								\\
\multicolumn{1}{|c|}{zilu\_disj1}	&disjunctive				& \textbf{2740} & \textbf{4} &\textbf{46}                  &4160 	& \textbf{4} &{65} 				\\
\multicolumn{1}{|c|}{zilu\_disj2}	&disjunctive				& \textbf{3193} & \textbf{3} & \textbf{27}	               &5887 & \textbf{3} & 30  				\\
\multicolumn{1}{|c|}{zilu\_disj3}	&disjunctive				& \textbf{7382} & \textbf{3} &\textbf{29}	               &8050 & 4  & 40 		  				 \\
\multicolumn{1}{|c|}{zilu\_poly1}	&polynomial					&\textbf{57}	&\textbf{3}	&\textbf{42}	               &136	&5	&115						\\
\multicolumn{1}{|c|}{zilu\_poly3}	&polynomial					&49	&\textbf{2}	&\textbf{17}				               &\textbf{41} &\textbf{2}	&21			\\
\multicolumn{1}{|c|}{zilu\_poly6}	&polynomial					&\textbf{163}	&\textbf{3}	&\textbf{74}	               &217	&\textbf{3}	&75					\\

\hline
\end{tabular}
\label{tbl:stats}
\end{table}

%\begin{table}[t]
%\scriptsize
%\centering
%\caption{Comparison between ZILU with/without Selective Sampling}
%\begin{tabular}{l c | c c c | c c c |}
%\cline{3-8}
%& &\multicolumn{3}{|c|}{\textsc{Zilu} + Selective Sampling}&\multicolumn{3}{c|}{\textsc{Zilu} - Selective Sampling}\\
%\hline
%\multicolumn{1}{|c|}{benchmark}&\multicolumn{1}{|c|}{inv type}& $\sharp$sample & $\sharp$iteration & time(s) & $\sharp$sample & $\sharp$iteration &time(s) \\
%\hline
%\multicolumn{1}{|c|}{05~\cite{isil2013inductive}}				&linear 			& 107 & \textbf{1} &\textbf{4}	& \textbf{100} & 2 & \textbf{4} 	\\
%\multicolumn{1}{|c|}{21~\cite{isil2013inductive}}				&linear 			& \textbf{133} & \textbf{1} & \textbf{4} & to & to & to 			\\
%\multicolumn{1}{|c|}{23~\cite{isil2013inductive}}				&conjunctive		& \textbf{255} & \textbf{3} & \textbf{5}	&  283 & 8 & 9 			\\
%\multicolumn{1}{|c|}{28~\cite{isil2013inductive}}				&conjunctive		& 307 & \textbf{4} & 8	& \textbf{233} & \textbf{4} & \textbf{7} 	\\
%\multicolumn{1}{|c|}{30~\cite{isil2013inductive}}				&conjunctive		& \textbf{1565} & \textbf{43} & \textbf{26} & to & to & to 			\\
%\multicolumn{1}{|c|}{35~\cite{isil2013inductive}}				&linear 			& 133 & \textbf{1} &\textbf{4}	& \textbf{120} & 2 & \textbf{4} 	 \\
%\multicolumn{1}{|c|}{43~\cite{isil2013inductive}}				&linear 			& \textbf{300} & \textbf{1} & 6	& 323 & 3 & \textbf{5} 				 \\
%\multicolumn{1}{|c|}{bound~\cite{isil2013inductive}}			&linear 			& 73 & \textbf{1} & \textbf{4} & \textbf{70} & 2 & \textbf{4} 		 \\
%\multicolumn{1}{|c|}{down~\cite{gupta2009invgen}}				&linear 			& \textbf{300} & \textbf{4} &\textbf{5}	& 350 & \textbf{4} & 6 		\\
%\multicolumn{1}{|c|}{f2~\cite{zilu:repo}}						&linear 			& \textbf{100} & \textbf{1} &\textbf{4}	& 140 & 2 & 4 				\\
%\multicolumn{1}{|c|}{fig1b}									&disjunctive		& \textbf{3533} & \textbf{7} & \textbf{34} & 6400 & 10 & 51 		\\
%\multicolumn{1}{|c|}{fm11~\cite{schwartznon}}					&conjunctive		&\textbf{166} & \textbf{2} &\textbf{12}	& to & to  & to 			 \\
%\multicolumn{1}{|c|}{interproc1~\cite{jeannet2010interproc}}	&linear 			& \textbf{293} & \textbf{5} & \textbf{6} & to & to & to 			 \\
%\multicolumn{1}{|c|}{interproc2~\cite{jeannet2010interproc}}	&linear 			& 133 & \textbf{1} &\textbf{7}	& \textbf{93} & 2 & 9 				 \\
%\multicolumn{1}{|c|}{interproc3~\cite{jeannet2010interproc}}	&linear 			& \textbf{180} & \textbf{1} & 8 & 230 & 3 & \textbf{6} 				\\
%\multicolumn{1}{|c|}{interproc4~\cite{jeannet2010interproc}}	&linear 			& \textbf{80} & \textbf{1} &\textbf{6}	& 120 & 2 & \textbf{6} 		\\
%\multicolumn{1}{|c|}{multivar\_1~\cite{jeannet2010interproc}}	&conjunctive		& 233 & \textbf{2} & 6 & \textbf{200} & 3  & \textbf{5} 			 \\
%
%\multicolumn{1}{|c|}{pldi08\_fig1~\cite{gulavani2008automatically}}&disjunctive		& \textbf{583} & \textbf{2} & \textbf{7}	& 777 & \textbf{2} & 13 \\
%\multicolumn{1}{|c|}{pldi08\_fig7~\cite{gulavani2008automatically}}	&linear 		&\textbf{27} & \textbf{1} &\textbf{4}	& 43 & 2 & \textbf{4} 		\\
%\multicolumn{1}{|c|}{terminator\_01~\cite{Dirk:SVCOMP:2016}}	&linear 			& 53 & \textbf{1} &\textbf{4} & \textbf{50} & 2 & 4					\\
%\multicolumn{1}{|c|}{up\_true\_2~\cite{Dirk:SVCOMP:2016}}		&conjunctive		& 920 & 33 & 31 & \textbf{600} & \textbf{8} & \textbf{20} 			 \\
%\multicolumn{1}{|c|}{xle10~\cite{sharma2012interpolants}}		&linear 			& \textbf{57} & \textbf{1} &\textbf{4}	& 60 & 2 & 4 				\\
%\multicolumn{1}{|c|}{xy0\_1~\cite{sharma2012interpolants}}		&conjunctive		& 273 & \textbf{4} & 7	& \textbf{220} & 4 & \textbf{6}				 \\
%\multicolumn{1}{|c|}{xy0\_2~\cite{sharma2012interpolants}}		&conjunctive		& \textbf{193} & \textbf{3} &\textbf{6}	& \textbf{187} & 4 & \textbf{6} \\
%\multicolumn{1}{|c|}{xy4\_1~\cite{sharma2012interpolants}}		&conjunctive		& \textbf{220} & \textbf{3} &\textbf{6}	& 313 & 5 & 8 				\\
%\multicolumn{1}{|c|}{xyle0~\cite{sharma2012interpolants}}		&polynomial 		& 433 & \textbf{4} & 91 & \textbf{267} & 5 & \textbf{79} 			\\
%\multicolumn{1}{|c|}{xyz\_2~\cite{sharma2012interpolants}}		&conjunctive		& \textbf{470} & \textbf{5} & \textbf{6} & to & to & to 			\\
%
%\multicolumn{1}{|c|}{zilu\_conj1}								&conjunctive		&\textbf{180} &\textbf{1}&\textbf{32}	& to & to &to				 \\
%\multicolumn{1}{|c|}{zilu\_disj1}								&disjunctive		& \textbf{2740} & \textbf{4} &\textbf{46} & 4160 & \textbf{4} &{65}	\\
%\multicolumn{1}{|c|}{zilu\_disj2}								&disjunctive		& \textbf{3193} & \textbf{3} & \textbf{27}	& 5887 & \textbf{3} & 30 \\
%\multicolumn{1}{|c|}{zilu\_disj3}								&disjunctive		& \textbf{7382} & \textbf{3} &\textbf{29}	& 8050 & {4}  & 40 		\\
%\multicolumn{1}{|c|}{zilu\_poly1}								&polynomial			&\textbf{57} & \textbf{2} &\textbf{37}	& 117 & 5  & 80 			 \\
%\multicolumn{1}{|c|}{zilu\_poly3}								&polynomial			& 50 & \textbf{1} &\textbf{7}		& \textbf{43} & 2  & 9			 \\
%\multicolumn{1}{|c|}{zilu\_poly6}								&polynomial			& \textbf{240} & \textbf{4} &\textbf{75} & to & to & to 			\\
%\hline
%\end{tabular}
%\label{tbl:stats}
%\end{table}

The first research question which we would like to answer is whether active learning helps to reduce the number of guess-and-check iterations as we expected. Thus, we compare \textsc{Zilu} with and without selective sampling. The experiment results are summarized in Table~\ref{summary}. The first row shows the number of programs which we can prove with \textsc{Zilu} in these two different settings, with a time limit of 10 minutes. Each experiment is executed three times since there is randomness in our approach and we report the average as the result. We claim that a program can be verified only if it is verified all three times. The results show that with help of selective sampling, \textsc{Zilu} verifies significantly more programs. Furthermore, the average time for verifying a program, the average number of guess-and-check iterations and the total number of samples are all reduced with selective sampling. In all but one experiments, \textsc{Zilu} is able to generate the invariant with fewer or equal number of guess-and-check iterations if selective sampling is applied. Though it rarely happens, due to the randomness in our approach, it may happen that the right invariant is learned by luck with few samples. This happens in the one case where \textsc{Zilu} without selective sampling has fewer iterations. It should be noted that for 7 programs, \textsc{Zilu} is unable to learn the invariant without the help of selective sampling, i.e., it timeouts due to too many guess-and-check iterations. \emph{We would like to highlight that for 14 programs, \textsc{Zilu} is able to learn the correct invariant with one guess-and-check iteration with selective sampling}, whereas this is never the case without selective sampling. Furthermore, it happens more often when the invariant is a linear predicate. We remark that being able to learn the correct invariant without program verification is useful for handling complex programs. That is, even if we are unable to automatically verify the generated invariant due to the limitation of existing program verification techniques, \textsc{Zilu}'s result is still useful in these cases as the generated invariant can be used to manually verify the program.

We present the results according to the type of loop invariants required to prove the Hoare triple in Table~\ref{tbl:stats}. We refer the readers to~\cite{zilu:repo} for the details on each program. The first column shows the name of the benchmark programs as well as where they are from. Note that we created a few programs that require polynomial or disjunctive loop invariants due to the lack of such examples in existing work. The second column shows the type of invariant required for proving the Hoare triple. The next three columns present statistics of \textsc{Zilu} with selective sampling, i.e., the number of samples generated in total, the number of guess-and-check iterations and the total execution time. The next three columns present the corresponding statistics. The winner of each measurement is highlighted using a bold font.

%We remark that though many other approaches on loop invariant generation have been reported~\cite{sharma2012interpolants,sharma2013verification,DBLP:conf/esop/0001GHALN13,sharma2014invariant}, the tools are no longer maintained and updated.

\begin{table}[t]
\scriptsize
\centering
\caption{Experiment results}
\begin{tabular}{| c | c | c | c | c | c | c | c | }
\hline
\multicolumn{1}{|c|}{benchmark}&\multicolumn{1}{|c|}{inv type} & Interproc & CPAChecker &BLAST &InvGen & Hola & \textsc{Zilu}  \\
\hline
\multicolumn{1}{|c|}{05~\cite{isil2013inductive}}				&linear 			& \cmark & to 			& \xmark  & \xmark  & \xmark   & 6\\
\multicolumn{1}{|c|}{21~\cite{isil2013inductive}}				&linear 			& \xmark & \xmark 		& \xmark  & \xmark  & \xmark   & 9\\
\multicolumn{1}{|c|}{23~\cite{isil2013inductive}}				&conjunctive		& \xmark & to 			& \xmark  & \xmark  & \xmark   & 10\\
\multicolumn{1}{|c|}{28~\cite{isil2013inductive}}				&conjunctive		& \cmark & to 			& \xmark  & \xmark  & \xmark   & 9\\
\multicolumn{1}{|c|}{30~\cite{isil2013inductive}}				&conjunctive		& \xmark & 4 			& \xmark  & \xmark  & \xmark   & 34\\
\multicolumn{1}{|c|}{35~\cite{isil2013inductive}}				&linear 			& \cmark & 2 			& \xmark  & \xmark  & \xmark   & 8\\
\multicolumn{1}{|c|}{43~\cite{isil2013inductive}}				&linear 			& \cmark & 2 			& \xmark  & \xmark  & \xmark   & 16\\
\multicolumn{1}{|c|}{bound~\cite{isil2013inductive}}			&linear 			& \cmark & 3 			& \xmark  & \xmark  & \xmark   & 6\\
\multicolumn{1}{|c|}{down~\cite{gupta2009invgen}}				&linear 			& \cmark & to 			& \xmark  & \xmark  & \xmark   & 19\\
\multicolumn{1}{|c|}{f2~\cite{zilu:repo}}						&linear 			& \cmark & \xmark 		& \xmark  & \xmark  & \xmark   & 15\\
\multicolumn{1}{|c|}{fig1b}										&disjunctive		& \xmark & 1 			& \xmark  & \xmark  & \xmark   & 32\\
\multicolumn{1}{|c|}{fm11~\cite{schwartznon}}					&conjunctive		& \cmark & 1 			& \xmark  & \xmark  & \xmark   & 34\\
\multicolumn{1}{|c|}{interproc1~\cite{jeannet2010interproc}}	&linear 			& \cmark & 2 			& \xmark  & \xmark  & \xmark   & 15\\
\multicolumn{1}{|c|}{interproc2~\cite{jeannet2010interproc}}	&linear 			& \cmark & 2 			& \xmark  & \xmark  & \xmark   & 51\\
\multicolumn{1}{|c|}{interproc3~\cite{jeannet2010interproc}}	&linear 			& \cmark & \xmark 		& \xmark  & \xmark  & \xmark   & 30\\
\multicolumn{1}{|c|}{interproc4~\cite{jeannet2010interproc}}	&linear 			& \xmark & \xmark 		& \xmark  & \xmark  & \xmark   & 22\\
\multicolumn{1}{|c|}{multivar\_1~\cite{jeannet2010interproc}}	&conjunctive		& \cmark & 2 			& \xmark  & \xmark  & \xmark   & 16\\
\multicolumn{1}{|c|}{pldi08\_fig1~\cite{gulavani2008automatically}}&disjunctive		& \xmark &  \xmark		& \xmark  & \xmark  & \xmark   & 7\\
\multicolumn{1}{|c|}{pldi08\_fig7~\cite{gulavani2008automatically}}	&linear 		& \cmark & 2 			& \xmark  & \xmark  & \xmark   & 7\\
\multicolumn{1}{|c|}{terminator\_01~\cite{Dirk:SVCOMP:2016}}	&linear 			& \cmark & 2 			& \xmark  & \xmark  & \xmark   & 7\\
\multicolumn{1}{|c|}{up\_true\_2~\cite{Dirk:SVCOMP:2016}}		&conjunctive		& \cmark & to 			& \xmark  & \xmark  & \xmark   & 60\\
\multicolumn{1}{|c|}{xle10~\cite{sharma2012interpolants}}		&linear 			& \cmark & 2 			& \xmark  & \xmark  & \xmark   & 7\\
\multicolumn{1}{|c|}{xy0\_1~\cite{sharma2012interpolants}}		&conjunctive		& \cmark & to 			& \xmark  & \xmark  & \xmark   & 23\\
\multicolumn{1}{|c|}{xy0\_2~\cite{sharma2012interpolants}}		&conjunctive		& \cmark & to 			& \xmark  & \xmark  & \xmark   & 27\\
\multicolumn{1}{|c|}{xy4\_1~\cite{sharma2012interpolants}}		&conjunctive		& \cmark & to 			& \xmark  & \xmark  & \xmark   & 18\\
\multicolumn{1}{|c|}{xyle0~\cite{sharma2012interpolants}}		&polynomial 		& \xmark & 2 			& \xmark  & \xmark  & \xmark   & 76\\
\multicolumn{1}{|c|}{xyz\_2~\cite{sharma2012interpolants}}		&conjunctive		& \cmark & \xmark		& \xmark  & \xmark  & \xmark   & 28\\
\multicolumn{1}{|c|}{zilu\_conj1}								&conjunctive		& \xmark & 2 			& \xmark  & \xmark  & \xmark   & 45\\
\multicolumn{1}{|c|}{zilu\_disj1}								&disjunctive		& \cmark & \xmark 		& \xmark  & \xmark  & \xmark   & 46\\
\multicolumn{1}{|c|}{zilu\_disj2}								&disjunctive		& \xmark & \xmark 		& \xmark  & \xmark  & \xmark   & 27\\
\multicolumn{1}{|c|}{zilu\_disj3}								&disjunctive		& \xmark & \xmark 		& \xmark  & \xmark  & \xmark   & 29\\
\multicolumn{1}{|c|}{zilu\_poly1}								&polynomial			& \xmark & 2 			& \xmark  & \xmark  & \xmark   & 42\\
\multicolumn{1}{|c|}{zilu\_poly3}								&polynomial			& \xmark & 2 			& \xmark  & \xmark  & \xmark   & 17\\
\multicolumn{1}{|c|}{zilu\_poly6}								&polynomial			& \xmark & \xmark 		& \xmark  & \xmark  & \xmark   & 74\\
\hline  & 
\end{tabular}
\label{tbl:stats2}
\end{table}


We have the following observations based on the experiment results. \textsc{Zilu} often takes more samples or guess-and-check iterations to learn conjunctive or disjunctive invariants. For conjunctive invariants, this is because the algorithm~\cite{sharma2012interpolants} for learning conjunctive classifiers often requires more samples before convergence. For disjunction invariants, this is because we need sufficient samples in each partition in order to learn the right invariant. In terms of time, \textsc{Zilu} often takes more time to learn conjunctive, polynomial or disjunctive invariants. This is because in such a case, SVM classification is invoked many times in one guess-and-check iteration. TODO: add more discussion when experiment results are updated. 

%What's more, the problem becomes much worse as the degree increases, as these undefined holes contains much more points than the defined samples, which is also the reason that we restrict the degree of our $polynomial$ algorithm up to 4.

%To illustrate the reason, consider the case of benchmark \emph{poly 1}, the loop invariant is $x^2 \le y^2$.
%However, the general form of an order-2 polynomial with 2 variables
%is: $a \cdot x^2 + b \cdot y^2 + c \cdot x y + d \cdot x + e \cdot y + f \geq 0$. It takes XXX iterations before $c$, $d$, $e$ and $f$ to converge to 0.
%%Lastly, the results show that \textsc{Zilu} complements existing tools. For instance,
%%\textsc{Zilu} can automatically generate polynomial, conjunctive or disjunctive loop invariants
%%which are often beyond the capability of Interproc and CPAChecker.
%%Interproc and CPAChecker are usually fast.
%% Interproc generates invariants within seconds. However, the invariant generated by Interproc may not be correct.
%This can be demonstrated using our running example introduced in Section~\ref{sec:introduction}.
%Since we have the loop condition $x < y$, it is impossible to execute the loop body when $(x \ge 0) \land (y < 0)$.
%However, Interproc outputs the loop invariant corresponding to this execution trace,
%because the invariants in Interproc are global constraints
%without considering their generation paths and conditions.
%In total 42 programs, CPAChecker successfully verified 17 programs,
% produced 11 false positives and 14 timeouts,

% section evaluations (end)

%\begin{table}[t]
%\scriptsize
%\centering
%\caption{Experiment results}
%\begin{tabular}{l c | c c c | c c c | c }
%\cline{3-8}
%& &\multicolumn{3}{|c|}{\textsc{Zilu} + Selective Sampling}&\multicolumn{3}{c|}{\textsc{Zilu} - Selective Sampling} & \\
%\hline
%\multicolumn{1}{|c|}{benchmark}&\multicolumn{1}{|c|}{inv type}& $\sharp$sample & $\sharp$iteration & time(s) & $\sharp$sample & $\sharp$iteration &time(s) & \multicolumn{1}{|c|}{Interproc} \\
%\hline % inserts single horizontal line
%%\multicolumn{1}{|c|}{TEMPLATE} 		        	&polynomial 	& & &  &  &  &  & &  \\
%%\hline
%\multicolumn{1}{|c|}{03}				&linear 			&- - 60 &- - 1 &\textbf{? ? 18.76}				&- - 90 &- - 2  &\textbf{? ? 22.28}					&\multicolumn{1}{|c|}{\cmark} \\
%\multicolumn{1}{|c|}{05}				&linear 			&80 140 120 &1 1 1 &\textbf{17.86 20.46 26.13}	&140 120 120 &3 3 2  &\textbf{27.24 28.18 28.46}	&\multicolumn{1}{|c|}{\cmark} \\
%\multicolumn{1}{|c|}{11}				&conjunctive		&- 360 - &- 4 -&\textbf{? 59.15 ?}				&- - - &- - -  &\textbf{? ? ?}						&\multicolumn{1}{|c|}{\cmark} \\
%\multicolumn{1}{|c|}{20}				&conjunctive		&630 600 - &6 3 - &\textbf{109.05 79.23 ?}		&390 - 390 &5 - 4   &\textbf{43.07 ? 47.28}			&\multicolumn{1}{|c|}{\cmark} \\
%\multicolumn{1}{|c|}{21}				&linear 			&220 180 160 &1 2 3 &\textbf{18.40 20.89 22.30}	&- 140 60 &- 3 1  &\textbf{? 24.72 19.60}			&\multicolumn{1}{|c|}{\cmark} \\
%\multicolumn{1}{|c|}{23}				&conjunctive		&240 210 180 &3 3 3 &\textbf{51.40 57.28 53.83}	&180 210 450 &3 3 5   &\textbf{50.74 57.28 57.70}	&\multicolumn{1}{|c|}{\cmark} \\
%\multicolumn{1}{|c|}{28}				&conjunctive		&280 280 320 &4 4 5 &\textbf{41.29 36.55 50.79}	&420 420 240 &9 10 5  &\textbf{75.63 71.95 44.49}	&\multicolumn{1}{|c|}{\cmark} \\
%\multicolumn{1}{|c|}{30}				&conjunctive		&1260 2040 680 &46 54 14 &\textbf{84.64 109.73 46.20}	&- - - &- - -  &\textbf{? ? ?}				&\multicolumn{1}{|c|}{\cmark} \\
%\multicolumn{1}{|c|}{35}				&linear 			&140 120 180 &1 1 1 &\textbf{18.76 19.54 19.41}	&340 340 - &7 7 -  &\textbf{41.83 42.55 ?}			&\multicolumn{1}{|c|}{\cmark} \\
%\multicolumn{1}{|c|}{41}				&conjunctive		&390 - - &5 - -&\textbf{75.66 ? ?}				&- - - &- - -  &\textbf{? ? ?}						&\multicolumn{1}{|c|}{\cmark} \\
%\multicolumn{1}{|c|}{43}				&linear 			&570 390 360 &2 1 1 &\textbf{25.25 29.62 28.01}	&360 240 360 &4 3 5   &\textbf{39.99 29.94 42.41}	&\multicolumn{1}{|c|}{\cmark} \\
%\multicolumn{1}{|c|}{44}				&conjunctive		&660 870 - &7 9 -&\textbf{80.63 105.12 ?}		&450 810 - &6 9 -  &\textbf{68.77 97.91 57.15}		&\multicolumn{1}{|c|}{\cmark} \\
%
%\multicolumn{1}{|c|}{afnp2014\_true}	&conjunctive		&740 - - &17 - -&\textbf{92.42 ? ?}				&1040 - - &25 - -  &\textbf{? ? ?}					&\multicolumn{1}{|c|}{\cmark} \\
%\multicolumn{1}{|c|}{bound}				&linear 			&50 30 100 &1 1 1 &\textbf{18.01 18.20 20.28}	&40 40 20 &2 2 1  &\textbf{21.96 22.95 20.16}		&\multicolumn{1}{|c|}{\cmark} \\
%\multicolumn{1}{|c|}{cggmp2005\_true}	&conjunctive		&- 870 670 &- 9 10&\textbf{? 105.87 104.10}		&- - - &- - -  &\textbf{? ? ?}						&\multicolumn{1}{|c|}{\cmark} \\
%\multicolumn{1}{|c|}{css2003\_true}		&conjunctive		&560 740 - &9 14 -&\textbf{73.88 110.94 ?}		&- 560 540 &- 12 13  &\textbf{? 95.92 93.88}		&\multicolumn{1}{|c|}{\cmark} \\
%
%\multicolumn{1}{|c|}{dillig\_01}		&conjunctive		&- - - &- - -&\textbf{? ? 17.35}				&- - - &- - -  &\textbf{? ? 28.30}					&\multicolumn{1}{|c|}{\cmark} \\
%\multicolumn{1}{|c|}{dillig\_03}		&conjunctive		&320 - - &7 - -&\textbf{87.90 ? ?}				&- - - &- - -  &\textbf{? ? 92.79}					&\multicolumn{1}{|c|}{\cmark} \\
%\multicolumn{1}{|c|}{dillig\_05}		&conjunctive		&920 - 880 &8 - 7 &\textbf{22.97 ? 16.76}		&1450 600 1160 &21 6 11  &\textbf{37.71 11.14 22.83}&\multicolumn{1}{|c|}{\cmark} \\
%\multicolumn{1}{|c|}{dillig\_07}		&conjunctive		&480 540 - &6 6 -&\textbf{61.55 62.95 ?}		&870 360 510 &13 5 7  &\textbf{? 44.16 88.72}		&\multicolumn{1}{|c|}{\cmark} \\
%\multicolumn{1}{|c|}{dillig\_15}		&conjunctive		&360 - - &5 - -&\textbf{81.07 ? ?}				&360 - - &5 - -  &\textbf{76.68? ?}					&\multicolumn{1}{|c|}{\cmark} \\
%\multicolumn{1}{|c|}{dillig\_19}		&linear 			&210 - 510 &2 - 6 &\textbf{33.32 ? 118.95}		&390 - 390 &4 - 5   &\textbf{97.98 ? 50.50}			&\multicolumn{1}{|c|}{\cmark} \\
%\multicolumn{1}{|c|}{dillig\_20}		&conjunctive		&- - 570 &- - 6 &\textbf{? ? 70.03}				&300 720 210 &4 10 6   &\textbf{38.16 67.92 44.84}	&\multicolumn{1}{|c|}{\cmark} \\
%\multicolumn{1}{|c|}{dillig\_28}		&conjunctive		&- - - &- - -&\textbf{? ? ?}					&- 390 - &- 5 -  &\textbf{? 91.28 ?}				&\multicolumn{1}{|c|}{\cmark} \\
%\multicolumn{1}{|c|}{dillig\_35}		&linear 			&140 180 140 &1 1 5 &\textbf{19.81 18.52 22.96}	&280 140 160 &6 2 4  &\textbf{38.03 23.75 37.57}	&\multicolumn{1}{|c|}{\cmark} \\
%
%\multicolumn{1}{|c|}{down}				&linear 			&480 300 330 &4 3 3 &\textbf{38.02 27.12 37.79}	&330 240 390 &5 4 6  &\textbf{37.93 32.79 50.39}	&\multicolumn{1}{|c|}{\cmark} \\
%\multicolumn{1}{|c|}{down\_true\_1}		&conjunctive		&- - - &- - -&\textbf{? ? ?}					&- - - &- - -  &\textbf{113.20 79.00 102.75}		&\multicolumn{1}{|c|}{\cmark} \\
%\multicolumn{1}{|c|}{down\_true\_2}		&conjunctive		&690 780 - &8 8 - &\textbf{110.06 91.31 ?}		&540 - 600 &6 - 8  &\textbf{103.49 ? 100.73}			&\multicolumn{1}{|c|}{\cmark} \\
%\multicolumn{1}{|c|}{ex49}				&conjunctive		&- 780 - &- 11 -&\textbf{? 80.89 ?}				&- - - &- - -  &\textbf{? ? ?}						&\multicolumn{1}{|c|}{\cmark} \\
%
%\multicolumn{1}{|c|}{f2}				&linear 			&100 140 140 &1 1 1 &\textbf{16.67 46.83 24.08}	&120 100 140 &3 2 1   &\textbf{25.09 21.79 69.14}	&\multicolumn{1}{|c|}{\cmark} \\
%\multicolumn{1}{|c|}{f3}				&linear 			&150 - 90 &1 - 1 &\textbf{20.60 ? 50.42}		&210 - - &3 - -  &\textbf{31.13 ? ?}				&\multicolumn{1}{|c|}{\cmark} \\
%\multicolumn{1}{|c|}{fm11}				&conjunctive		&180 160 160 &2 2 2&\textbf{15.02 9.13 10.56}	&280 440 - &6 9 -  &\textbf{22.64 80.91 ?}				&\multicolumn{1}{|c|}{\cmark} \\
%\multicolumn{1}{|c|}{half}				&conjunctive		&- - - &- - -&\textbf{? ? ?}					&- - - &- - -  &\textbf{? ? ?}						&\multicolumn{1}{|c|}{\cmark} \\
%
%\multicolumn{1}{|c|}{interproc1}		&linear 			&50 20 40 &1 1 1 &\textbf{17.54 17.64 19.56}	&40 40 50 &2 2 2  &\textbf{19.20 20.59 22.73}		&\multicolumn{1}{|c|}{\cmark} \\
%\multicolumn{1}{|c|}{interproc2}		&linear 			&100 100 120 &1 1 1 &\textbf{19.66 18.88 20.45}	&- 60 - &- 1 -  &\textbf{? 30.62 ?}					&\multicolumn{1}{|c|}{\cmark} \\
%\multicolumn{1}{|c|}{interproc3}		&linear 			&180 180 210 &1 1 1&\textbf{28.88 92.20 23.13}	&270 300 210 &2 5 3  &\textbf{47.03 41.38 34.89}	&\multicolumn{1}{|c|}{\cmark} \\
%\multicolumn{1}{|c|}{interproc4}		&linear 			&140 140 120 &1 1 1 &\textbf{38.31 39.18 48.12}	&120 120 160 &3 3 3  &\textbf{66.12 47.59 80.36}	&\multicolumn{1}{|c|}{\cmark} \\
%\multicolumn{1}{|c|}{interproc5}		&conjunctive		&- 240 840 &- 3 19 &\textbf{? 27.18 116.72}		&- 460 560 &- 10 14  &\textbf{? 59.31 93.43}		&\multicolumn{1}{|c|}{\cmark} \\
%
%%\multicolumn{1}{|c|}{large\_const}		&conjunctive		&- - - &- - -&\textbf{? ? ?}					&- - - &- - -  &\textbf{? ? ?}						&\multicolumn{1}{|c|}{\cmark} \\
%\multicolumn{1}{|c|}{multivar\_1}		&conjunctive		&260 220 - &2 3 -&\textbf{26.40 33.72 ?}		&340 260 240 &4 5 5  &\textbf{36.88 32.52 43.74}	&\multicolumn{1}{|c|}{\cmark} \\
%% \multicolumn{1}{|c|}{new\_copy}			&conjunctive		&- - - &- - -&\textbf{x x x}			&- - - &- - -  &\textbf{x x x}		&\multicolumn{1}{|c|}{\cmark} \\
%\multicolumn{1}{|c|}{pldi08\_fig7}		&linear 			&20 40 40 &1 1 1 &\textbf{18.29 19.69 15.76}	&20 40 40 &1 2 1   &\textbf{18.91 24.97 20.78}		&\multicolumn{1}{|c|}{\cmark} \\
%%\multicolumn{1}{|c|}{simple\_if}		&conjunctive		&- - - &- - -&\textbf{? ? ?}					&- - - &- - -  &\textbf{? ? ?}						&\multicolumn{1}{|c|}{\cmark} \\
%%\multicolumn{1}{|c|}{substring1}		&conjunctive		&- - - &- - -&\textbf{? ? ?}					&- - - &- - -  &\textbf{? ? ?}						&\multicolumn{1}{|c|}{\cmark} \\
%\multicolumn{1}{|c|}{terminator\_01}	&linear 			&30 40 40 &1 1 1 &\textbf{17.89 17.30 22.72}	&40 40 40 &2 2 2  &\textbf{20.95 20.51 25.85}		&\multicolumn{1}{|c|}{\cmark} \\
%\multicolumn{1}{|c|}{up\_true\_2}		&conjunctive		&420 410 630 &5 4 6 &\textbf{80.03 87.60 106.09}&630 510 450 &8 7 5  &\textbf{88.50 70.64 80.58}	&\multicolumn{1}{|c|}{\cmark} \\
%
%\multicolumn{1}{|c|}{xle10}				&linear 			&90 40 70 &2 1 2 &\textbf{21.07 17.53 26.57}	&60 60 20 &3 3 2  &\textbf{26.04 25.37 25.48}		&\multicolumn{1}{|c|}{\cmark} \\
%\multicolumn{1}{|c|}{xy0\_1}			&conjunctive		&520 500 580 &8 9 12 &\textbf{57.82 54.40 100.64}	&640 740 200 &15 16 4   &\textbf{98.86 83.74 37.38}	&\multicolumn{1}{|c|}{\cmark} \\
%\multicolumn{1}{|c|}{xy0\_2}			&conjunctive		&160 220 180 &3 3 3 &\textbf{31.09 34.40 30.66}	&180 200 180 &3 4 4  &\textbf{31.22 36.19 30.97}		&\multicolumn{1}{|c|}{\cmark} \\
%\multicolumn{1}{|c|}{xy4\_1}			&conjunctive		&320 540 280 &3 11 4 &\textbf{39.66 100.12 55.13}	&720 500 300 &17 11 6  &\textbf{103.03 93.11 65.28}	&\multicolumn{1}{|c|}{\cmark} \\
%\multicolumn{1}{|c|}{xy4\_2}			&conjunctive		&- 240 - &- 3 -&\textbf{? 47.68 ?}				&- - 220 &- - 4  &\textbf{? ? 38.01}				&\multicolumn{1}{|c|}{\cmark} \\
%\multicolumn{1}{|c|}{xy10}				&linear 			&- 450 3 &- 6 6 &\textbf{? 81.50 73.13}			&210 450 3 &3 7 6   &\textbf{27.01 111.32 34.35}	&\multicolumn{1}{|c|}{\cmark} \\
%\multicolumn{1}{|c|}{xyle0}				&polynomial 		&300 940 260 &5 2 5&\textbf{214.72 52.87 140.46}&260 - 300 &5 - 5  &\textbf{172.39 ? 181.34}		&\multicolumn{1}{|c|}{\cmark} \\
%\multicolumn{1}{|c|}{xyz\_2}			&conjunctive		&2910 600 240 &42 7 4&\textbf{67.18 12.60 7.83}	&- - 2710 &- - 41  &\textbf{? ? 67/08}				&\multicolumn{1}{|c|}{\cmark} \\
%% \multicolumn{1}{|c|}{xyz2\_1}			&conjunctive		&- - - &- - -&\textbf{? ? ?}					&- - - &- - -  &\textbf{? ? ?}						&\multicolumn{1}{|c|}{\cmark} \\
%% \multicolumn{1}{|c|}{xyz2\_2}			&conjunctive		&- - - &- - -&\textbf{? ? ?}					&- - - &- - -  &\textbf{? ? ?}						&\multicolumn{1}{|c|}{\cmark} \\
%	
%\multicolumn{1}{|c|}{zilu\_conj1}		&conjunctive		&220 180 140 &1 1 1 &\textbf{15.03 65.45 16.39}	&220 300 -& 2 2 - &\textbf{99.32 97.38 -}			&\multicolumn{1}{|c|}{\cmark} \\
%\multicolumn{1}{|c|}{zilu\_conj2}		&conjunctive		&- - 160 &- - 3 &\textbf{? ? 74.77}				&- - 80 &- - 2  &\textbf{? ? ?}						&\multicolumn{1}{|c|}{\cmark} \\
%\multicolumn{1}{|c|}{zilu\_linear1}		&linear				&120 80 200 &2 1 3 &\textbf{24.98 19.77 31.74}	&180 220 140 &3 3 3  &\textbf{96.34 36.61 60.09}	&\multicolumn{1}{|c|}{\cmark} \\
%\multicolumn{1}{|c|}{zilu\_linear2}		&linear				&390 150 300 &4 2 3 &\textbf{40.93 22.71 28.43}	&330 210 180 &4 3 3   &\textbf{31.57 25.47 30.38}	&\multicolumn{1}{|c|}{\cmark} \\
%\multicolumn{1}{|c|}{zilu\_linear3}		&linear				&- 150 - &- 1 -&\textbf{? 65.17 ?}				&- - 210 &- - 3  &\textbf{? ? 29.05}				&\multicolumn{1}{|c|}{\cmark} \\
%\multicolumn{1}{|c|}{zilu\_linear4}		&linear				&320 160 220 &1 1 1 &\textbf{18.13 19.42 20.26}	&80 300 300 &1 3 3  &\textbf{17.60 82.76 49.72}		&\multicolumn{1}{|c|}{\cmark} \\
%\multicolumn{1}{|c|}{zilu\_poly1}		&polynomial			&50 50 70 &2 2 3 &\textbf{30.34 29.53 50.03}	&120 100 130 &5 4 6  &\textbf{93.37 65.88 ?}		&\multicolumn{1}{|c|}{\cmark} \\
%\multicolumn{1}{|c|}{zilu\_poly2}		&polynomial			&110 - - &1 - -&\textbf{24.27 53.67 44.06}			&- - - &- - -  &\textbf{? 59.00 ?}				&\multicolumn{1}{|c|}{\cmark} \\
%\multicolumn{1}{|c|}{zilu\_poly3}		&polynomial			&40 50 60 &1 1 1 &\textbf{9.41 5.94 5.86}		&40 50 40 &2 2 2  &\textbf{11.86 9.13 8.42}			&\multicolumn{1}{|c|}{\cmark} \\
%\multicolumn{1}{|c|}{zilu\_poly4}		&polynomial			&- - - &- - -&\textbf{? ? ?}					&- - - &- - -  &\textbf{? ? ?}						&\multicolumn{1}{|c|}{\cmark} \\
%\multicolumn{1}{|c|}{zilu\_poly5}		&polynomial			&- - - &- - -&\textbf{? ? ?}					&- - - &- - -  &\textbf{? ? ?}						&\multicolumn{1}{|c|}{\cmark} \\
%\multicolumn{1}{|c|}{zilu\_poly6}		&polynomial			&160 300 200 &1 6 2&\textbf{10.82 183.61 40.08}	&420 - - &10 - -  &\textbf{472.29 ? ?}				&\multicolumn{1}{|c|}{\cmark} \\
%
%
%
%
%
%
%% \multicolumn{1}{|c|}{f2~\cite{DBLP:conf/pldi/GulwaniSV08}}      	&linear 		&260 &1 &\textbf{10.15}  		&120 &1   &12.03  			&\multicolumn{1}{|c|}{\cmark} \\
%% \multicolumn{1}{|c|}{xy10~\cite{sharma2012interpolants}} 	    	&linear 		&810 &4 &\textbf{39.69} 		&840  &11  &40.51  			&\multicolumn{1}{|c|}{\cmark} \\
%% \multicolumn{1}{|c|}{xyz~\cite{sharma2012interpolants}}   	   		&linear 		&180 &1 &105.15  				&330  &5  &\textbf{93.04}  	&\multicolumn{1}{|c|}{\cmark} \\
%% \multicolumn{1}{|c|}{xy0\_1~\cite{sharma2012interpolants}}      	&conjunctive 	&1140 &20 &\textbf{51.07}		&1980 &48 &168.97  			&\multicolumn{1}{|c|}{\cmark} \\
%% \multicolumn{1}{|c|}{xy0\_2~\cite{sharma2012interpolants}}      	&conjunctive 	&180  &2 &16.09					&260 &6 &\textbf{15.98}  	&\multicolumn{1}{|c|}{\cmark} \\
%% \multicolumn{1}{|c|}{pldi08~\cite{gulavani2008automatically}}		&disjunctive 	&- & - &timeout  				&-  &-  &timeout  		&\multicolumn{1}{|c|}{\xmark} \\
%% %\hline
%% \multicolumn{1}{|c|}{interproc1~\cite{jeannet2010interproc}}     	&linear 		&40 &1 &\textbf{9.21}  			&40 &2   &10.38  			&\multicolumn{1}{|c|}{\cmark} \\
%% \multicolumn{1}{|c|}{interproc2~\cite{jeannet2010interproc}}     	&linear 		&600 &1 &\textbf{11.66} 		&240  &1  &171.14  			&\multicolumn{1}{|c|}{\cmark} \\
%% \multicolumn{1}{|c|}{interproc3~\cite{jeannet2010interproc}}     	&linear 		&210 &1 &\textbf{30.32}  		&420 &5   &43.34  			&\multicolumn{1}{|c|}{\cmark} \\
%% \multicolumn{1}{|c|}{interproc4~\cite{jeannet2010interproc}}     	&linear 		&140 &4 &\textbf{4.8}  			&240 &5   &38.25  			&\multicolumn{1}{|c|}{\xmark} \\
%% \multicolumn{1}{|c|}{interproc5~\cite{jeannet2010interproc}}     	&linear 		&160 &1 &\textbf{9.18}  		&180 &3   &28.05  			&\multicolumn{1}{|c|}{\cmark} \\
%
%% %\hline
%% \multicolumn{1}{|c|}{zilu\_linear1}         			&linear 		&120 &2 &\textbf{24.79}  		&180 &2   &206.07  			&\multicolumn{1}{|c|}{\xmark} \\
%% \multicolumn{1}{|c|}{zilu\_linear2}         			&linear 		&420 &4 &\textbf{21.28}  		&720  &9  &82.24  			&\multicolumn{1}{|c|}{\cmark} \\
%% \multicolumn{1}{|c|}{zilu\_linear3}         			&linear 		&90 &1 &\textbf{16.19}  		&270 &3  &179.39  			&\multicolumn{1}{|c|}{\cmark} \\
%% \multicolumn{1}{|c|}{zilu\_linear4}         			&linear 		&140 &1 &\textbf{10.19}  		&420 &2  &24.51  			&\multicolumn{1}{|c|}{\cmark} \\
%% \multicolumn{1}{|c|}{zilu\_linear5}         			&linear 		&30 &1 &\textbf{8.57}  			&60 &3   &12.58  			&\multicolumn{1}{|c|}{\cmark} \\
%
%% \multicolumn{1}{|c|}{zilu\_poly1}         				&polynomial 	&50 &2 &\textbf{20.48}			&70 &1 &28.48  				&\multicolumn{1}{|c|}{\xmark} \\
%% \multicolumn{1}{|c|}{zilu\_poly2}         				&polynomial 	&110  &1 &\textbf{24.27}  		&70   &2 &41.62  			&\multicolumn{1}{|c|}{\xmark} \\
%% \multicolumn{1}{|c|}{zilu\_poly3}         				&polynomial 	&30 &1 &\textbf{11.83}  		&70  &2  &18.27  			&\multicolumn{1}{|c|}{\xmark} \\
%% \multicolumn{1}{|c|}{zilu\_poly4}         				&polynomial 	&- &-&timeout  					&-  &-  &timeout 			&\multicolumn{1}{|c|}{\xmark} \\
%% \multicolumn{1}{|c|}{zilu\_poly5}         				&polynomial 	&- &- &timeout  				&-  &-  &timeout  			&\multicolumn{1}{|c|}{\xmark} \\
%% \multicolumn{1}{|c|}{zilu\_poly6}         				&polynomial 	&300 &4 &\textbf{85.61}  		&620   &12 &472.68  		&\multicolumn{1}{|c|}{\xmark} \\
%
%% \multicolumn{1}{|c|}{zilu\_conj1}         				&conjunctive 	&120 &6 &\textbf{22.13}  		&220  &2  &97.47  			&\multicolumn{1}{|c|}{\xmark} \\
%% \multicolumn{1}{|c|}{zilu\_conj2}         				&conjunctive 	&280 &3 &\textbf{173.09}  		&-  &-  &timeout  			&\multicolumn{1}{|c|}{\cmark} \\
%
%% \multicolumn{1}{|c|}{terminator\_01\_safe~\cite{beyer:SVCOMP:2013}}         	&linear 		&30 &1 &\textbf{9.2}  				&90  &4  &13.06  			&\multicolumn{1}{|c|}{\cmark} \\
%% \multicolumn{1}{|c|}{afnp2014\_true~\cite{Dirk:SVCOMP:2016}}         			&conjunctive	&1240 &27 &\textbf{39.33}			&- &- &timeout  		&\multicolumn{1}{|c|}{\xmark} \\
%% \multicolumn{1}{|c|}{multivar\_true\_1~\cite{Dirk:SVCOMP:2016}}         		&conjunctive 	&340 &3 &16.84  					&220 &4   &\textbf{15.22}  	&\multicolumn{1}{|c|}{\cmark} \\
%% \multicolumn{1}{|c|}{cggmp2005\_variant~\cite{Dirk:SVCOMP:2016}}   				&conjunctive 	&1020 &13 &\textbf{108.57}			&- &- &timeout  		&\multicolumn{1}{|c|}{\cmark} \\
%% \multicolumn{1}{|c|}{css2003\_true~\cite{Dirk:SVCOMP:2016}}         			&conjunctive 	&1420 &33 &\textbf{57.78}			&5080 &125 &258.65  		&\multicolumn{1}{|c|}{\cmark} \\
%% \multicolumn{1}{|c|}{up\_true\_2~\cite{Dirk:SVCOMP:2016}}         				&conjunctive 	&1200 &14 &\textbf{84.02}  			&540 &7   &89.77  			&\multicolumn{1}{|c|}{\cmark} \\
%% \multicolumn{1}{|c|}{down\_true\_1~\cite{Dirk:SVCOMP:2016}}         			&conjunctive 	&1320  &15 &\textbf{116.21}  		&-  &-  &timeout  		&\multicolumn{1}{|c|}{\cmark} \\
%% \multicolumn{1}{|c|}{down\_true\_2~\cite{Dirk:SVCOMP:2016}}         			&conjunctive 	&1110 &14 &52.96  					&720 &10   &\textbf{44.99}  &\multicolumn{1}{|c|}{\cmark} \\
%
%\hline
%\end{tabular}
%\label{tbl:stats}
%\end{table}

The second research question which we would like to answer is whether \textsc{Zilu} can outperform existing program verification tools on verifying these programs. Ideally, we would like to compare with those tools reported in~\cite{sharma2012interpolants,sharma2013verification,DBLP:conf/esop/0001GHALN13,sharma2014invariant}. Unfortunately, those are not maintained. We instead compare \textsc{Zilu} with five state-of-the-art tools on loop invariant generation and program verification. In particular, Interproc is a program verifier which generates invariants based on abstract interpretation. In the experiments, it is set to use its most expressive abstract domain, i.e., the reduced product of polyhedra and linear congruences abstraction. CPAChecker~\cite{DBLP:conf/cav/BeyerK11} is a state-of-the-art program verifier. The binary we use is the version used for SV-COMP 2016~\cite{Dirk:SVCOMP:2016}. Note that CPAChecker supports a variety of verification methods and it is configured in the exact same way in SV-COMP 2016. TODO: briefly introduce BLAST and InvGen and Hola. 
The results are shown in Table~\ref{tbl:stats2}, where the winner for each program is highlighted in bold. We remark that the comparison however should be taken with a grain of salt as the methods are different.

For all of these programs, \textsc{Zilu} is able to find a loop invariant which proves the Hoare triple. In comparison, Interproc failed in 13 cases and CPAChecker failed in 18 cases (with 8 timeouts and 10 false positives). Secondly, \textsc{Zilu} is relatively efficient. For all these programs, \textsc{Zilu} finishes the proof within 75 seconds. A close look reveals that most of the time is spent on classification and selective sampling. It should be noted though that when CPAChecker is able to prove the Hoare triple, it is usually very efficient.
