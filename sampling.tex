%!TEX root = paper.tex

\section{Sampling}
In this step, we sample, either randomly or using tools based on the idea of concolic testing~\cite{}, a set $T$ of program states and test the program starting with each program state $s$ in $T$. We write $Body^*(s)$ to denote the set of program states which could be reached after executing zero or more iterations of the loop starting from $s$. We write $Body^*(T)$ to denote $\{s' | \exists s \in T \cdot s' \in Body^*(s)\}$. Furthermore, we write $s \Rightarrow s'$ to denote that starting with a program state $s$ would result in state $s'$ when the loop terminates. We categorize program states in $Body^*(T)$ into four sets:
\begin{itemize}
    \item Set $CT_T$ is $\{s \in Body^*(T) | s \in Pre \land s \Rightarrow s' \land s' \notin Post\}$;
    \item Set $P_T$ is $\{s \in Body^*(T) | s \in Pre \land s \Rightarrow s' \land s' \in Post\}$;
    \item Set $N_T$ is $\{s \in Body^*(T) | s \notin Pre \land s \Rightarrow s' \land s' \notin Post\}$;
    \item Set $NP_T$ is $\{s \in Body^*(T) | s \notin Pre \land s \Rightarrow s' \land s' \in Post\}$;
\end{itemize}
We remark that anytime a program state in $CT_T$ is identified, a counterexample is found 
and \textsc{Zilu} reports that verification is failed. 
Otherwise, because $Inv$ must satisfy (1),(2) and (3), we know that $P_T \subseteq Inv$ 
and $N_T \land Inv = \emptyset$. 
The program states in $NP_T$ may or not may be in $Inv$. 
If we know that a program state $s \in NP_T$ is in $Inv$, $Body^*(s) \subseteq Inv$.

\begin{example}
\end{example}