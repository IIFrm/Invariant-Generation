%!TEX root = paper.tex

\section{Labeling}
In this step, we sample, either randomly or using tools based on the idea of concolic testing~\cite{}, 
a set $T$ of program states and test the program starting with each program state $s$ in $T$. 
We write $Body^*(s)$ to denote the set of program states which could be reached after executing zero or more iterations of the loop starting from $s$. 
We write $Body^*(T)$ to denote $\{s' | \exists s \in T \cdot s' \in Body^*(s)\}$. 
Furthermore, we write $s \Rightarrow s'$ to denote that starting with a program state $s$ would result in state $s'$ when the loop terminates. 

We categorize program states in $Body^*(T)$ into four sets:
$C_T$ which stands for counter-example trace, 
$P_T$ which stands for traces with positive labels, 
$N_T$ which stands for traces with negative labels 
and $U_T$ which stands for traces with unknown labels.
They can be judged by the following rules: 
\begin{itemize}
    \item Set $C_T$ is $\{s \in Body^*(T) | s \in Pre \land s \Rightarrow s' \land s' \nin Post\}$;
    \item Set $P_T$ is $\{s \in Body^*(T) | s \in Pre \land s \Rightarrow s' \land s' \in Post\}$;
    \item Set $N_T$ is $\{s \in Body^*(T) | s \nin Pre \land s \Rightarrow s' \land s' \nin Post\}$;
    \item Set $U_T$ is $\{s \in Body^*(T) | s \nin Pre \land s \Rightarrow s' \land s' \in Post\}$;
\end{itemize}
We remark that anytime a program state in $C_T$ is identified, a counter-example is found and \textsc{Zilu} reports that verification is failed immediately. 
Otherwise, because $Inv$ must satisfy (1),(2) and (3), we know that $P_T \subseteq Inv$ and $N_T \inter Inv = \emptyset$. 
The program states in $U_T$ may or not may be in $Inv$. 
If we know that a program state $s \in U_T$ is in $Inv$, $Body^*(s) \subseteq Inv$.

\begin{example}
\end{example}