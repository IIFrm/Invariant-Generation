%!TEX root = paper.tex

\section{Sampling} % (fold)
\label{sec:sampling}
%Before learning procedure, we need to gather some data from program. 
%So 
In a broader view, \textsc{Zilu} is a dynamic program verification technique, 
thus we are quite interested in program states during its execution.
Sampling is a way to tell our preference on program states. 
So in this step, we would like to gather more information through sampling.
%, either randomly or using tools based on the idea of concolic testing~\cite{}, 
Actually, in our setting, we do sample a set $S$ of program states in three different approaches: 
random sampling, selective sampling and counter-example sampling, 
which are applied according to the information we gathered by that time.
If compared sampling to asking question, random sampling is like asking random questions,
and selective sampling is like asking tentative questions, 
while counter-example sampling is like an eye-to-eye debate.
The following paragraphs shows how these three sampling technique works and why we apply them in $\textsc{Zilu}$.

%In this step, we sample, either randomly or using tools based on the idea of concolic testing~\cite{}, 
%a set $T$ of program states and test the program starting with each program state $s$ in $T$. 
%We write $Body^*(s)$ to denote the set of program states which could be reached after executing zero or more iterations of the loop starting from $s$. 
%We write $Body^*(T)$ to denote $\{s' | \exists s \in T \cdot s' \in Body^*(s)\}$. 
%Furthermore, we write $s \Rightarrow s'$ to denote that starting with a program state $s$ would result in state $s'$ when the loop terminates. 

%We categorize program states in $Body^*(T)$ into four sets:
%$C_T$ which stands for counter-example trace, 
%$P_T$ which stands for traces with positive labels, 
%$N_T$ which stands for traces with negative labels 
%and $U_T$ which stands for traces with unknown labels.
%They can be judged by the following rules: 
%\begin{itemize}
%    \item Set $C_T$ is $\{s \in Body^*(T) | s \in Pre \land s \Rightarrow s' \land s' \notin Post\}$;
%    \item Set $P_T$ is $\{s \in Body^*(T) | s \in Pre \land s \Rightarrow s' \land s' \in Post\}$;
%    \item Set $N_T$ is $\{s \in Body^*(T) | s \notin Pre \land s \Rightarrow s' \land s' \notin Post\}$;
%    \item Set $U_T$ is $\{s \in Body^*(T) | s \notin Pre \land s \Rightarrow s' \land s' \in Post\}$;
%\end{itemize}
%We remark that anytime a program state in $C_T$ is identified, a counter-example is found and \textsc{Zilu} reports that verification is failed immediately. 
%Otherwise, because $Inv$ must satisfy (1),(2) and (3), we know that $P_T \subseteq Inv$ and $N_T \cap Inv = \emptyset$. 
%The program states in $U_T$ may or not may be in $Inv$. 
%If we know that a program state $s \in U_T$ is in $Inv$, $Body^*(s) \subseteq Inv$.

\subsection{Random Sampling}
Random sampling is well known in statistics, including simple random sampling and systematic random sampling.
In this context, we refer to the former meaning, in which each sample is chosen randomly and entirely by chance, 
such that each sample has the same probability of being chosen at any stage during the sampling process.
In our setting, we generate samples by calling random functions supported by most programming language.

It is a quite intuitive but important sampling approach, especially when we do not know the system at the beginning. 
It is proved that, simple random sampling is most effecient when no information is given.
$\textsc{Zilu}$ applies this technique all along the inference procedure, not only the beginning. 
That is because our assumption is there always much more data points than we could sample in a given time,
otherwise we can verify the program by enumeration.
Although our sampling technique is most effecient from statistical persperctive, 
but there is no guarantee this sampling works every time.
So in the subsequent iterations, we need to add little randomness to avoid, or at least reduce, 
the impact of the bias resulted from previous sampling results.


\subsection{Selective Sampling}
Selective sampling is a technique in active learning, which can generate samples which are more informative and instructive.
It is an aggressive sampling technique that help $\textsc{Zilu}$ to learn the target much faster.

%When we have a guess of loop invariants, we can apply selective sampling approach to find more useful samples.
%Actually we apply this sampling method all along the learning procedure except the first iteration.
It is obvious that, due to the limited set of samples we have (which is often referred to as labeled samples in the machine learning community), 
the guessed classifier obtained from previous iteration might be far from being correct. 
In fact, without labeled samples which are right on the boundary of the `actual' classifier, 
it is very unlikely that we would find it. 
Intuitively and intelligently, in order to get the `actual' classifier, 
we would require samples which would distinguish the actual one from any nearby one, 
This problem has been discussed and addressed in machine learning using active learning and selective sampling~\cite{DBLP:conf/icml/SchohnC00}.

The concept of active learning or selective sampling refers to the approaches 
that aim at reducing the labeling effort by selecting only the most informative samples to be labeled. 
SVM(supported vector machines) selective sampling techniques have been proven effective in achieving a high accuracy 
with fewer examples in many applications~\cite{DBLP:conf/mm/TongC01,DBLP:journals/jmlr/TongK01}. 
The basic idea of selective sampling is that at each round, we select the samples that are the closest to the classification boundary 
so that they are the most difficult to classify and the most informative to label. 
Since an SVM classification function is represented by support vectors which are the samples closest to the boundary, 
this selective sampling effectively learns an accurate function with fewer labeled data~\cite{DBLP:conf/icml/SchohnC00}. 
In our setting, this means that we should sample a program state right by the classifier and test the program 
with that state to label that feature vector so that the classifier would be improved.


%Algorithm~\ref{alg:active} presents details on how active learning is implemented in \textsc{Zilu}. 
%At line 2, we obtain a classifier based on Algorithm~\ref{classify}. 
%We compare the newly obtained classifier with the previous one at line 4, if they are identical, we return the classifier; 
%otherwise we apply selective sampling so that we can generate additional labeled samples for improving the classifier. 
%In particular, at line 5, we apply standard techniques~\cite{DBLP:conf/icml/SchohnC00} to select the most informative sample. 
%Notice that in our setting, the most informative samples are those which are exactly on the lines and 
%therefore can be obtained by solving an equation system. 
%At line 8, we test the program with the newly generated samples so as to label them accordingly.
After the above discussion, apparently the prerequirement of selective sampling is there is a guess.
Thus in our implementation, after learning an invariant candiate, 
which is usually a single polynomials or conjunction or disjunction of polynomials,
we do sampling along this or these polynomials by using \textbf{GSL} (GNU Scientific Library) to solve them.

This technique is applied all along the learning process except the very begining.



\subsection{Counter-Example Sampling}
Compared with the above sampling techniques, we should admit counter-example sampling is more directly and objective.  
Having an invariant candidate, $\textsc{Zilu}$ tries to valid it using concolic testing~\cite{} and constraint solving,
which is shown in detail in section verification.
If it fails to validate, the constraint solver could provide us with counter-examples that can directly refute our invariant candidate.
And as a result, it is quite useful for the invariant candidate refinement in the next learning procedure.

As counter-example sampling technique sounds good, it seems this technique should be applied almost all the time, 
but the fact is it is applied only after failure of invariant candidate verification.
That is because appling concolic tesint and constraint solving is a more time consuming job than the other two sampling methods.

\subsection {Labeling}
With the sample set $S$ from the last step, we test the program starting with each program state $s$ in $S$. 
As said in the introduction part, $Body(s)$ is the state which could be reached after executing $Body$ from state s.
We write $Body^*(s)$ to denote the set of program states which could be reached after executing zero or more iterations of the loop starting from $s$.

So if there is a trace $Trace\{ s_0, s_1, \ldots, s_n\}$, then 
\begin{itemize}
\item $s_{i+1} \in Body(s_i)\ \forall i \in [0, \ldots, n-1]$.
\item $s_{i} \in Body^*(s_0)\ \forall i \in [0, \ldots, n]$.
\end{itemize}
%We write $Body^*(S)$ to denote $\{s' | \exists s \in S \cdot s' \in Body^*(s)\}$. 
Furthermore, we write $s \Rightarrow\Rightarrow s'$ to denote that starting with a program state $s$ would result in state $s'$ when the loop terminates. 


\subsubsection*{Positive State, Negative State \& Implication State}
These three concepts are introduced in \cite{sharma2014invariant}.\
In this paper, we use predicates and sets of states interchangeably.
Let $C$ be a candidate invariant.

From equation (1) we know, for an invariant $Inv$, 
any state that satisfies $Pre$ also satisfies $Inv$. 
We call any state that must be satisfied by an actual invariant a positive state. 


Now consider equation (2).
A pair $(s, t)$ satisfies the property that $s$ satisfies $Cond$ and if the execution of $S$
is started in state $s$ then $S$ can terminate in state $t$. 
Since an actual invariant $Inv$ is inductive, it should satisfy $s \in Inv \Rightarrow t \in {Inv}$. 
Hence, a pair $(s, t)$ satisfying $s \in C \land t \notin C$ proves $C$ is not an invariant. 

Finally, consider equation (3).
The `existence of a state $s \in C \wedge \neg B \wedge \neg Post$ proves $C$ is inadequate to discharge the postcondition. 
We call a state $s$ which satisfies $\neg{B} \land \neg{Post}$ a negative state. 



\subsubsection*{Positive Trace, Negative Trace \& Implication Trace}
In our approach, we assume $\{s_0, s_1, s_2, ..., s_i, ... , s_n\}$ is a trace in the target program,
where $s_0$ is the initial state before entering the loop, 
and $s_i$ is a state just after the loop has iterated $i$ times in the program.
We assume $s_n$ satisfies $\neg B$ so it is the state that can jump out the loop body.



For a trace $\{s_0, s_1, s_2, ..., s_i, ... , s_n\}$, 
if $s_0$ satisfies $Pre$, and $s_n$ satisfies $Post$,
we say this is a positive trace,
Because if state $s_0$ satisfy $Pre$,
$s_0$ is a positive state that must satisfy $Inv$, according to equation 1.
Furthermore, according to equation 2,
all of $\{s_1, s_2, ..., s_i, ... , s_n\}$ are positive states.
So now we can get a positive trace  $\{s_0, s_1, s_2, ..., s_i, ... , s_n\}$.


On the contrary, for a trace $\{s_0, s_1, s_2, ..., s_i, ... , s_n\}$, 
if $s_0$ satisfies $\neg Pre$, and $s_n$ satisfies $\neg Post$,
we say this is a negative trace, 
which means all the states in this trace should be negative states.  

Actually for an arbitrary trace there are also two other possibilities we have not mentioned yet.
One is a trace begins with a state $s_0$ that satisfies $Pre$ but end with a state $s_n$ that satisfies $\neg Post$,
this is a counterexample to disprove the program.
That means there is something wrong with at least one of precondition, loop condition, loop body or postcondition.
We need to find out what happens and update the program, 
after which we can reapply our approach to learn loop invariants.
The other case is a trace begin with a state $s_0$ that satisfies $\neg Pre$ but end with a state $s_n$ that satisfies $Post$.
Under this condition, we could not justify whether $s_0$ and $s_n$ satisfy invariants or not,
not to mention other states $\{s_1, s_2, ..., s_i, ... , s_{n-1}\}$.
The only thing we can ensure is this is an implication trace.
In total, we can have table.~\ref{LabelingTable}.


We categorize program states in $Body^*(S)$ into four sets:
$C_{S}$ which stands for counter-example trace, 
$P_{S}$ which stands for traces with positive labels, 
$N_{S}$ which stands for traces with negative labels 
and $U_{S}$ which stands for traces with unknown labels.

They can be judged according to Table~\ref{LabelingTable}: 
\begin{table}[htb]
\label{LabelingTable}
\centering
\caption{Trace Labeling Table}
\begin{tabular}[float]{|c|c|c|}
\hline
$s_0 \Rightarrow \Rightarrow s_n$ & $s_n \models Post$            & $s_n \models \neg Post$\\
\hline
$s_o \models Pre$                 & $Body^*(s_0) \in P_{S}$       & $Body^*(s_0) \in C_{S}$\\
\hline
$s_0 \models \neg Pre$            & $Body^*(s_0) \in U_{S}$       & $Body^*(s_0) \in N_{S}$\\
\hline
\end{tabular}
\end{table}



%\begin{table}[htb]
%\label{LabelingTable}
%\centering
%\caption{Trace Labeling Table}
%\begin{tabular}[float]{|c|c|c|}
%\hline
%$Trace\{s_0, s_1, ..., s_n\}$ & $s_n \in Post$            & $s_n \in \neg Post$\\
%\hline
%$s_o \in Pre$                 & $Trace \in P_{S}$         & $Trace \in C_{S}$\\
%\hline
%$s_0 \in \neg Pre$            & $Trace \in U_{S}$         & $Trace \in N_{S}$\\
%\hline
%\end{tabular}
%\end{table}


%\begin{itemize}
%    \item Set $C_{T}$ is $\{s \in Body^*(T) | s \in Pre \land s \Rightarrow s' \land s' \nin Post\}$;
%    \item Set $P_{T}$ is $\{s \in Body^*(T) | s \in Pre \land s \Rightarrow s' \land s' \in Post\}$;
%    \item Set $N_{T}$ is $\{s \in Body^*(T) | s \nin Pre \land s \Rightarrow s' \land s' \nin Post\}$;
%    \item Set $U_{T}$ is $\{s \in Body^*(T) | s \nin Pre \land s \Rightarrow s' \land s' \in Post\}$;
%\end{itemize}
%We remark that anytime a program state in $C_T$ is identified, a counter-example is found and \textsc{Zilu} reports that verification is failed immediately. 
%Otherwise, because $Inv$ must satisfy (1),(2) and (3), we know that $P_T \subseteq Inv$ and $N_T \inter Inv = \emptyset$. 
%The program states in $U_T$ may or not may be in $Inv$. 
%If we know that a program state $s \in U_T$ is in $Inv$, $Body^*(s) \subseteq Inv$.





%Among all these possibilities,
%we can get more samples than previous approaches can with executing program just once.
%So with the sample information, 
%the learner can learn an as good invariant as, if not better than, the previous approaches, 
%as it can utilize more information to do invariant learning task.
%This also implies our approach can get convergence faster than before.











\begin{example}
\end{example}

\begin{proposition}
Algorithm $activeLearning$ always eventually terminates. \hfill \qed
\end{proposition}


\begin{example}
\end{example}


% section sampling (end)
