%!TEX root = paper.tex

\section{Conclusion} % (fold)
\label{sec:related}
In this work, we propose an approach to improve loop invariant generation through guess-and-check. In particular, we propose to apply active learning techniques so as to learn accurate candidate loop invariants prior to the invariant checking phase. Furthermore, we propose a path-sensitive way of learning disjunctive loop invariants through classification. In principle, our approach can be extended to learn arbitrary mathematical classifiers using methods like SVM with kernel methods~\cite{svm:kernel}.
Nonetheless, %due to the limited verification capability of existing program verification techniques,
we focus on invariants in the form of polynomial inequalities or conjunctions/disjunctions of polynomial inequalities in our evaluation.
%Furthermore, we assume there is a bound $k$ on the number of clauses in the variant.
%In practice, we would expect (refer to empirical evidence in Section~\ref{sec:evaluations}) often $k$ is of a small value.
The experiment results show that our approach effectively learns loop invariant for proving a set of benchmark programs and complements existing approaches. 

As for future work, we are currently exploring methods for learning more expressive loop invariants as well as methods for discovering and synthesizing new features for classification. 

% \begin{table}
%     \begin{center}
%     \begin{tabular}{| l | c | c | l |}
%         \hline
%         Name & Inference Strategy & Inference Source \\
%         \hline
%         ABS & Eager & \\
%         \hline
%         CONS & Eager & \\
%         \hline
%         CEGAR & Lazy & Counterexample \\
%         \hline
%         INTER & Lazy & Counterexample \\
%         \hline
%         G\&C & Eager & Empirical \\
%         \hline
%         ABD & Lazy & Semantic \\
%         \hline
%         DAL & Eager & All Above \\
%         \hline
%     \end{tabular}
%     \end{center}
%     \caption{Existing Invariant Inference Approaches}
%     \label{tab:related}
% \end{table}

