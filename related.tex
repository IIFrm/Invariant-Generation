%!TEX root = paper.tex

\section{Conclusion and Related Work} % (fold)
\label{sec:related}
In this work, we propose a framework for improving loop invariant learning through active learning. 
In theory, our framework can be extended to learn arbitrary mathematical classifiers using methods like $\mathit{SVM}$ with kernel methods~\cite{svm:kernel}. 
Nonetheless, due to the limited verification capability of existing program verification techniques, 
we focus on invariants in the form of polynomial inequalities or conjunctions of polynomial inequalities in our evaluation.
%Furthermore, we assume there is a bound $k$ on the number of clauses in the variant.
%In practice, we would expect (refer to empirical evidence in Section~\ref{sec:evaluations}) often $k$ is of a small value.


% \begin{table}
%     \begin{center}
%     \begin{tabular}{| l | c | c | l |}
%         \hline
%         Name & Inference Strategy & Inference Source \\
%         \hline
%         ABS & Eager & \\
%         \hline
%         CONS & Eager & \\
%         \hline
%         CEGAR & Lazy & Counterexample \\
%         \hline
%         INTER & Lazy & Counterexample \\
%         \hline
%         G\&C & Eager & Empirical \\
%         \hline
%         ABD & Lazy & Semantic \\
%         \hline
%         DAL & Eager & All Above \\
%         \hline
%     \end{tabular}
%     \end{center}
%     \caption{Existing Invariant Inference Approaches}
%     \label{tab:related}
% \end{table}

This work is closely related and inspired by the work documented in~\cite{sharma2012interpolants,sharma2013verification,DBLP:conf/esop/0001GHALN13,sharma2014invariant}, where the same learn-and-check approach is applied. 
In~\cite{sharma2012interpolants}, the authors proposed to learn loop invariants based on $\mathit{SVM}$ classification. 
The required samples are generated through constraint solving. In~\cite{sharma2013verification}, the authors proposed to apply PAC learning. 
It has been demonstrated that their approach may learn invariants in the form of arbitrary boolean
combinations of polynomial inequalities. In~\cite{DBLP:conf/esop/0001GHALN13}, the authors developed a guess-and-check algorithm for generating algebraic equation invariants. 
In~\cite{sharma2014invariant}, the authors proposed a framework for generating invariant based on randomized search. 
In particular, their approach has two phases. 
The search phase uses randomized search to discover candidate invariants and the validate phase uses the checker to either prove or refute that the candidate is an actual invariant. 
\textsc{Zilu} complements the above approaches with active learning and selective sampling techniques, so as to reduce the need of checking, sometimes completely.

In addition, this work is related to a large body of approaches on loop invariant generation. The existing approaches can be mainly categorized as:
the ones based abstract interpretation~\cite{cousot1978automatic,mine2006octagon,cousot1979systematic,karr1976affine,vincent2009subpolyhedra}, 
the ones based on constraint synthesis~\cite{ashutosh2009invgen,michael2003linear,sumit2009constraint}, 
the ones based on counterexample guided abstraction refinement ($\mathit{CEGAR}$)~\cite{henzinger2003software,thomas2001slam,edmund2003counterexample}, 
the ones based on computing interpolation~\cite{kenneth2010lazy,thomas2004abstractions,kenneth2003interpolation,Kenneth2006lazy}, 
the ones based on abductive inference~\cite{isil2013inductive}, 
the ones based on guess-and-check~\cite{cormac2001houdini,ernst2007daikon}.

On one hand, invariant inference methods based on abstract interpretation and constraint synthesis
often try to generate all possible invariants in certain domain~\cite{mine2006octagon,vincent2009subpolyhedra,ashutosh2009invgen}, regardless of
whether they are useful to prove the Hoare triple or not.
As a result, the invariants inferred by them can be complex sometimes and yet fail to prove the program correctness.
On the other hand, other methods based on $\mathit{CEGAR}$, interpolation and abduction only generate those related to the program verification~\cite{isil2013inductive}.
Different from the above approaches, \textsc{Zilu} initially treats the given program as a black box and only collects relevant program states by executing the program. 
This step has no scalability issue. 
\textsc{Zilu} only opens up the black box after candidate invariants have converged. From this point of view, \textsc{Zilu} is lightweight compared to the above approaches. 
%In addition, \textsc{Zilu} adopts an active learning approach in an iterative refinement scheme.
%After generating the invariant, we check its correctness and refine it
%based on various new information (e.g., counter-examples, new samples)
%if it cannot prove the program correctness.
%Different from most of the existing refinement approaches,
%our method is driven by data samples
%rather than syntactic~\cite{cormac2001houdini} or semantic~\cite{ashutosh2009invgen,isil2013inductive} clues.
%Hence, it is more flexible and extensible to capture new types of invariants,
%and it is platform- and language-independent.
%Based on the needs in practice, under-approximation or over-approximation
%can also be applied to the data samples with ease.

% section related (end)
